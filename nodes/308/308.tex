\usepackage{amsthm}

\newtheorem{theorem}{Theorem}[chapter]
\newtheorem{lemma}           [theorem] {Lemma}   
\newtheorem{folg}           [theorem] {Folgerung}   

\newtheorem{frage}       [theorem] {Frage}   
\newtheorem{question}       [theorem] {Question}   
\newtheorem{aufgabe}       [theorem] {Aufgabe}   
\newtheorem{exercise}       [theorem] {Exercise}  

\newtheorem{proposition}     [theorem] {Proposition}  
\newtheorem{satz}     [theorem] {Satz}  
\newtheorem{fact}{Fact}
\newtheorem{definition}      [theorem] {Definition} 

\theoremstyle{definition} 
\newtheorem{bemerkung}     [theorem] {Bemerkung}  
\newtheorem{beispiel}       [theorem] {Beispiel}  
\newtheorem{example}       [theorem] {Example}  
\newtheorem*{example*} {Example}  
\newtheorem{notation}       [theorem] {Notation}  
\newtheorem*{Faust}[theorem]{Rule of Thumb}
\newtheorem*{Boxx}[theorem]{Concept}

Now we show that a~uniformly convergent sequence of continuous functions has to converge to a~continuous function.
\begin{Theorem}{}\label{th:uniform_conv}
 Let $I\subset \mathbb{K}$ and let $(f_n)_{n\in\mathbb{N}}$ be a sequence of continuous functions $f_n:I\to\mathbb{K}$ that uniformly converges to some $f:I\to\mathbb{K}$. 
Then $f$ is continuous.
\end{Theorem}

{\em Proof:} Let $\varepsilon>0$ and let $x_0\in I$.
Since $(f_n)_{n\in\mathbb{N}}$ converges uniformly to $f:I\to\mathbb{K}$, there exists some $N$ such that for all $n\geq N$ and $x\in I$ holds
\[|f(x)-f_n(x)|<\frac\varepsilon3.\]
Since $f_n$ is continuous on $I$, there exists some $\delta>0$ such that for all $x\in I$ with $|x-x_0|<\delta$ holds
\[|f_n(x_0)-f_n(x)|<\frac\varepsilon3.\]
Altogether, we then have
\[
\begin{aligned}
|f(x)-f(x_0)|=&\,|f(x)-f_n(x)+f_n(x)-f_n(x_0)+f_n(x_0)-f(x_0)|\\
\leq &\,|f(x)-f_n(x)|+|f_n(x)-f_n(x_0)|+|f_n(x_0)-f(x_0)|\\
< &\,\frac\varepsilon3+\frac\varepsilon3+\frac\varepsilon3=\varepsilon.
\end{aligned}
\]
Therefore, $f$ is continuous by the $\varepsilon$-$\delta$ criterion.
$\Box$

\begin{Remark}{}
The above result also gives a~sufficient characterisation for a~sequence of pointwisely convergent functions $(f_n)_{n\in\mathbb{N}}$ that are not uniformly convergent: If $(f_n)_{n\in\mathbb{N}}$ converges to a~discontinuous function,
then this sequence cannot be uniformly convergent. For instance, consider the sequence $(f_n)_{n\in\mathbb{N}}$ with $f_n:[0,1]\to\mathbb{R}$, $x\mapsto x^n$. The pointwise limit
is given by
\[f(x)=\lim_{n\to\infty}f_n(x)=\begin{cases}0&,\text{ if }x\in[0,1),\\1&,\text{ if }x=1,\end{cases}\]
which is discontinuous at $x=1$ though each $f_n$ is continuous. Therefore, this sequence cannot be uniformly convergent.
\end{Remark}

