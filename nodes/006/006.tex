\usepackage{amsthm}

\newtheorem{theorem}{Theorem}[chapter]
\newtheorem{lemma}           [theorem] {Lemma}   
\newtheorem{folg}           [theorem] {Folgerung}   

\newtheorem{frage}       [theorem] {Frage}   
\newtheorem{question}       [theorem] {Question}   
\newtheorem{aufgabe}       [theorem] {Aufgabe}   
\newtheorem{exercise}       [theorem] {Exercise}  

\newtheorem{proposition}     [theorem] {Proposition}  
\newtheorem{satz}     [theorem] {Satz}  
\newtheorem{fact}{Fact}
\newtheorem{definition}      [theorem] {Definition} 

\theoremstyle{definition} 
\newtheorem{bemerkung}     [theorem] {Bemerkung}  
\newtheorem{beispiel}       [theorem] {Beispiel}  
\newtheorem{example}       [theorem] {Example}  
\newtheorem*{example*} {Example}  
\newtheorem{notation}       [theorem] {Notation}  
\newtheorem*{Faust}[theorem]{Rule of Thumb}
\newtheorem*{Boxx}[theorem]{Concept}

%\subsection*{Injectivity, surjectivity, bijectivity, inverse}

\begin{Definition}[Injective{,} surjective and bijective]
A map $f: X \to Y$ is called
\begin{itemize}
 \item \emph{injective} if every fiber of $f$ has only one element: $x_1 \neq x_2 \Rightarrow f(x_1) \neq f(x_2)$.
 \item \emph{surjective} if $\mathrm{Ran}(f)=Y$. With quantifiers: $\forall y\in Y~ \exists x\in X \,:\, f(x)=y$.
 \item \emph{bijective} if $f$ is both injective and surjective.
\end{itemize}
\end{Definition}


\begin{example}
Define the function that maps each student to
her or his chair. This means that $X$ is the set of all students in the room,
and $Y$ the set of all chairs in the room.
\begin{itemize}
 \item well-defined: every student has a chair
 \item surjective: every chair is taken
 \item injective: on each chair there is no more than one student
 \item bijective: every student has his/her own chair, and no chair is empty
\end{itemize}
\end{example}


\begin{center}
     \begin{tikzpicture}[scale=1]
% Mengen
\draw (0,0)  arc[x radius = 1, y radius = 1.65, start angle= 80, end angle= 440];
\draw (3,0)  arc[x radius = 1, y radius = 1.65, start angle= 100, end angle=460];
% Punkte
\fill (2.8,-0.5) circle (0.07);
\fill (2.8,-1.5) circle (0.07);
% Beschriftung
\node at (4.2,-0.3) {\begin{footnotesize} not \end{footnotesize}};
\node at (4.2,-0.7) {\begin{footnotesize} injective \end{footnotesize}};
\node at (4.2,-1.3) {\begin{footnotesize} not \end{footnotesize}};
\node at (4.2,-1.7) {\begin{footnotesize} surjective \end{footnotesize}};
\node at (0,-2.5) {\begin{small} $X$ \end{small}};
\node at (3,-2.5) {\begin{small} $Y$ \end{small}};
% Pfeile
\draw[->, thick] (0.3,-0.5) -- (2.65,-0.5);
\draw[->, thick] (0.3,-1.75) -- (2.72,-0.6);
\end{tikzpicture}
\end{center}



\begin{Faust}{Surjective{,} injective{,} bijective}
A map  $f: X \rightarrow Y$ is
\begin{align*}
\text{surjective}\ &\Leftrightarrow\ \text{at each $y\in Y$ arrives at least one arrow} \\
&\Leftrightarrow\ f(X)=Y\\
&\Leftrightarrow\ \text{the equation $f(x)=y$ has for all $y\in Y$ a solution} \\
\\
\text{injective}\ &\Leftrightarrow\ \text{at each $y\in Y$ arrives at most one arrow}\\
&\Leftrightarrow\ \left( x_1 \neq x_2\quad \Rightarrow\quad f(x_1)\neq f(x_2) \right) \\
&\Leftrightarrow\ \left( f(x_1)=f(x_2)\quad \Rightarrow\quad x_1=x_2 \right) \\
&\Leftrightarrow\ \text{the equation $f(x)=y$ has for all $y\in f(X)$ 
a unique solution} \\
\\
\text{bijective}\ &\Leftrightarrow\ \text{at each $y\in Y$ arrives exactly one arrow} \\
&\Leftrightarrow\ \text{the equation $f(x)=y$ has for all $y\in Y$
a unique solution}
\end{align*}
\end{Faust}


Thus, if $f$ is bijective, there is a well defined inverse map 
\begin{align*}
f^{-1}:Y&\to X\\
       y &\mapsto x \text{ where } f(x)=y \,.
\end{align*}
Then $f$ is called \emph{invertible}
and $f^{-1}$ is called the \emph{inverse map} of $f$.

\begin{example}{} \label{Bsp:Umkehrabbildung}
    Consider the function $f: \mathbb{N} \rightarrow \{1, 4, 9, 16, \ldots\}$
given by $f(n) = n^2$. This is a bijective function.
The inverse map $f^{-1}$ is given by:
\begin{align*}
    f^{-1}:\lbrace1,4,9,16,25,\dots \rbrace &\rightarrow \mathbb{N} \\
m & \mapsto \sqrt{m} \\
\text{or: } n^2 &\mapsto n
\end{align*}
\begin{center}
     \begin{tikzpicture}[scale=0.8]
%% Mengen:
\draw (0,0) ellipse (2 and 5/2);
\draw (5,0) ellipse (2 and 5/2);
%% linke Menge:
\node at (-0.75,1.2) {$1$};
\fill (-0.5,1.2) circle (0.07);
\node at (0.25,0.6) {$4$};
\fill (0.5,0.6) circle (0.07);
\node at (-0.75,0) {$9$};
\fill (-0.5,0) circle (0.07);
\node at (0.15,-0.6) {$16$};
\fill (0.5,-0.6) circle (0.07);
\node at (-0.85,-1.2) {$25$};
\fill (-0.5,-1.2) circle (0.07);
\node at (0,-1.7) {$\dots$};
%% rechte Menge:
\fill (4.1,1.2) circle (0.07); 
\node at (4.35,1.2) {$1$}; 
\fill (5,0.72) circle (0.07); 
\node at (5.25,0.72) {$2$}; 
 \fill (3.9,0.15) circle (0.07); 
\node at (4.15,0.15) {$3$}; 
\fill (5.3,-0.6) circle (0.07); 
\node at (5.55,-0.6) {$4$};
\fill (4.9,-1.2) circle (0.07);
\node at (5.2,-1.2) {$5$};
\node at (5,-1.7) {$\dots$};
%% Beschriftung:
         \node at (0,1.8) {\begin{small} $f(\mathbb{N})$ \end{small}};
             \node at (5,1.8) {\begin{small} $\mathbb{N}$ \end{small}};
%% Pfeile:
\draw[->] (-0.5,1.2) -- (3.9,1.2);
\draw[->] (0.5,0.6) -- (4.8,0.72);
\draw[->] (-0.5,0) -- (3.7,0.15);
\draw[->] (0.5,-0.6) -- (5.1,-0.6);
\draw[->] (-0.5,-1.2) -- (4.7,-1.2);
\end{tikzpicture}
\end{center}
\end{example}

\begin{example}
    For a function $f:\mathbb{R}\rightarrow\mathbb{R}$, we can sketch the graph $\lbrace(x,f(x)): x\in X\rbrace$ in the $x$-$y$-plane:
\begin{figure}[htbp]
  \begin{minipage}[b]{0.5\textwidth}
   \centering
    \begin{tikzpicture}[scale=0.8, domain=-0.2:4] % Zeichenbereich
% Gitter
\draw[very thin,color=black!30] (-0.3,-0.3) grid (3.9,3.9);
% Achsen
\draw[->] (-0.2,0) -- (4.2,0) node[right] {$x$};
\draw[->] (0,-0.2) -- (0,4.2) node[above] {$y$};
% Funktionen
        \draw plot[samples=300] (\x,\x/2+1) node[right] at (4.2,3.5) {\begin{small} $f: \mathbb{R} \rightarrow \mathbb{R}$ \end{small}};
\draw plot[samples=300] (\x,\x/2+1) node[right] at (4.95,3) {\begin{small} $x \mapsto \frac{x}{2}+1$ \end{small}};
\end{tikzpicture}
  \end{minipage}
  \begin{minipage}[b]{0.5\textwidth}
  \centering
    \begin{tikzpicture}[scale=0.8, domain=-2.25:2.25] % Zeichenbereich
% Gitter
\draw[very thin,color=black!30] (-2.9,-1.1) grid (2.9,3.9);
% Achsen
\draw[->] (-2.45,0) -- (3.2,0) node[right] {$x$};
\draw[->] (0,-1.2) -- (0,4.5) node[above] {$y$};
% Funktionen
        \draw plot[samples=300] (\x,\x*\x-1) node[right] at (2.5,4.3) {\begin{small} $f: \mathbb{R} \rightarrow \mathbb{R}$ \end{small}};
\draw plot (\x,\x*\x-1) node[right] at (3.25,3.8) {\begin{small} $x \mapsto x^2-1$ \end{small}};
\end{tikzpicture}
  \end{minipage}
\end{figure}
\begin{center}
     \begin{tikzpicture}[scale=0.9, domain=-0.4:6.5] % Zeichenbereich
% Gitter
\draw[very thin,color=black!30] (-0.5,-1.2) grid (6.6,1.2);
% Achsen
\draw[->] (-0.5,0) -- (6.7,0) node[right] {$x$};
\draw[->] (0,-1.2) -- (0,1.5) node[above] {$y$};
% Funktionen
         \draw plot[samples=300] (\x,{sin(\x r)}) node[right] at (4.8,1.8) {\begin{small} $f: \mathbb{R} \rightarrow \mathbb{R}$ \end{small}};
\draw plot[samples=300] (\x,{sin(\x r)}) node[right] at (5.47,1.3) {\begin{small} $x \mapsto \sin x$ \end{small}};
\end{tikzpicture}
\end{center}
Which of the functions are injective, surjective or bijective?
\end{example}

These notions might seem a little bit off-putting, but we will use them so often that you need to get use to them. Maybe the video will help you as well.
