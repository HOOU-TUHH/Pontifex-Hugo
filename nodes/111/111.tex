\usepackage{amsthm}

\newtheorem{theorem}{Theorem}[chapter]
\newtheorem{lemma}           [theorem] {Lemma}   
\newtheorem{folg}           [theorem] {Folgerung}   

\newtheorem{frage}       [theorem] {Frage}   
\newtheorem{question}       [theorem] {Question}   
\newtheorem{aufgabe}       [theorem] {Aufgabe}   
\newtheorem{exercise}       [theorem] {Exercise}  

\newtheorem{proposition}     [theorem] {Proposition}  
\newtheorem{satz}     [theorem] {Satz}  
\newtheorem{fact}{Fact}
\newtheorem{definition}      [theorem] {Definition} 

\theoremstyle{definition} 
\newtheorem{bemerkung}     [theorem] {Bemerkung}  
\newtheorem{beispiel}       [theorem] {Beispiel}  
\newtheorem{example}       [theorem] {Example}  
\newtheorem*{example*} {Example}  
\newtheorem{notation}       [theorem] {Notation}  
\newtheorem*{Faust}[theorem]{Rule of Thumb}
\newtheorem*{Boxx}[theorem]{Concept}

\begin{Theorem}[Theorem of Heine-Borel]
For a~subset $C\subset\mathbb{K}$, the following statements are equivalent:
\begin{enumerate}[(i)]
 \item $C$ is compact;
 \item $C$ is bounded and closed.
\end{enumerate}
\end{Theorem}
{\em Proof:} \\
``(i)$\Rightarrow$(ii)'': Let $C$ be compact. 

Let $(a_n)_{n\in\mathbb{N}}$ be a convergent sequence in $\mathbb{K}$ with $a_n\in C$ and $a:=\lim_{n\rightarrow\infty} a_n \in\mathbb{K}$. 
Since $C$ is compact, there is a subsequence $(a_{n_k})_{k\in\mathbb{N}}$ such that $b:=\lim_{k\rightarrow\infty} a_{n_k}\in C$.
By Theorem~\ref{thm:convsubseq} we have $a=b\in C$.

Now assume that $C$ is unbounded. Then for all $n\in\mathbb{N}$, there exists some $a_n\in C$ with $|a_n|\geq n$. Consider an~arbitrary subsequence $(a_{n_k})_{k\in\mathbb{N}}$. Due to $|a_{n_k}|\geq n_k\geq k$, we have that $(a_{n_k})_{k\in\mathbb{N}}$ is unbounded, i.e., it cannot be convergent. This is also a~contradiction to compactness.\\
~\\
``(ii)$\Rightarrow$(i)'':
Let $C$ be closed and bounded. Let  $(a_n)_{n\in\mathbb{N}}$ be a~sequence in $C$. The boundedness of $C$ then implies the boundedness of $(a_n)_{n\in\mathbb{N}}$. By the Theorem of Bolzano-Weierstra\ss, there exists a~convergent subsequence  $(a_{n_k})_{k\in\mathbb{N}}$, i.e.,
\[\lim_{k\to\infty}a_{n_k}=a\]
for some $a\in\mathbb{K}$. For compactness, we now have to show that $a\in C$. However, this is guaranteed by the closedness of $C$.\hfill$\Box$
\begin{Remark}{}
    Taking a~closer look to the proof ``(i)$\Rightarrow$(ii)', we did not explicitly use that we are dealing with one of the spaces $\mathbb{R}$ or $\mathbb{C}$. Indeed, the implication that compact sets are bounded and closed holds true for all normed spaces. However, ``(ii)$\Rightarrow$(i)'' does not hold true in arbitrary normed spaces. Indeed, there are examples of normed spaces that have bounded and closed subsets which are not compact.
\end{Remark}
