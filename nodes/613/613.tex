\usepackage{amsthm}

\newtheorem{theorem}{Theorem}[chapter]
\newtheorem{lemma}           [theorem] {Lemma}   
\newtheorem{folg}           [theorem] {Folgerung}   

\newtheorem{frage}       [theorem] {Frage}   
\newtheorem{question}       [theorem] {Question}   
\newtheorem{aufgabe}       [theorem] {Aufgabe}   
\newtheorem{exercise}       [theorem] {Exercise}  

\newtheorem{proposition}     [theorem] {Proposition}  
\newtheorem{satz}     [theorem] {Satz}  
\newtheorem{fact}{Fact}
\newtheorem{definition}      [theorem] {Definition} 

\theoremstyle{definition} 
\newtheorem{bemerkung}     [theorem] {Bemerkung}  
\newtheorem{beispiel}       [theorem] {Beispiel}  
\newtheorem{example}       [theorem] {Example}  
\newtheorem*{example*} {Example}  
\newtheorem{notation}       [theorem] {Notation}  
\newtheorem*{Faust}[theorem]{Rule of Thumb}
\newtheorem*{Boxx}[theorem]{Concept}

Now we use integrals on unbounded domains to check whether series are convergent or not.
\begin{Theorem}[Integral Criterion for Series]
Let $f:[0,\infty)\to\mathbb{R}$ be monotonically decreasing and non-negative. Then the series
\[\sum_{k=0}^{\infty}f(k)\]
is convergent if and only if the integral
\[\int_{0}^{\infty}f(x)dx\]
converges.
In the case of convergence, the following estimate holds true:
\begin{equation}
\label{eq:intcrit} 
	0\leq \sum_{k=0}^\infty f(k) -\int_0^\infty f(x)~dx\leq f(0)
\end{equation}
\end{Theorem}
{\em Proof:} Monotonicity of $f$ implies that for $k-1\leq x\leq k$ holds $f(k)\leq f(x)\leq f(k-1)$. Monotonicity of the integral therefore leads to the inequality
\[f(k)=\int_{k-1}^kf(k)dx\leq \int_{k-1}^kf(x)dx\leq \int_{k-1}^kf(k-1)dx=f(k-1).\]
Therefore
\[\int_1^{n+2}f(x)dx\leq\sum_{k=1}^{n+1} f(k)\leq \int_0^{n+1}f(x)dx\leq \sum_{k=0}^{n} f(k).\]
Using this inequality, we can directly conclude that, if one of the limits (integral or sum) as $n\to\infty$ exists, then the other limit (sum or integral) also exists.

In case of convergence necessarily $(f(k))_{k\in\mathbb{N}_0}$ is a zero sequence. Therefore
\[0\leq \sum_{k=0}^n f(k) -\int_0^{n+1} f(x)~dx\leq \sum_{k=0}^n f(k) - \sum_{k=1}^{n+1}f(k) =f(0)-f(n+1)\]
implies (\ref{eq:intcrit}) for $n\rightarrow \infty$.
$\Box$

\begin{Remark}{}
Since the convergence of $\int_{a}^{\infty}f(x)dx$ does not depend on $a\in\mathbb{R}$, the above result can also be slightly generalised in a~way that
the convergence of the series $\sum_{k=a}^{\infty}f(k)$ for $a\in\mathbb{N}$ is equivalent to that of the integral $\int_{a}^{\infty}f(x)dx$.
\end{Remark}

\begin{example}
\begin{enumerate}[a)]
\item Using the results of Example~\ref{ex:infint} b), we see that
\[\sum_{k=1}^\infty\frac1{k^\alpha}\]
converges for $\alpha>1$ and is divergent for $\alpha\leq1$.
\item For $\alpha\in\mathbb{R}$ consider
\[\sum_{n=2}^\infty\frac1{n(\log(n))^\alpha}.\]
We have
\[
\begin{aligned}
\int_2^\infty\frac1{x(\log(x))^\alpha}~dx=&\lim_{n\to\infty}\begin{cases}\left.\frac{(\log(x))^{1-\alpha}}{1-\alpha}\right|_{x=2}^{x=n}&:\alpha\neq1,\\
\left.\log(\log(x))\right|_{x=2}^{x=n}&:\alpha=1\end{cases}\\
=&\begin{cases}-\frac{(\log(2))^{1-\alpha}}{1-\alpha}&:\alpha>1,\\
\infty&:\alpha\leq1.\end{cases}
\end{aligned}
\]
Therefore, the series \[\sum_{n=2}^\infty\frac1{n(\log(n))^\alpha}.\]
is convergent if and only if $\alpha>1$.
\end{enumerate}
\end{example}

