\usepackage{amsthm}

\newtheorem{theorem}{Theorem}[chapter]
\newtheorem{lemma}           [theorem] {Lemma}   
\newtheorem{folg}           [theorem] {Folgerung}   

\newtheorem{frage}       [theorem] {Frage}   
\newtheorem{question}       [theorem] {Question}   
\newtheorem{aufgabe}       [theorem] {Aufgabe}   
\newtheorem{exercise}       [theorem] {Exercise}  

\newtheorem{proposition}     [theorem] {Proposition}  
\newtheorem{satz}     [theorem] {Satz}  
\newtheorem{fact}{Fact}
\newtheorem{definition}      [theorem] {Definition} 

\theoremstyle{definition} 
\newtheorem{bemerkung}     [theorem] {Bemerkung}  
\newtheorem{beispiel}       [theorem] {Beispiel}  
\newtheorem{example}       [theorem] {Example}  
\newtheorem*{example*} {Example}  
\newtheorem{notation}       [theorem] {Notation}  
\newtheorem*{Faust}[theorem]{Rule of Thumb}
\newtheorem*{Boxx}[theorem]{Concept}

\begin{definition}[Conditional $A \rightarrow B$ (``If $A$ then $B$'')]
   $A \rightarrow B$ is only false if $A$ is true but $B$ is false.
\end{definition}
 
   \begin{equation}
   \mbox{ Truth table }\qquad 
    \begin{array}{cc|c}
     A & B & A \rightarrow B \\ \hline
     T & T& T\\
     T & F & F\\
     F & T & T\\
     F & F & T
    \end{array}
   \end{equation}


\begin{definition}[Biconditional  $A \leftrightarrow B$ (``$A$ if and only if $B$'')]
$A \leftrightarrow B$ is true if and only if $A \rightarrow B$ and $B \rightarrow A$ is true.
\end{definition}

   \begin{equation}
   \mbox{ Truth table }\qquad 
    \begin{array}{cc|c}
     A & B & A \leftrightarrow B \\ \hline
     T & T& T\\
     T & F & F\\
     F & T & F\\
     F & F & T
    \end{array}
   \end{equation}
   
If a conditional or biconditional is true,
we have a short notation for this that is used throughout   
the whole field of mathematics:
   
\begin{definition}[Implication and equivalence]
If $A \rightarrow B$ is true,
we call this an \emph{implication} and write:
	$$
	A \Rightarrow B  \,.
	$$
If $A \leftrightarrow B$ is true,
we call this an \emph{equivalence} and write:
	$$
	A \Leftrightarrow B  \,.
	$$
\end{definition}
 
This means that we speak of \emph{equivalence} of $A$ and $B$ if
the truth values in the truth table are exactly the same. For example, we have
	$$
		A \leftrightarrow B ~~ \Leftrightarrow
		~~
		(A \rightarrow B) \wedge (B \rightarrow A)
		\,.
	$$
 
   
Now one can ask:
\emph{What to do with truth-tables?} 
Let us show that $\neg B \rightarrow \neg A$ is the same as $A \rightarrow B$.
   \begin{equation}
   \mbox{ Truth table }\qquad 
    \begin{array}{cc|cc|c}
     A & B &  \neg A & \neg B &\neg B \rightarrow \neg A \\ \hline
     T & T &   F & F & T\\
     T & F &   F & T & F\\
     F & T &   T & F & T\\
     F & F &   T & T & T
    \end{array}
   \end{equation}
Therefore:
	$$
		(A \rightarrow B) ~~ \Leftrightarrow
		~~
		(\neg B \rightarrow \neg A) \,.
	$$
This is the \emph{proof by contraposition}:


``Assume that $B$ does not hold, then we can show that $A$ cannot hold as well''. Hence $A$ implies $B$.
%
\begin{Boxx}[Contraposition]
If  $A \Rightarrow B$, then also $\neg B \Rightarrow \neg A$.
\end{Boxx}

\begin{Faust}[Contraposition]
To get the contraposition $A\Rightarrow B$, you should exchange $A$ and $B$ and set a $\neg$-sign in front of both:
$\ \neg B\Rightarrow\neg A$.

It is clear: The contraposition of the contraposition is again $A\Rightarrow B$.
\end{Faust}

The contraposition is an example of a \emph{deduction rule}, which basically
tells us how to get new true proposition from other true propositions. The most important deduction rules are given just by using the implication.

\begin{Boxx}[Modus ponens]
If $A \Rightarrow B$ and $A$ is true, then also $B$ is true.
\end{Boxx}

\begin{Boxx}[Chain syllogism]
 If $A \Rightarrow B$ and $B \Rightarrow C$, then also $A \Rightarrow C$.
\end{Boxx}

\begin{Boxx}[Reductio ad absurdum]
 If $A \Rightarrow B$ and $A \Rightarrow \neg B$, then $\neg A$ is true.
\end{Boxx}

One can easily prove these rules by truth tables.
However, here we do not state every deduction in this formal manner. We may still use deduction in the intuitive way as well. Try it here:

\begin{exercise}{}
Let \emph{``All birds can fly''} be a true proposition (axiom). 
Are the following deductions correct?
 			\begin{itemize}
 				\item If Seagulls are birds, then Seagulls can fly.
 				\item If Penguins are birds, then Penguins can fly.
 				\item If Butterflies are birds, then Butterflies can fly.
 				\item If Butterflies can fly, then Butterflies are birds.
			\end{itemize} 			 
\end{exercise}
