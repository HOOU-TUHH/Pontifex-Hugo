\usepackage{amsthm}

\newtheorem{theorem}{Theorem}[chapter]
\newtheorem{lemma}           [theorem] {Lemma}   
\newtheorem{folg}           [theorem] {Folgerung}   

\newtheorem{frage}       [theorem] {Frage}   
\newtheorem{question}       [theorem] {Question}   
\newtheorem{aufgabe}       [theorem] {Aufgabe}   
\newtheorem{exercise}       [theorem] {Exercise}  

\newtheorem{proposition}     [theorem] {Proposition}  
\newtheorem{satz}     [theorem] {Satz}  
\newtheorem{fact}{Fact}
\newtheorem{definition}      [theorem] {Definition} 

\theoremstyle{definition} 
\newtheorem{bemerkung}     [theorem] {Bemerkung}  
\newtheorem{beispiel}       [theorem] {Beispiel}  
\newtheorem{example}       [theorem] {Example}  
\newtheorem*{example*} {Example}  
\newtheorem{notation}       [theorem] {Notation}  
\newtheorem*{Faust}[theorem]{Rule of Thumb}
\newtheorem*{Boxx}[theorem]{Concept}

%\subsection*{Image and preimage}

For every well-defined map $f: X\to Y$ and $A\subset X$, $B \subset Y$ we are interested in the following sets:
\begin{Definition}{} 
Let $f: X\rightarrow Y$ be a function
and $A\subset X$ and $B\subset Y$ some sets.

%\begin{figure}[htbp]
  \begin{minipage}[l]{0.5\textwidth}
    \begin{center}
$f(A):= \lbrace f(x): x\in A\rbrace$ \\ is called the \emph{image} 
of $A$ under $f$.
\end{center}
  \end{minipage}
  \begin{minipage}[r]{0.5\textwidth}
  \flushright 
     \begin{tikzpicture}[scale=0.8]
% Ellipsen
\draw (-0.3,0) ellipse (3/2 and 2);
\draw (5,0) ellipse (3/2 and 2);
\draw[fill=black!15] (0,0) ellipse (0.75 and 1);
\draw[fill=black!15] (5,0) ellipse (1 and 1/2);
% Pfeile
\draw[->] (2,0) -- (3,0);
\draw (0,1) -- (5,1/2);
\draw (0,-1) -- (5,-1/2);
\draw[->] (2,1.1) to[out=45, in=135] (3,1.1);
% Beschriftungen
\node at (2.5,1.6) {\begin{footnotesize} $f$ \end{footnotesize}};
\node at (-0.6,1.5) {\begin{footnotesize} $X$ \end{footnotesize}};
\node at (5.3,1.5) {\begin{footnotesize} $Y$ \end{footnotesize}};
\node (A) at (0,0) {\begin{footnotesize} $A$ \end{footnotesize}};
\node (B) at (5,0) {\begin{footnotesize} $f(A)$ \end{footnotesize}};
\end{tikzpicture}
  \end{minipage}
%\end{figure}

~\\

%\begin{figure}[htbp]
  \begin{minipage}[l]{0.5\textwidth}
    \begin{center}
    $f^{-1}(B):= \lbrace x\in X: f(x) \in B \rbrace$ \\ is called
    the \emph{preimage} of $B$ under $f$.
\end{center}
  \end{minipage}
  \begin{minipage}[r]{0.5\textwidth}
  \flushright
\begin{tikzpicture}[scale=0.8]
% Ellipsen
\draw (-0.3,0) ellipse (3/2 and 2);
\draw (5,0) ellipse (3/2 and 2);
\draw[fill=black!15] (0,0) ellipse (0.9 and 1);
\draw[fill=black!15] (5,0) ellipse (1 and 1/2);
% Pfeile
\draw[<-] (2,0) -- (3,0);
\draw (0,1) -- (5,1/2);
\draw (0,-1) -- (5,-1/2);
%\draw[->] (2,1.1) to[out=45, in=135] (3,1.1);
% Beschriftungen
\node at (2.5,1.6) {\begin{footnotesize}  \end{footnotesize}};
\node at (-0.6,1.5) {\begin{footnotesize} $X$ \end{footnotesize}};
\node at (5.3,1.5) {\begin{footnotesize} $Y$ \end{footnotesize}};
\node (A) at (0,0) {\begin{footnotesize} $f^{-1}(B)$ \end{footnotesize}};
\node (B) at (5,0) {\begin{footnotesize} $B$ \end{footnotesize}};
\end{tikzpicture}
  \end{minipage}
%\end{figure}
\end{Definition}
%
Note that the preimage can also be the empty set if none of the
elements in $B$ are ``hit'' by the map.

To describe the behaviour of a map,
the following sets are very important:

\begin{Definition}[Range and fiber]
Let $f: X\rightarrow Y$ be a map. Then
\begin{align*}
 \mathrm{Ran}(f) &:= f(X) = \{ f(x) : x \in X \} 
\end{align*}
is called the \emph{range} of $f$.
For each $y\in Y$ the set
\begin{align*}
 f^{-1}(\{y \}) &:= \{ x \in X : f(x) = y \} 
\end{align*}
is called a \emph{fiber} of $f$.
\end{Definition}
