\usepackage{amsthm}

\newtheorem{theorem}{Theorem}[chapter]
\newtheorem{lemma}           [theorem] {Lemma}   
\newtheorem{folg}           [theorem] {Folgerung}   

\newtheorem{frage}       [theorem] {Frage}   
\newtheorem{question}       [theorem] {Question}   
\newtheorem{aufgabe}       [theorem] {Aufgabe}   
\newtheorem{exercise}       [theorem] {Exercise}  

\newtheorem{proposition}     [theorem] {Proposition}  
\newtheorem{satz}     [theorem] {Satz}  
\newtheorem{fact}{Fact}
\newtheorem{definition}      [theorem] {Definition} 

\theoremstyle{definition} 
\newtheorem{bemerkung}     [theorem] {Bemerkung}  
\newtheorem{beispiel}       [theorem] {Beispiel}  
\newtheorem{example}       [theorem] {Example}  
\newtheorem*{example*} {Example}  
\newtheorem{notation}       [theorem] {Notation}  
\newtheorem*{Faust}[theorem]{Rule of Thumb}
\newtheorem*{Boxx}[theorem]{Concept}

\begin{Theorem}[Continuous functions defined on a compact set]\label{thm:fctcompact}
Let $I\subset\mathbb{K}$ be compact and let $f:I\to\mathbb{K}$ be continuous. Then $f(I)$ is compact. In particular, by the Theorem of Heine-Borel,
$f(I)$ is bounded and closed. If further $f(I) \subset \mathbb{R}$, so that there exist $x^+,x^-\in I$ such that
\[f(x^+)=\max\{f(x)\,:\,x\in I\},\qquad f(x^-)=\min\{f(x)\,:\,x\in I\}.\]
\end{Theorem}
\begin{proof}
Let $(y_n)_{n\in\mathbb{N}}$ be a sequence in $f(I)$. Then for each $n\in\mathbb{N}$ there is an $x_n\in I$ such that $y_n=f(x_n)$. 
Since $I$ is compact, there exists a subsequence $(x_{n_k})_{k\in\mathbb{N}}$ that converges to some $x\in I$. 
Now, since $f$ is continuous, we have $$\lim_{k\rightarrow \infty} y_{n_k} = \lim_{k\rightarrow \infty} f(x_{n_k}) = f(x)=:y \in f(I).$$
Hence we found a subsequence $(y_{n_k})_{k\in\mathbb{N}}$ of $(y_n)_{n\in\mathbb{N}}$ that converges in $f(I)$. Therefore $f(I)$ is compact.
\end{proof}

% As a~consequence of the above result, we have for compact $I$ holds that the set of continuous functions
% \[C(I)=\{f: I\to\mathbb{R}\,:\,f\text{ continuous on $I$}\}\]
% fulfills $C(I)\subset\mathcal{B}(I)$. Since sums and scalar multiples of continuous functions are again continuous by Theorem~\ref{eq:sumscont}, $C(I)$ is even a~subspace of $\mathcal{B}(I)$ (provided that $I$ is compact!).

