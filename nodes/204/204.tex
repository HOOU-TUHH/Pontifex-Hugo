\usepackage{amsthm}

\newtheorem{theorem}{Theorem}[chapter]
\newtheorem{lemma}           [theorem] {Lemma}   
\newtheorem{folg}           [theorem] {Folgerung}   

\newtheorem{frage}       [theorem] {Frage}   
\newtheorem{question}       [theorem] {Question}   
\newtheorem{aufgabe}       [theorem] {Aufgabe}   
\newtheorem{exercise}       [theorem] {Exercise}  

\newtheorem{proposition}     [theorem] {Proposition}  
\newtheorem{satz}     [theorem] {Satz}  
\newtheorem{fact}{Fact}
\newtheorem{definition}      [theorem] {Definition} 

\theoremstyle{definition} 
\newtheorem{bemerkung}     [theorem] {Bemerkung}  
\newtheorem{beispiel}       [theorem] {Beispiel}  
\newtheorem{example}       [theorem] {Example}  
\newtheorem*{example*} {Example}  
\newtheorem{notation}       [theorem] {Notation}  
\newtheorem*{Faust}[theorem]{Rule of Thumb}
\newtheorem*{Boxx}[theorem]{Concept} 

\begin{Theorem}[Cauchy Criterion]
A~series $\displaystyle\sum_{k=1}^\infty a_k$ in $\mathbb{R}$ is convergent if and only if for all $\varepsilon>0$, there exists some $N$ such that for all $n\geq m\geq N$ holds
\[\left|\sum_{k=m}^n a_k\right|<\varepsilon.\]
\end{Theorem}

{\em Proof:} By the theorems on completeness and convergence of cauchy sequences, a~series converges if and only if the sequence $(s_{n})_{n\in\mathbb{N}}$ of partial sums is a~Cauchy sequence.\\
On the other hand, for $n\geq m$, we have
\[\left|s_n-s_{m-1}\right|=\left|\sum_{k=m}^n a_k\right|.\]
Therefore, the Cauchy criterion is really equivalent to the fact that $(s_{n})_{n\in\mathbb{N}}$ is a~Cauchy sequence in $\mathbb{R}$.\hfill$\Box$

%  \begin{Remark}{}
%  %For incomplete spaces, the Cauchy criterion is only necessary (but not sufficient) for convergence of a~series. This is a~consequence of the fact that %the Cauchy criterion is equivalent to the fact that
%  %the sequence of partial sums is a~Cauchy sequence.\\
%  Reconsidering the example at the very beginning of this chapter, the divergence of this sequence can be directly verified be employing the Cauchy criterion.
%  \end{Remark}
As a corollary, we can formulate the following criterion.

\begin{Theorem}[Necessary criterion for convergence of series]
\label{eq:conv0}
Let \[\sum_{k=1}^\infty a_k\] be a~convergent series in $\mathbb{R}$. Then $(a_n)_{n \in \mathbb{N}}$ is convergent with
\[\lim_{n\to\infty}a_n=0.\]
\end{Theorem}
{\em Proof:}
Since the series converges, the Cauchy criterion implies that for all $\varepsilon>0$, there exists some $N$ such that for all $n\geq m\geq N$ holds
\[\left|\sum_{k=m}^n a_k\right|<\varepsilon.\]
Now considering the special case $n=m$, we have that for all $n\geq N$ holds
\[|a_n|<\varepsilon.\]
However, this is nothing but convergence of $(a_n)_{n\in\mathbb{N}}$ to zero.
\hfill$\Box$
\begin{Remark}{}
The zero convergence of $(a_n)_{n\in\mathbb{N}}$ is by far not sufficient for convergence. We have already seen that the harmonic series diverges though the sequence $(a_n)_{n\in\mathbb{N}}=(\frac1n)_{n\in\mathbb{N}}$ converges to zero.
\end{Remark}

