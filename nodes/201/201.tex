\usepackage{amsthm}

\newtheorem{theorem}{Theorem}[chapter]
\newtheorem{lemma}           [theorem] {Lemma}   
\newtheorem{folg}           [theorem] {Folgerung}   

\newtheorem{frage}       [theorem] {Frage}   
\newtheorem{question}       [theorem] {Question}   
\newtheorem{aufgabe}       [theorem] {Aufgabe}   
\newtheorem{exercise}       [theorem] {Exercise}  

\newtheorem{proposition}     [theorem] {Proposition}  
\newtheorem{satz}     [theorem] {Satz}  
\newtheorem{fact}{Fact}
\newtheorem{definition}      [theorem] {Definition} 

\theoremstyle{definition} 
\newtheorem{bemerkung}     [theorem] {Bemerkung}  
\newtheorem{beispiel}       [theorem] {Beispiel}  
\newtheorem{example}       [theorem] {Example}  
\newtheorem*{example*} {Example}  
\newtheorem{notation}       [theorem] {Notation}  
\newtheorem*{Faust}[theorem]{Rule of Thumb}
\newtheorem*{Boxx}[theorem]{Concept}

%\subsection{The Geometric and Harmonic Series}
Before we give some criteria for the convergence of series, we first present the probably most important series and analyze their convergence.
\begin{example}\label{ex:geomharm}
 \begin{enumerate}[(a)]
  \item For $q\in\mathbb{K}$, the \emph{geometric series}
\[ \sum_{k=0}^\infty q^k\]
is convergent if and only if $|q|<1$.

{\em Proof:} We can show that the $n$-th partial sum is given by
\[s_n=\sum_{k=0}^nq^k=\begin{cases}\frac{1-q^{n+1}}{1-q}&:\text{ if }q\neq1,\\n+1&:\text{ if }q=1.\end{cases}\]
Hence, $(s_n)_{n\in\mathbb{N}}$ is convergent if and only if $|q|<1$. In this case we have
\[ \sum_{k=0}^\infty q^k=\lim_{n\to\infty}s_n=\lim_{n\to\infty}\frac{1-q^{n+1}}{1-q}=\frac{1}{1-q}.\]
\item The \emph{harmonic series}
\[ \sum_{k=1}^\infty \frac1k\]
is divergent to $+\infty$.\\
{\em Proof:} If we construct some unbounded subsequence $(s_{n_l})_{l\in\mathbb{N}}$, the divergence of the harmonic series is proven (since
it is monotonically increasing). 
Indeed, we now show the unboundedness of the subsequence
$(s_{2^l})_{l\in\mathbb{N}}$: First, observe that
\[s_{2^l}=s_1+(s_2-s_1)+(s_4-s_2)+(s_8-s_4)+\ldots+(s_{2^l}-s_{2^{l-1}})=s_1+\sum_{j=1}^l(s_{2^j}-s_{2^{j-1}}).\]
Now we take a~closer look to the number $s_{2^j}-s_{2^{j-1}}$: By definition of $s_n$, we have
\[s_{2^j}-s_{2^{j-1}}=\sum_{k=2^{j-1}+1}^{2^j}\frac1k>\sum_{k=2^{j-1}+1}^{2^j}\frac1{2^{j}}= 2^{j-1}\frac1{2^{j}}=\frac12.\]
The inequality in the above formula holds true since every summand is replaced by the smallest summand $\frac1{2^{j}}$. The second last equality sign then comes from the fact that the number
$\frac1{2^{j}}$ is summed up $2^{j-1}$-times. Now using this inequality together with the above sum representation for $s_{2^l}$, we obtain
\[s_{2^l}=s_1+\sum_{j=1}^l(s_{2^j}-s_{2^{j-1}})>1+\sum_{j=1}^l\frac12=1+\frac{l}2.\]
As a consequence, the subsequence $(s_{2^l})_{l\in\mathbb{N}}$ is unbounded.\hfill$\Box$
\item For $\alpha>1$, the sequence
\[ \sum_{k=1}^\infty \frac1{k^\alpha}\]
is convergent.\\
{\em Proof:} The sequence of partial sums is strictly monotonically increasing due to
\[s_{n+1}-s_n=\frac1{(n+1)^\alpha}\geq0.\]
Therefore, by theorems about convergence of sequences, the convergence of $(s_n)_{n\in\mathbb{N}}$ is shown if we find some 
bounded subsequence $(s_{n_j})_{j\in\mathbb{N}}$. Again we use the representation for $s_{2^j}-s_{2^{j-1}}$
as in example b). We can estimate
\[
\begin{aligned}
s_{2^j}-s_{2^{j-1}}=&\,\sum_{k=2^{j-1}+1}^{2^j}\frac1{k^\alpha}<\sum_{k=2^{j-1}+1}^{2^j}\frac1{(2^{j-1}+1)^\alpha}\\
=&\,\frac{2^{j-1}}{(2^{j-1}+1)^\alpha}< \frac{2^{j-1}}{(2^{j-1})^\alpha}=\left(\frac{2}{2^\alpha}\right)^{j-1}=\left(\frac{1}{2^{\alpha-1}}\right)^{j-1},
\end{aligned}\]
so we have $s_{2^j}-s_{2^{j-1}}<q^{j-1}$ for $q=\frac{1}{2^{\alpha-1}}$ and, due to $\alpha>1$, it holds that $0<q<1$. Using that $s_1=1=q^0$, we obtain
\[s_{2^l}=s_1+\sum_{j=1}^l(s_{2^j}-s_{2^{j-1}})<1+\sum_{j=0}^{l-1}q^j=1+\frac{1-q^l}{1-q}<1+\frac{1}{1-q}.\]
Hence, the sequence $(s_{2^l})_{l\in\mathbb{N}}$ is bounded. This implies the desired result.\hfill$\Box$
\end{enumerate}
\end{example}
\begin{Remark}{}
Except for the first example, we have not computed the limits of the other stated convergent series. 
We only proved existence or non-existence of limits. Indeed, the computation of limits of series is, in general, a~very difficult issue and is not possible in many cases.\\
The function
\[\zeta(\alpha)=\sum_{k=1}^\infty \frac1{k^\alpha}\]
is very popular in analytic number theory under the name \emph{Riemann Zeta Function}. In b) and c), we have implicitly proven that $\zeta(\cdot)$ is defined on the interval $(1,\infty)$ and has a pole at 1. This function is subject of the {\em Riemann hypothesis} which is one of the most important unsolved problems in modern mathematics. Some known values of the Zeta function are (without proof)
\[\sum_{k=1}^\infty \frac1{k^2}=\zeta(2)=\frac{\pi^2}6,\qquad
\sum_{k=1}^\infty \frac1{k^4}=\zeta(4)=\frac{\pi^4}{90},\qquad
\sum_{k=1}^\infty \frac1{k^6}=\zeta(6)=\frac{\pi^6}{945}.
\]
\end{Remark}{}

