\usepackage{amsthm}

\newtheorem{theorem}{Theorem}[chapter]
\newtheorem{lemma}           [theorem] {Lemma}   
\newtheorem{folg}           [theorem] {Folgerung}   

\newtheorem{frage}       [theorem] {Frage}   
\newtheorem{question}       [theorem] {Question}   
\newtheorem{aufgabe}       [theorem] {Aufgabe}   
\newtheorem{exercise}       [theorem] {Exercise}  

\newtheorem{proposition}     [theorem] {Proposition}  
\newtheorem{satz}     [theorem] {Satz}  
\newtheorem{fact}{Fact}
\newtheorem{definition}      [theorem] {Definition} 

\theoremstyle{definition} 
\newtheorem{bemerkung}     [theorem] {Bemerkung}  
\newtheorem{beispiel}       [theorem] {Beispiel}  
\newtheorem{example}       [theorem] {Example}  
\newtheorem*{example*} {Example}  
\newtheorem{notation}       [theorem] {Notation}  
\newtheorem*{Faust}[theorem]{Rule of Thumb}
\newtheorem*{Boxx}[theorem]{Concept}
\begin{Definition}[Supremum and infimum]
Let $M\subset \mathbb{R}$ be a~set.
\begin{enumerate}[(a)]
 \item  A real number $s$
     is called the \emph{supremum} of $M$ if:
 	\begin{itemize}
 		\item $x\leq s$ for all $x \in M$,
 		\item for all $\varepsilon > 0$ there is an $x \in M$ with $s-\varepsilon < x$.
 	\end{itemize}
In this case we write $s=\sup M$.
 \item  A real number $l$
     is called the \emph{infimum} of $M$ if:
 	\begin{itemize}
 		\item $x\geq l$ for all $x \in M$,
 		\item for all $\varepsilon > 0$ there is an $x \in M$ with $l + \varepsilon > x$.
 	\end{itemize}
In this case we write $l=\inf M$.
\item We further define
\begin{itemize}
 \item $\sup M=\infty$ if $M$ is not bounded from above;
 \item $\inf M=-\infty$ if $M$ is not bounded from below;
 \item $\sup \emptyset =-\infty$;
 \item $\inf \emptyset  =\infty$.
\end{itemize}
\end{enumerate}
\end{Definition}

\begin{Remember}[Sup and Inf]
 The infimum is the greatest lower bound and the supremum is the lowest upper bound.
\end{Remember}

\begin{example}
\begin{enumerate}[(a)]
\item $\sup[0,1]=1$, $\;\;\;\inf[0,1]=0$;
	\whiteskipsmall
\item $\sup (0,1) = 1$, $\;\;\;\;\inf (0,1) = 0$;
	\whiteskipsmall
    \item $\sup\{\frac1n\;:\;n\in\mathbb{N}\}=1$, $\qquad\qquad\;\;\;\,\inf\{\frac1n\;:\;n\in\mathbb{N}\}=0$;
	\whiteskipsmall
    \item $\sup\{x\in\mathbb{Q}\;:\;x^2<2\}=\sqrt{2}$, $\qquad\inf\{x\in\mathbb{Q}\;:\;x^2<2\}=-\sqrt{2}$;
\end{enumerate}
\end{example}

\begin{Remark}{Difference between sup and max (resp.\ inf and min)}
In contrast to the maximum, the supremum does not need to belong to the respective set. For instance, we have $1=\sup(0,1)$, but $\max(0,1)$ does not exist.
The analogous statement holds true for $\inf$ and $\min$. However, we can make the following statement: If $\max M$ ($\min M$) exists, then $\max M=\sup M$ ($\min M=\inf M$).
\end{Remark}
