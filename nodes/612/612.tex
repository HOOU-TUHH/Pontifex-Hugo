\usepackage{amsthm}

\newtheorem{theorem}{Theorem}[chapter]
\newtheorem{lemma}           [theorem] {Lemma}   
\newtheorem{folg}           [theorem] {Folgerung}   

\newtheorem{frage}       [theorem] {Frage}   
\newtheorem{question}       [theorem] {Question}   
\newtheorem{aufgabe}       [theorem] {Aufgabe}   
\newtheorem{exercise}       [theorem] {Exercise}  

\newtheorem{proposition}     [theorem] {Proposition}  
\newtheorem{satz}     [theorem] {Satz}  
\newtheorem{fact}{Fact}
\newtheorem{definition}      [theorem] {Definition} 

\theoremstyle{definition} 
\newtheorem{bemerkung}     [theorem] {Bemerkung}  
\newtheorem{beispiel}       [theorem] {Beispiel}  
\newtheorem{example}       [theorem] {Example}  
\newtheorem*{example*} {Example}  
\newtheorem{notation}       [theorem] {Notation}  
\newtheorem*{Faust}[theorem]{Rule of Thumb}
\newtheorem*{Boxx}[theorem]{Concept}

So far, we have integrated bounded functions on compact intervals $[a,b]$. In this part we skip these two assumptions by extending the integral notion to functions that may have a pole and/or are defined on unbounded intervals.
\subsection{Unbounded Interval}
\begin{Definition}[Integration on unbounded intervals]
Let $f:[a,\infty)\to\mathbb{R}$ be a~function with the property that for all $b>a$ the restriction of $f$ to $[a,b]$ belongs to $\mathcal{R}([a,b])$. If
\[\lim_{b\to\infty}\int_a^bf(x)dx\]
exists, then we say that
\[\int_a^\infty f(x)dx\]
is \emph{convergent}. Otherwise, we speak of \emph{divergence}.
\end{Definition}
\begin{example}\label{ex:infint}
\begin{enumerate}[a)]
\item For integrating the function $\exp(-x)$ on the interval $[0,\infty)$, we compute
\[
\begin{aligned}
\int_0^\infty\exp(-x)dx=&\lim_{b\to\infty}\int_0^b\exp(-x)dx\\=&\lim_{b\to\infty}\left.-\exp(-x)\right|_{x=0}^{x=b}\\
=&1-\lim_{b\to\infty}\exp(-b)=1.
\end{aligned}
\]
\item For $\alpha>0$, consider
\[\int_1^\infty \frac1{x^\alpha}dx.\]
We know that for $b>1$ holds 
\[
\int_1^b\frac1{x^\alpha}dx=\begin{cases}\left.\frac1{1-\alpha}\frac1{x^{\alpha-1}}\right|_{x=1}^{x=b}&:\alpha\neq1,\\\left.\log(x)\right|_{x=1}^{x=b}&:\alpha=1.
\end{cases}\]
Since $\lim_{b\to\infty}\log(b)=\infty$ and
\[\lim_{b\to\infty}\frac1{x^{\alpha-1}}=\begin{cases}0&:\alpha>1,\\\infty&:\alpha<1,\end{cases}\]
we have
\[\int_1^\infty \frac1{x^\alpha}dx=\begin{cases}\frac1{\alpha-1}&:\alpha>1,\\\infty&:\alpha\leq1.\end{cases}\]
\end{enumerate}
\end{example}
It is straightforward to define the integral of a function defined on some interval unbounded from below by
\[\int_{-\infty}^af(x)dx=\lim_{b\to-\infty}\int_{b}^af(x)dx.\]
We now define the integral of functions defined on the whole real axis.




\begin{Definition}{}
Let $f:\mathbb{R}\to\mathbb{R}$ be a~function with the property that for all $a,b\in\mathbb{R}$ the restriction of $f$ to $[a,b]$ belongs to $\mathcal{R}([a,b])$. If there exists some $c\in\mathbb{R}$ such that both integrals
\[\int_c^\infty f(x)dx,\qquad \int_{-\infty}^c f(x)dx\]
exist, then we say that
\[\int_{-\infty}^\infty f(x)dx\]
is \emph{convergent}. Otherwise, we speak of \emph{divergence}.
In case of convergence we set
\[\int_{-\infty}^\infty f(x)dx=\int_{-\infty}^c f(x)dx+\int_c^\infty f(x)dx.\]
\end{Definition}
We remark without proof that the above definition is independent of $c$.

\begin{example}
\begin{enumerate}[a)]
\item Consider
\[
\begin{aligned}
\int_{-\infty}^\infty\frac1{1+x^2}dx=&
\int_{-\infty}^0\frac1{1+x^2}dx+\int_0^\infty \frac1{1+x^2}dx\\
=&\lim_{a\to-\infty}\int_{a}^0\frac1{1+x^2}dx+\lim_{b\to\infty}\int_{0}^b\frac1{1+x^2}dx\\
=&\lim_{a\to-\infty}\left.\arctan(x)\right|_{x=a}^{x=0}+\lim_{b\to\infty}\left.\arctan(x)\right|_{x=0}^{x=b}\\
=&-\lim_{a\to-\infty}\arctan(a)+\lim_{b\to\infty}\arctan(b)\\
=&\frac\pi2+\frac\pi2=\pi.
\end{aligned}
\]
\item The integral
\[
\int_{-\infty}^\infty xdx
\]
diverges since both integrals
\[
\begin{aligned}
\int_{-\infty}^0 xdx&=\lim_{a\to-\infty}\int_{a}^0 xdx=\lim_{a\to-\infty}\left.\frac12x^2\right|_{x=a}^{x=0}=-\infty\\
\int_{0}^\infty xdx&=\lim_{b\to\infty}\int_{0}^b xdx=\lim_{b\to\infty}\left.\frac12x^2\right|_{x=0}^{x=b}=\infty
\end{aligned}
\]
diverge. This example shows that the convergence of $\int_{-\infty}^\infty f(x)dx$ is not equivalent to the existence of the limit
\[\lim_{a\to\infty}\int_{-a}^af(x)dx.\]
However, in case of convergence, the integral $\int_{-\infty}^\infty f(x)dx$ coincides with the above limit.
\end{enumerate}
\end{example}
The majorant criterion for series says that if the absolut values of the addends of a given series can be bounded from above 
by the addends of a convergent series, then (absolut) convergence of the given series can be concluded. 

Conversely, the minorant criterion for series says that if the addends of a given series can be bounded from below by the addends 
of a series that diverges to $+\infty$, then also the given series diverges to $+\infty$.

Analogue criteria hold true for integrals. We skip the proofs since they are totally analogous to those of the majorant and minorant criteria.

\begin{Theorem}{}
Let $f,g:[a,\infty)\to\mathbb{R}$ such that for all $b\in [a,\infty)$, the restrictions of $f$ and $g$ to $[a,b]$ are Riemann-integrable.
\begin{enumerate}[(i)]
\item If $|f(x)|\leq g(x)$ for all $x\in [a,\infty)$ and $\int_{a}^\infty g(x)dx$ converges, then also $\int_{a}^\infty f(x)dx$ converges
and it holds that
\[
\left|\int_{a}^\infty f(x)dx\right|\leq \int_{a}^\infty |f(x)|dx\leq \int_{a}^\infty g(x)dx.
\]
\item If $g(x)\leq f(x)$ for all $x\in [a,\infty)$ and $\int_{a}^\infty g(x)dx=+\infty$, then $\int_{a}^\infty f(x)dx=+\infty$.
\end{enumerate}
\end{Theorem}





\begin{example}
\begin{enumerate}[a)]
\item Consider
\[\int_1^\infty\frac{x}{x^2+1}dx.\]
We have
\[\lim_{x\to\infty}x\cdot\frac{x}{x^2+1}=1.\]
For large enough $x\in\mathbb{R}$, we therefore have
\[\left|x\cdot\frac{x}{x^2+1}\right|=\frac{x^2}{x^2+1}\geq\frac12\]
and thus
\[\frac{x}{x^2+1}\geq\frac1{2x}.\]
Since
\[\int_1^\infty\frac{1}{2x}dx\]
is divergent,
\[\int_1^\infty\frac{x}{x^2+1}dx\]
is divergent, too.
\item Consider
\[\int_1^\infty\frac{\sqrt{x}}{x^2+1}dx.\]
We have
\[\lim_{x\to\infty}x^{3/2}\cdot\frac{\sqrt{x}}{x^2+1}=1.\]
For large enough $x\in\mathbb{R}$, we therefore have
\[\left|x^{3/2}\cdot\frac{\sqrt{x}}{x^2+1}\right|=\frac{x^2}{x^2+1}\leq2\]
and thus
\[\frac{\sqrt{x}}{x^2+1}\leq\frac2{x^{3/2}}.\]
Since
\[\int_1^\infty\frac2{x^{3/2}}dx\]
is convergent, so
\[\int_1^\infty\frac{\sqrt{x}}{x^2+1}dx\]
is convergent, too.
\end{enumerate}
\end{example}





