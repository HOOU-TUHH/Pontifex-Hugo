\usepackage{amsthm}

\newtheorem{theorem}{Theorem}[chapter]
\newtheorem{lemma}           [theorem] {Lemma}   
\newtheorem{folg}           [theorem] {Folgerung}   

\newtheorem{frage}       [theorem] {Frage}   
\newtheorem{question}       [theorem] {Question}   
\newtheorem{aufgabe}       [theorem] {Aufgabe}   
\newtheorem{exercise}       [theorem] {Exercise}  

\newtheorem{proposition}     [theorem] {Proposition}  
\newtheorem{satz}     [theorem] {Satz}  
\newtheorem{fact}{Fact}
\newtheorem{definition}      [theorem] {Definition} 

\theoremstyle{definition} 
\newtheorem{bemerkung}     [theorem] {Bemerkung}  
\newtheorem{beispiel}       [theorem] {Beispiel}  
\newtheorem{example}       [theorem] {Example}  
\newtheorem*{example*} {Example}  
\newtheorem{notation}       [theorem] {Notation}  
\newtheorem*{Faust}[theorem]{Rule of Thumb}
\newtheorem*{Boxx}[theorem]{Concept} 

For some particular cases, we nevertheless are able to show convergence.
\begin{Theorem}[Leibniz Convergence Criterion]
\label{th:Leibnizcrit}
Let $(a_n)_{n\in\mathbb{N}}$ be a~monotonically decreasing real sequence with
\[\lim_{n\to\infty}a_n=0.\]
Then the alternating sequence
\[\sum_{k=1}^\infty (-1)^ka_k\]
converges.
\end{Theorem}
{\em Proof:} Since $(a_n)_{n\in\mathbb{N}}$ is convergent to zero and monotonically decreasing, we have $a_n\geq0$ for all $n\in\mathbb{N}$. Let $(s_n)_{n\in\mathbb{N}}$ be the corresponding sequence of partial sums.
Then we have for all $l\in \mathbb{N}$ that
\[s_{2l+2}-s_{2l}=-a_{2l+1}+a_{2l+2}\leq 0,\quad s_{2l+3}-s_{2l+1}=a_{2l+2}-a_{2l+3}\geq 0,\]
i.e., the subsequence $(s_{2l})_{l\in\mathbb{N}}$ is monotonically decreasing and $(s_{2l+1})_{l\in\mathbb{N}}$ is monotonically increasing. Furthermore, due to $s_{2l+1}-s_{2l}=-a_{2l+1}\leq0$ holds
\[s_{2l+1}\leq s_{2l}.\]

Altogether, the $(s_{2l})_{l\in\mathbb{N}}$ is monotonically decreasing and bounded from below, and $(s_{2l+1})_{l\in\mathbb{N}}$ is monotonically increasing and bounded from above.
By the Theorem on monotonic and bounded sequences, both subsequences are convergent. Due to
\[\lim_{l\to\infty}(s_{2l+1}-s_{2l})=\lim_{l\to\infty}a_{2l+1}=0,\]
an application of Theorem on the formulae for convergent sequences yields that both subsequence have the same limit, i.e.,
\[\lim_{l\to\infty}s_{2l+1}=\lim_{l\to\infty}s_{2l}=s\]
for some $s\in\mathbb{R}$. Let $\varepsilon>0$. Then there exists some $N_1$ such that for all $l\geq N_1$ holds $|s_{2l}-s|<\varepsilon$. Furthermore, there exists some $N_2$ such that for all $l\geq N_2$ holds $|s_{2l+1}-s|<\varepsilon$. Now choosing $N=\max\{2N_1,2N_2+1\}$, we can say the following for some $m\geq N$:\\
In the case where $m$ is even, we have some $l\in\mathbb{N}$ with $m=2l$. By the choice of $N$, we also have $l\geq N_1$ and thus
\[|s_{m}-s|=|s_{2l}-s|<\varepsilon.\]
In the case where $m$ is odd, we have some $l\in\mathbb{N}$ with $m=2l+1$. By the choice of $N$, we also have $l\geq N_2$ and thus
\[|s_{m}-s|=|s_{2l+1}-s|<\varepsilon.\]
\hfill$\Box$

\begin{example}\label{ex:leib}
\begin{enumerate}[(a)]
 \item  The {\em alternating harmonic series}
\[\sum_{k=1}^\infty \frac{(-1)^k}{k}.\]
converges. The limit is (without proof) $\log(2)$.
 \item  The {\em alternating Leibniz series}
\[\sum_{k=1}^\infty \frac{(-1)^k}{2k+1}.\]
converges. The limit is (without proof) $\frac\pi4$.
 \item  The series
\[\sum_{k=1}^\infty \frac{(-1)^k}{\sqrt{k}}.\]
converges. The limit is not expressible in a closed form.
\end{enumerate}
\end{example}
We will later treat the topic of {\em Taylor series}. Thereafter we will be able to determine some further limits of sequences.

