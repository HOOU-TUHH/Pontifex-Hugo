\usepackage{amsthm}

\newtheorem{theorem}{Theorem}[chapter]
\newtheorem{lemma}           [theorem] {Lemma}   
\newtheorem{folg}           [theorem] {Folgerung}   

\newtheorem{frage}       [theorem] {Frage}   
\newtheorem{question}       [theorem] {Question}   
\newtheorem{aufgabe}       [theorem] {Aufgabe}   
\newtheorem{exercise}       [theorem] {Exercise}  

\newtheorem{proposition}     [theorem] {Proposition}  
\newtheorem{satz}     [theorem] {Satz}  
\newtheorem{fact}{Fact}
\newtheorem{definition}      [theorem] {Definition} 

\theoremstyle{definition} 
\newtheorem{bemerkung}     [theorem] {Bemerkung}  
\newtheorem{beispiel}       [theorem] {Beispiel}  
\newtheorem{example}       [theorem] {Example}  
\newtheorem*{example*} {Example}  
\newtheorem{notation}       [theorem] {Notation}  
\newtheorem*{Faust}[theorem]{Rule of Thumb}
\newtheorem*{Boxx}[theorem]{Concept}
%\section{Natural Numbers and Induction}
The natural numbers are $\mathbb{N} = \{1,2,3 \ldots \}$.

Using natural numbers is our first mathematical abstraction. We learn this as children in the kindergarden. 

What is this abstraction? A number is an abstraction for all finite sets of the same size.

\begin{itemize}
 \item \emph{Question 1:} When are two sets $S,T$ of the same size?  Have the same \emph{cardinality} $|S|=|T|$?
 \emph{Answer:} They have the same size
 if there is a bijective map $S\to T$.
 \white{3cm}{For example, $\mathbb{N}$ and the set of all even numbers have the same cardinality.}
 \item \emph{Question 2:} When is a set $S$ finite? \emph{Answer:}
 It is finite if removing one element changes the \emph{cardinality} of $S$. 
 \white{3cm}{}
\end{itemize}

In mathematical language: ``Natural numbers are equivalence classes of finite sets of the same cardinality.'' 

\subsubsection{Mathematical induction}

Mathematical induction is an important technique of proof: Proof step by step. It is a close relative to 
recursion in computer science:

``Assume I can solve a problem of size $n$. How can I solve one of size $n+1$?''

In mathematics:

``If an assertion is true for $n$, show that it is true for $n+1$''


\begin{example}
What is the sum of the first $n$ natural numbers?
\[
 s_n := \sum_{k=1}^n k = \,?
\]

To make this practical, we need three ingredients:
\begin{itemize}
 \item[(i)] An idea what the result could be. (Induction hypothesis)
 \item[(ii)] The verification that our hypothesis is true for $n=1$ (Base case)
 \item[(iii)] A proof, that if it holds for $n$, then also for $n+1$. (Induction step)
\end{itemize}
\white{5cm}{
Getting the first ingredient is often the most difficult one. Often one has to try it out:
\begin{align*}
 s_1 = 1\\
 s_2 = s_1+2 = 3\\
 s_3 = s_2+3 = 6\\
 s_4 = s_3+4=10\\
 s_5 = s_4+5=15\\
 s_{n+1}=s_n+n+1
\end{align*}
}
Ideas? Let's take the hypothesis
\[
 s_n = \frac{(n+1)n}{2}\qquad  \mbox{ (Induction hypothesis). }
\]
Very good! We can verify our formula for these examples. In particular:
\[
 s_1 = \frac{(1+1)1}{2}=1 \qquad  \mbox{ (Base case). }
\]
\emph{Induction step:} We have to show
\[
 \frac{(n+2)(n+1)}{2} \mbox{ is equal to } s_{n+1}=s_n + (n+1)
 = \frac{(n+1)n}{2}+n+1
\]
where we used the induction hypothesis in the last step.
So let us compute:
\begin{align*}
 s_n + (n+1) 
 = \frac{(n+1)n}{2}+n+1&=\frac{n^2+n+2n+2}{2} =
 \frac{(n+2)(n+1)}{2}
 \,.
\end{align*}
This proves that $s_n = \frac{(n+1)n}{2}$ for all $n \in \mathbb{N}$.
\end{example}

\begin{Faust}[Mathematical induction]
 To show that the predicate $A(n)$ is true \emphblue{for all} $n \in \mathbb{N}$,
 we have to show two things:
 	\begin{enumerate}[(1)]
 		\item Show that $A(1)$ is true.
 		\item Show that $A(n+1)$ is true under the assumption that $A(n)$
 		is true.
 	\end{enumerate}
\end{Faust}

Sometimes it can happen that a claim $A(n)$ is indeed false
for finitely many natural numbers, but it is eventually true. This means
that the base case cannot be shown for $n=1$ but
for some other natural number $n_0 \in \mathbb{N}$. Then the induction proof
shows that $A(n)$ is true for all natural number $n \geq n_0$.
