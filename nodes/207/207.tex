\usepackage{amsthm}

\newtheorem{theorem}{Theorem}[chapter]
\newtheorem{lemma}           [theorem] {Lemma}   
\newtheorem{folg}           [theorem] {Folgerung}   

\newtheorem{frage}       [theorem] {Frage}   
\newtheorem{question}       [theorem] {Question}   
\newtheorem{aufgabe}       [theorem] {Aufgabe}   
\newtheorem{exercise}       [theorem] {Exercise}  

\newtheorem{proposition}     [theorem] {Proposition}  
\newtheorem{satz}     [theorem] {Satz}  
\newtheorem{fact}{Fact}
\newtheorem{definition}      [theorem] {Definition} 

\theoremstyle{definition} 
\newtheorem{bemerkung}     [theorem] {Bemerkung}  
\newtheorem{beispiel}       [theorem] {Beispiel}  
\newtheorem{example}       [theorem] {Example}  
\newtheorem*{example*} {Example}  
\newtheorem{notation}       [theorem] {Notation}  
\newtheorem*{Faust}[theorem]{Rule of Thumb}
\newtheorem*{Boxx}[theorem]{Concept}

The following criterion can be seen as a ``series version'' of the comparison criterion for sequences.  
\begin{Theorem}[Majorant criterion]
\label{thm:majcrit}
Let $\sum_{k=1}^\infty a_k$ be a~series in $\mathbb{K}$. Moreover, let $n_0\in \mathbb{N}$ and let $\sum_{k=1}^\infty b_k$ 
be a~real convergent series such that $|a_k|\leq b_k$ for all $k\geq n_0$. Then $\sum_{k=1}^\infty a_k$ converges absolutely.
\end{Theorem}
{\em Proof:} Let $\varepsilon>0$. By the Cauchy criterion applied to $\sum_{k=1}^\infty b_k$ there is an $N\geq n_0$ such that for all $n\geq m\geq N$ holds
\[0\leq \sum_{k=m}^n|a_k|\leq\sum_{k=m}^nb_k=\left|\sum_{k=m}^nb_k\right|<\varepsilon.\]
The Cauchy criterion now implies that $\sum_{k=1}^\infty |a_k|$ converges.\hfill$\Box$
%
%
\begin{Remark}{}
 The series $\sum_{k=1}^\infty b_k$ with the properties as stated in Theorem about the majorant criterion is called a~{\em majorant of $\sum_{k=1}^\infty a_k$}.
\end{Remark}
%
%
Now we present a~kind of {\em reversed majorant criterion} that gives us a~sufficient criterion for divergence.
\begin{Theorem}[Minorant criterion]\label{thm:mincrit}
Let $\sum_{k=1}^\infty a_k$ be a~real series. Moreover, let $n_0\in \mathbb{N}$ and $\sum_{k=1}^\infty b_k$ 
be a~divergent series such that $a_k\geq b_k\geq 0$ for all $k\geq n_0$. Then $\sum_{k=1}^\infty a_k$ diverges.
\end{Theorem}
{\em Proof:} We prove the result by contradiction: Let $\sum_{k=1}^\infty b_k$ be divergent. 
Assume that $\sum_{k=1}^\infty a_k$ converges. Then, due to $a_k\geq b_k\geq 0$, the majorant criterion implies the convergence of 
$\sum_{k=1}^\infty b_k$, too. This is a~contradiction to our assumption.\hfill$\Box$

\begin{Remark}{}
 The series $\sum_{k=1}^\infty b_k$ with the properties as stated in the minorant criterion is called a~{\em minorant of $\sum_{k=1}^\infty a_k$}.
\end{Remark}
%
