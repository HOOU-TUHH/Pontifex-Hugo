\usepackage{amsthm}

\newtheorem{theorem}{Theorem}[chapter]
\newtheorem{lemma}           [theorem] {Lemma}   
\newtheorem{folg}           [theorem] {Folgerung}   

\newtheorem{frage}       [theorem] {Frage}   
\newtheorem{question}       [theorem] {Question}   
\newtheorem{aufgabe}       [theorem] {Aufgabe}   
\newtheorem{exercise}       [theorem] {Exercise}  

\newtheorem{proposition}     [theorem] {Proposition}  
\newtheorem{satz}     [theorem] {Satz}  
\newtheorem{fact}{Fact}
\newtheorem{definition}      [theorem] {Definition} 

\theoremstyle{definition} 
\newtheorem{bemerkung}     [theorem] {Bemerkung}  
\newtheorem{beispiel}       [theorem] {Beispiel}  
\newtheorem{example}       [theorem] {Example}  
\newtheorem*{example*} {Example}  
\newtheorem{notation}       [theorem] {Notation}  
\newtheorem*{Faust}[theorem]{Rule of Thumb}
\newtheorem*{Boxx}[theorem]{Concept}
%\subsection*{Predicates and quantifiers}

\begin{Definition}[Predicate]
If $X$ is any set and $A(x)$ is a logical statement depending on $x \in X$ (and true or false for every $x\in X$),
we call $A(x)$ a \emph{predicate} with variable $x$.
Usually, one writes simply $A(x)$ instead of $A(x)=$ true. 
\end{Definition}

\begin{example}{}
\[
 X=\mathbb{R}\qquad   A(x) \;=\; x < 0
\]
Then we can define the set
	$$
	\{ x \in X : A(x) \}	= \{ x \in \mathbb{R} : x < 0 \}
	$$
\end{example}

\begin{Definition}[Quantifiers $\forall$ and $\exists$]
We use
$\forall$~(``for all'') and $\exists$~(``it exists'')
and call them \emph{quantifiers}.
Moreover, we use
the double point `` $:$ '' inside the set brackets, which means
``that fulfil''.  
\end{Definition}


The quantifiers and predicates are very useful for a compact notation:
\begin{itemize}
 \item $\forall x \in X : A(x)~~$  \emph{for all $x\in X$ $A(x)$ is true}
 \item $\exists x \in X : A(x)~~$  \emph{there exists at least one $x\in X$ for which $A(x)$ is true}
 \item $\exists! x \in X : A(x)~~$ \emph{there exists exactly one $x\in X$ for which $A(x)$ is true}
\end{itemize}

Negation of statements with quantifiers:
\begin{itemize}
 \item $\neg (\forall x \in X : A(x)) ~\Leftrightarrow~ \exists x \in X : \neg A(x)$
 \item $\neg (\exists x \in X : A(x)) ~\Leftrightarrow~ \forall x \in X : \neg A(x)$
\end{itemize}

\begin{example}
There is no greatest natural number:
\begin{center}
 $A(n) \;=\; n$ is the greatest natural number
\end{center}
In our notation: $\neg (\exists n \in \mathbb{N} : A(n))$ this is the same as $\forall n \in \mathbb{N} : \neg A(n)$,
i.e. \emph{Each $n \in \mathbb{N}$ is not the greatest natural number }.
But this is clear, because $n+1 > n$. 
\end{example}

\begin{Faust}[Negation of the quantifier ($\forall$ and $\exists$)]
\begin{center}
$\neg\forall = \exists \neg$ and $\neg\exists = \forall \neg$
\end{center}
\end{Faust}

\begin{example}{}
The set $M:=\{x\in\mathbb{Z} \colon x^2=25\}$ 
is defined by the set of each integer $x$
that squares to 25. We immediately see that this is just $-5$ and $5$.
\white{5cm}{
\begin{align*}
\{x\in\Z\st x^2=25\} &= \{-5,5\}, \\
\{x\in\N\st x^2=25\} &= \{5\},\\
\{x\in\R\st x^2=-25\} &= \varnothing .
\end{align*}
In other words: The equation $x^2=25$ with unknown $x$ has, depending in which number realm you want to solve it, 
one or two solutions, and the equation $x^2=-25$ has no solution in the real numbers. However, we will find solutions in the complex numbers as we will see later.
}
\end{example}
