\usepackage{amsthm}

\newtheorem{theorem}{Theorem}[chapter]
\newtheorem{lemma}           [theorem] {Lemma}   
\newtheorem{folg}           [theorem] {Folgerung}   

\newtheorem{frage}       [theorem] {Frage}   
\newtheorem{question}       [theorem] {Question}   
\newtheorem{aufgabe}       [theorem] {Aufgabe}   
\newtheorem{exercise}       [theorem] {Exercise}  

\newtheorem{proposition}     [theorem] {Proposition}  
\newtheorem{satz}     [theorem] {Satz}  
\newtheorem{fact}{Fact}
\newtheorem{definition}      [theorem] {Definition} 

\theoremstyle{definition} 
\newtheorem{bemerkung}     [theorem] {Bemerkung}  
\newtheorem{beispiel}       [theorem] {Beispiel}  
\newtheorem{example}       [theorem] {Example}  
\newtheorem*{example*} {Example}  
\newtheorem{notation}       [theorem] {Notation}  
\newtheorem*{Faust}[theorem]{Rule of Thumb}
\newtheorem*{Boxx}[theorem]{Concept}

\begin{Definition}[Continuity]
Let $I\subset\mathbb{K}$ and let $f:I\to\mathbb{K}$ be a~function. Then $f$ is called \emph{ continuous in $x_0\in I$} if
\[\lim_{x\to x_0}f(x)=f(x_0).\]
Moreover, $f$ is called \emph{continuous on $I$} if it is continuous in $x_0$ for all $x_0\in I$.
\end{Definition}
\begin{Remark}{}
Sometimes we will just say $f:I\to\mathbb{K}$ is continuous whereby we mean it is continuous on $I$.
\end{Remark}

\begin{example}
\begin{enumerate}[a)]
 \item The {\em constant function} $f:\mathbb{R}\to\mathbb{R}$ with $f(x)=c$ for some $c\in\mathbb{R}$ is continuous on $\mathbb{R}$.
 \whiteskipsmall
 \item The Heaviside function is discontinuous at $x_0=0$, but continuous everywhere else.
  \whiteskipsmall
 \item The function $f:\mathbb{R}\to\mathbb{R}$ as in Example~\ref{ex:funclim} b) is discontinuous at $x_0=0$, but continuous everywhere else.
  \whiteskipsmall
 \item Polynomials are continuous on $\mathbb{R}$.
   \whiteskipsmall
 \item {\em Rational functions} $f:I\to\mathbb{K}$ with $f(x)=\frac{p(x)}{q(x)}$ for some polynomials $p,q$ ($q$ is not the zero polynomial) are defined on $I=\{x\in\mathbb{R}\,:\,q(x)\neq0\}$ and are continuous
on $I$ (due to the formulae for convergent sequences).
	 \whiteskipsmall
 \item The {\em absolute value function} $|\cdot|:\mathbb{R}\to\mathbb{R}$, i.e.,
\[|x|=\begin{cases}x&,\text{ if }x\geq 0,\\-x&,\text{ if }x<0.\end{cases}\]
 is continuous on $\mathbb{R}$.
 	  \whiteskipsmall
 \item The function $f:\mathbb{R}\to\mathbb{R}$ with
\[f(x)=\begin{cases}1&,\text{ if }x\in\mathbb{Q},\\0&,\text{ if }x\notin\mathbb{Q}\end{cases}\]
 is everywhere discontinuous.\\
       \whiteskipsmall
{\em Proof:} Let $x_0\in\mathbb{R}$:\\
First case: $x_0\in\mathbb{Q}$. Then take a~sequence $(x_n)_{n\in\mathbb{N}}$ with $\lim_{n\to \infty}x_n=x_0$ and $x_n\in\mathbb{R}\backslash\mathbb{Q}$ (for instance, $x_n=x_0+\frac{\sqrt{2}}{n}$). Then $f(x_n)=0$ for all $n\in\mathbb{N}$ and thus $\lim_{n\to \infty}x_n=0\neq f(x_0)=1$.\\
Second case: $x_0\in\mathbb{R}\backslash\mathbb{Q}$. Then take a~sequence $(x_n)_{n\in\mathbb{N}}$ with $\lim_{n\to \infty}x_n=x_0$ and $x_n\in\mathbb{Q}$ (this exists since every real number can be approximated by a~rational number in arbitrary good precision). Then $f(x_n)=1$ for all $n\in\mathbb{N}$ and thus $\lim_{n\to \infty}x_n=1\neq f(x_0)=0$.
\end{enumerate}
\end{example}

