\usepackage{amsthm}

\newtheorem{theorem}{Theorem}[chapter]
\newtheorem{lemma}           [theorem] {Lemma}   
\newtheorem{folg}           [theorem] {Folgerung}   

\newtheorem{frage}       [theorem] {Frage}   
\newtheorem{question}       [theorem] {Question}   
\newtheorem{aufgabe}       [theorem] {Aufgabe}   
\newtheorem{exercise}       [theorem] {Exercise}  

\newtheorem{proposition}     [theorem] {Proposition}  
\newtheorem{satz}     [theorem] {Satz}  
\newtheorem{fact}{Fact}
\newtheorem{definition}      [theorem] {Definition} 

\theoremstyle{definition} 
\newtheorem{bemerkung}     [theorem] {Bemerkung}  
\newtheorem{beispiel}       [theorem] {Beispiel}  
\newtheorem{example}       [theorem] {Example}  
\newtheorem*{example*} {Example}  
\newtheorem{notation}       [theorem] {Notation}  
\newtheorem*{Faust}[theorem]{Rule of Thumb}
\newtheorem*{Boxx}[theorem]{Concept}
\begin{Theorem}[Monotonicity of limits]\label{thm:monseq}
  Let $(a_n)_{n\in\mathbb{N}}$ and $(b_n)_{n\in\mathbb{N}}$ be two convergent real sequences with
\[\lim_{n\to\infty}a_n=a,\qquad \lim_{n\to\infty}b_n=b.\]
  Further, assume that for all $n\in\mathbb{N}$ holds $a_n\leq b_n$. Then the following holds true:
\begin{enumerate}[(i)]
 \item $a\leq b$;
 \item \emph{Sandwich-Theorem}: If $a=b$ and $(c_n)_{n\in\mathbb{N}}$ is another sequence with $a_n\leq c_n\leq b_n$ for all $n \in \mathbb{N}$, then $(c_n)_{n\in\mathbb{N}}$ is convergent with
\[\lim_{n\to\infty}c_n=a=b.\]
 
\end{enumerate}
\end{Theorem}
\begin{proof}
\begin{enumerate}[(i)]
\item 
  Consider the sequence of differences between $b_n$ and $a_n$, i.e., $(b_n-a_n)_{n\in\mathbb{N}}$. By the calculation rules for converging sequences, it suffices to show that
\[b-a=\lim_{n\to\infty}(b_n-a_n)\geq0.\]
Assume the converse statement, i.e., $b-a<0$. Then, we have that both numbers $a-b$ and $b_n-a_n$ are positive and thus
    \[|a-b-(a_n-b_n)|=a-b+(b_n-a_n)>a-b.\] In particular, there exists no $n\in\mathbb{N}$ such that $|a-b-(a_n-b_n)|<\varepsilon$ for $\varepsilon=a-b>0$. This is a~contradiction to $\lim_{n\to\infty}(b_n-a_n)=b-a$.
\item Again consider the sequence $(b_n-a_n)_{n\in\mathbb{N}}$ which is tending to zero according to the formulae for converging sequences. Further, consider the sequence $(c_n-a_n)_{n\in\mathbb{N}}$. 
Then we have
  for all $n\in\mathbb{N}$ that $0\leq c_n-a_n\leq b_n-a_n$.
Let $\varepsilon>0$. Since $b_n-a_n$ is tending to zero, there exists some $N$ such that for all $n\geq N$ holds $|b_n-a_n-0|<\varepsilon$. Due to $0\leq c_n-a_n\leq b_n-a_n$, we can conclude that for $n\geq N$ holds
\[|c_n-a_n-0|=c_n-a_n\leq b_n-a_n= |b_n-a_n-0|<\varepsilon.\]
    This implies that $(c_n-a_n)_{n\in\mathbb{N}}$ is convergent with $\lim_{n\to\infty}(c_n-a_n)=0$.
Hence $a=0+a=\lim_{n\rightarrow\infty} (c_n-a_n) + \lim_{n\rightarrow\infty} a_n = \lim_{n\rightarrow\infty} c_n $.
\end{enumerate}
\end{proof}

\begin{Remark}{}\label{rem:n_0mon}
  Since the modification of finitely many sequence elements does not change the limits (take a~closer look at the definition of convergence, the statements of Theorem~\ref{thm:monseq} can be slightly generalised by only claiming that there exists some $n_0$ such that for all $n\geq n_0$ holds $a_n\leq b_n$ (resp.\ for all $n\geq n_0$ holds $a_n\leq c_n\leq b_n$ in (ii)). In the proof of (i), one has to replace the words ``there exists no $n\in\mathbb{N}$ such that'' by ``there exists no $n\geq n_0$ such that'' and in the proof of (ii) the number $N$ has to be replaced by $\max\{N,n_0\}$.
\end{Remark}

\begin{Attention}{}
  From the fact that we have the strict inequality $a_n<b_n$, we cannot conclude that the limits satisfy $a<b$. To see this, consider the sequences\linebreak $(a_n)_{n\in\mathbb{N}}=(0,0,0,\ldots)$ and
  $(b_n)_{n\in\mathbb{N}}=(\frac1n)_{n\in\mathbb{N}}$. In this case, we have $a=b=0$ though the strict inequality $a_n=0<\frac1n=b_n$ holds true for all $n\in\mathbb{N}$.
\end{Attention}

\begin{example}
\begin{enumerate}[a)]
  \item Consider $(\frac1{n^k})_{n\in\mathbb{N}}$ for some $k\in\mathbb{N}$. We state two alternative ways to show that this sequence tends to zero.
The first possibility is, of course, an~argumentation as in statement (b) of the remarks on limit theorems. The second way to treat this problem is by making use of the inequality
\[\frac1n\geq\frac1{n^k}>0.\]
    Since we know from Example~\ref{ex:basicconvseq} a) that the sequence $(\frac1n)_{n\in\mathbb{N}}$ tends to zero, statement (ii) of Theorem~\ref{thm:monseq} directly leads to the fact that
    $(\frac1{n^k})_{n\in\mathbb{N}}$ also tends to zero.
  \item Consider $(a_n)_{n\in\mathbb{N}}$ with
\[a_n=\frac{2n^2+5n-1}{-5n^2+n+1}.\]
Rewriting
\[a_n=\frac{2+\frac5n-\frac1{n^2}}{-5+\frac1n+\frac1{n^2}},\]
    and using that both $(\frac1{n})_{n\in\mathbb{N}}$ and $(\frac1{n^2})_{n\in\mathbb{N}}$ tend to zero, we can apply the limit theorems to obtain that
\[\lim_{n\to\infty}a_n=\lim_{n\to\infty}\frac{2n^2+5n-1}{-5n^2+n+1}=\lim_{n\to\infty}\frac{2+\frac5n-\frac1{n^2}}{-5+\frac1n+\frac1{n^2}}=-\frac25.\]
\item Consider $(a_n)_{n\in\mathbb{N}}$ with $a_n=\sqrt{n^2+1}-n$. At first glance, none of the so far presented results seem to help to analyse convergence of this sequence. However, we can compute
\[
\begin{aligned}
a_n\,=&\sqrt{n^2+1}-n
=\frac{(\sqrt{n^2+1}-n)(\sqrt{n^2+1}+n)}{\sqrt{n^2+1}+n}\\
=&\frac{n^2+1-n^2}{\sqrt{n^2+1}+n}
=\frac{1}{\sqrt{n^2+1}+n}<\frac1n.
\end{aligned}
\]
By Theorem~\ref{thm:monseq}, we now get that $\lim_{n\to\infty}a_n=0$.
\end{enumerate}
\end{example}
