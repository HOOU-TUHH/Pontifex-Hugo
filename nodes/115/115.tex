\usepackage{amsthm}

\newtheorem{theorem}{Theorem}[chapter]
\newtheorem{lemma}           [theorem] {Lemma}   
\newtheorem{folg}           [theorem] {Folgerung}   

\newtheorem{frage}       [theorem] {Frage}   
\newtheorem{question}       [theorem] {Question}   
\newtheorem{aufgabe}       [theorem] {Aufgabe}   
\newtheorem{exercise}       [theorem] {Exercise}  

\newtheorem{proposition}     [theorem] {Proposition}  
\newtheorem{satz}     [theorem] {Satz}  
\newtheorem{fact}{Fact}
\newtheorem{definition}      [theorem] {Definition} 

\theoremstyle{definition} 
\newtheorem{bemerkung}     [theorem] {Bemerkung}  
\newtheorem{beispiel}       [theorem] {Beispiel}  
\newtheorem{example}       [theorem] {Example}  
\newtheorem*{example*} {Example}  
\newtheorem{notation}       [theorem] {Notation}  
\newtheorem*{Faust}[theorem]{Rule of Thumb}
\newtheorem*{Boxx}[theorem]{Concept}

\begin{Definition}[Cauchy sequences]
A sequence $(a_n)_{n\in\mathbb{N}}$ in $\mathbb{K}$ is called \textit{Cauchy sequence} if for all $\varepsilon>0$, there exists some $N$ such that for all $n,m\geq N$ holds
\[|a_n-a_m|<\varepsilon.\]
\end{Definition}
\begin{Remark}{}
By the expression ``$n,m\geq N$'', we mean that both $n$ and $m$ are greater or equal than $N$, i.e., $n\geq N$ \underline{and} $m\geq N$.
\end{Remark}

Now we show that convergent sequences are indeed Cauchy sequences.
\begin{Theorem}{}\label{thm:convcauch}
    Let $(a_n)_{n\in\mathbb{N}}$ be a~convergent sequence. Then $(a_n)_{n\in\mathbb{N}}$ is a~Cauchy sequence.
\end{Theorem}
{\em Proof:}
Let $a=\lim_{n	o\infty}a_{n}$ and $\varepsilon>0$. Then there exists some $N$ such that for all $k\geq N$ holds $|a-a_k|<\frac{\varepsilon}2$. Hence, for all $m,n\geq N$ holds
\[|a_n-a_m|=|(a_n-a)+(a-a_m)|\leq|a_n-a|+|a-a_m|< \frac{\varepsilon}2+\frac{\varepsilon}2=\varepsilon.\]
$\Box$

We know that convergent sequences are bounded.
The following theorem shows that this is also the case for Cauchy sequences. 
\begin{Theorem}[Cauchy sequences are bounded]\label{thm:cauchseqbnd}
    Let $(a_n)_{n\in\mathbb{N}}$ be a~Cauchy sequence. Then $(a_n)_{n\in\mathbb{N}}$ is bounded.
\end{Theorem}
{\em Proof:} Take $\varepsilon=1$. Then there exists some $N$ such that for all $n,m\geq N$ holds $|a_n-a_m|<1$. Thus, for all $n\geq N$ holds
\[|a_n|=|a_n-a_N+a_N|\leq |a_n-a_N|+|a_N|<1+|a_N|.\]
Now choose
\[c=\max\{|a_1|,|a_2|,\ldots,|a_{N-1}|,|a_N|+1\}\]
and consider some arbitrary sequence element $a_k$.\\
If $k<N$, we have that $|a_k|\leq \max\{|a_1|,|a_2|,\ldots,|a_{N-1}|\}\leq c$.\\
If $k\geq N$, we have, by the above calculations, that $|a_k|<|a_N|+1\leq c$.\\
Altogether, this implies that $|a_k|\leq c$ for all $k\in\mathbb{N}$, so $(a_n)_{n\in\mathbb{N}}$ is bounded by $c$.\hfill$\Box$
