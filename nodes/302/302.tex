\usepackage{amsthm}

\newtheorem{theorem}{Theorem}[chapter]
\newtheorem{lemma}           [theorem] {Lemma}   
\newtheorem{folg}           [theorem] {Folgerung}   

\newtheorem{frage}       [theorem] {Frage}   
\newtheorem{question}       [theorem] {Question}   
\newtheorem{aufgabe}       [theorem] {Aufgabe}   
\newtheorem{exercise}       [theorem] {Exercise}  

\newtheorem{proposition}     [theorem] {Proposition}  
\newtheorem{satz}     [theorem] {Satz}  
\newtheorem{fact}{Fact}
\newtheorem{definition}      [theorem] {Definition} 

\theoremstyle{definition} 
\newtheorem{bemerkung}     [theorem] {Bemerkung}  
\newtheorem{beispiel}       [theorem] {Beispiel}  
\newtheorem{example}       [theorem] {Example}  
\newtheorem*{example*} {Example}  
\newtheorem{notation}       [theorem] {Notation}  
\newtheorem*{Faust}[theorem]{Rule of Thumb}
\newtheorem*{Boxx}[theorem]{Concept}

We now present an alternative convergence concept for functions:

\begin{Definition}[Uniform convergence]
A~sequence $(f_n)_{n\in\mathbb{N}}$ of functions $f_n:I\to\mathbb{K}$ is called \emph{uniformly convergent} to $f:I\to\mathbb{K}$ if for all $\varepsilon>0$ there exists some $N$ such that for all $n\geq N$ and $x\in I$ holds
\[|f(x)-f_n(x)|<\varepsilon.\]
Using logical quantifiers this reads (compare with the quantifiers for pointwise convergence!):
\begin{equation}\label{def:lunifconv}
		 \forall \varepsilon > 0 
	\quad \exists N \in \mathbb{N}
	\quad \forall n \geq N
	\quad \forall x \in I
	\quad : \quad
	|f_n(x)-f(x)|<\varepsilon \ .
\end{equation}
\end{Definition}

As a rule of thumb, you can think of pushing one quantifier to the right but, of course, this will change a lot.
The interpretation is that we can measure the distance between two functions 
$f$ und $g$ as the largest distance between the two graphs, that means the distance you can measure at a given point:
	$$
		\sup_{x \in I} | f(x) - g(x) |
	$$
We have the uniform convergence 
if this measured distance between  $f_n$ and
$f$ is convergent to zero. (See below.)

% Look at the following example:
% \begin{figure}[ht]
% \begin{center}
% \begin{tikzpicture}[domain=-2:10]
%   \draw[line width=1.7pt, color=blue] (0 ,1) -- (2 ,1) -- (4 ,3)--(6,3)
%   node[right=5mm] %
%         {$f_1$};
%     \draw[line width=1.2pt, color=orange] (0 ,1) -- (3 ,1) -- (4 ,3)--(6,3)  node[below=5mm,
%     right=5mm] %
%         {$f_2$};
%     \draw[line width=1pt, color=green] (0 ,1) -- (3.6 ,1) -- (4 ,3)--(6,3)  node[below=10mm, right=5mm] %
%         {$f_3$};
%       %  \draw[line width=1.5pt, color=NavyBlue] (0 ,8) -- (1/2 ,8) -- (1/2 ,0);
%   \draw[->] (-1,0) -- (8,0); \draw[->] (0,0) -- (0,3.5);
%     %\draw[color=black, fill=NavyBlue](1/2 ,0) circle(1.5pt) node[below] {$\frac{1}{n}$};
%       \draw[color=black] (4 ,-0.1) -- (4 ,0.1) node[below=1mm] {$0$};
%       \draw[color=black](-0.1 ,1)-- (0.1 ,1) node[left=1mm] {$-1$};
%       \draw[color=black](-0.1 ,3)-- (0.1 ,3) node[left=1mm] {$1$};
%     %\draw[color=black, fill=blue](2 ,0) circle(1.5pt) node[below] {$1$};
% \end{tikzpicture} 
% \end{center}
% \end{figure}
% 
% One sees that the functions $f_n$ pointwisely converges to a limit function.
% We also see that we can build a jump by increasing $n$ for the limit function.
% The distance between the limit function to each member of the sequence is indeed
% 	$$
% 	\sup_{x \in \mathbb{R}} \abs{ f(x) - f_n(x) } = 2
% 	$$
% since there is always an $x \in \mathbb{R}$ which can be chosen close enough to $0$ to get an approximation of this distance.
% This means that the sequence of functions $(f_n)_{n \in \mathbb{N}}$
% does not converges uniformly in spite of being pointwisely convergent.
% The uniform convergence is in fact a much stronger notion.

We will now see that uniform convergence is a stronger property than pointwise convergence.
\begin{Theorem}[Uniform convergence implies pointwise convergence]
Let a~sequence $(f_n)_{n\in\mathbb{N}}$ with $f_n:I\to\mathbb{K}$ be uniformly convergent to $f:I\to\mathbb{K}$. Then
$(f_n)_{n\in\mathbb{N}}$ is also pointwisely convergent to $f$.
\end{Theorem}
{\em Proof:} Let $\varepsilon>0$. Then there exists some $N$ such that for all $n\geq N$ and $x\in I$ holds
\[|f(x)-f_n(x)|<\varepsilon.\]
In particular, for some arbitrary $x\in I$ holds
\[|f(x)-f_n(x)| <\varepsilon.\]
Hence, the sequence $(f_n(x))_{n\in\mathbb{N}}$ in $\mathbb{K}$ converges to $f(x)$.\hfill$\Box$

\begin{Remark}{}
%By the definition of the norm of $\mathcal{B}(I)$, uniform continuity can be equivalently characterized as follows: For all $\varepsilon>0$, there exists some $N$ such that for all $n\geq N$ and $x\in I$ holds
%\[|f(x)-f_n(x)|<\varepsilon.\]
Uniform convergence to $f:I\to\mathbb{K}$ means that for all $\varepsilon>0$ holds that all (except finitely many) functions $f_n$ are ``inside some $\varepsilon$-stripe around $f$''.
\end{Remark}

\begin{Theorem}\label{thm:supunifconv}
A~sequence $(f_n)_{n\in\mathbb{N}}$ of functions $f_n:I\to\mathbb{K}$ converges uniformly to $f:I\to\mathbb{K}$, if, and only if,
\begin{equation}
\lim_{n\to\infty}||f_n-f||_\infty=\lim_{n\to\infty}\sup\{|f(x)-f_n(x)|\;:\;x\in I\}=0.
\end{equation}
This means that uniform convergence is nothing but convergence with respect to the infinity norm $||\cdot||_\infty$ .
\end{Theorem}

% \begin{figure}[h!]
% \begin{center}
% \begin{psfrags}
% \psfrag{f-}{\small$\!\!\!\!\!\!\!\!f(x)-\varepsilon$}
% \psfrag{f+}{\small$f(x)+\varepsilon$}
% \psfrag{f}{\small$f(x)$}
% \psfrag{fn}{\small$\!\!\!\!f_n(x)$}
% \centering
% \includegraphics[width=0.81\textwidth,angle=0]{pics/unif_conv.eps}
% \end{psfrags}
% \caption{Uniform convergence graphically illustrated}\label{fig:unif_conv}
% \end{center}
% \end{figure}


\begin{proof}
Assume that $(f_n)_{n\in\mathbb{N}}$ converges uniformly to $f$. Let $\varepsilon>0$. Then there exists some $N$ such that for all $n\geq N$ and $x\in I$ holds
\[|f(x)-f_n(x)|<\frac\varepsilon2.\]
Therefore, for all $n\geq N$, we have
\[\sup\{|f(x)-f_n(x)|\;:\;x\in I\}\leq\frac\varepsilon2<\varepsilon,\]
and thus, the equation in Theorem~\ref{thm:supunifconv} holds true.\\
Conversely, assuming that the equation in Theorem~\ref{thm:supunifconv} holds true, we obtain that for $\varepsilon>0$, there exists some $N$ with the property that for all $n\geq N$ holds
\[\sup\{|f(x)-f_n(x)|\;:\;x\in I\}<\varepsilon.\]
This means that for all $n\geq N$ and $x\in I$, there holds that
\[|f(x)-f_n(x)|<\varepsilon.\]
However, this statement is nothing but uniform convergence of $(f_n)_{n\in\mathbb{N}}$ towards~$f$.
\end{proof}


\begin{example}\label{ex:funcconv}
 \begin{enumerate}[a)]
\item Let $I=[0,1]$ and consider the sequence $f_n(x)= x^n$. Then we have the pointwise limit
\[f(x)=\lim_{n\to\infty}f_n(x)=\begin{cases}0&,\text{ if }x\in[0,1),\\1&,\text{ if }x=1.\end{cases}\]
Is $(f_n)_{n\in\mathbb{N}}$ also uniformly convergent to $f$?\\

The answer is no, since for $x_n=1/\sqrt[n]{2}$, there holds
\[|f(x_n)-f_n(x_n)|=\left|0-{\textstyle\frac12}\right|={\textstyle\frac12}.\]

\item We now consider the same sequence on the smaller interval $[0,\frac12]$. The pointwise limit is now $f=0$. For $n\in\mathbb{N}$, we have
\[\sup\left\{|f(x)-f_n(x)|\,:\,x\in[0,{\textstyle\frac12}]\right\}=\sup\left\{x^{n}:\,x\in[0,{\textstyle\frac12}]\right\}=\frac1{2^n}.\]

Therefore
\[\lim_{n\to\infty}\sup\left\{|f(x)-f_n(x)|\,:\,x\in[0,{\textstyle\frac12}]\right\}=0\]
and hence, we have uniform convergence.

\item Define the function $f_n:[0,1]\to\mathbb{R}$ by
\[f_n(x)=\begin{cases}n^2x(1-nx)&,\text{ if }x\in[0,\frac1n[,\\0&,\text{ if }x\in[\frac1n,1].\end{cases}\]

Then for all $x\in[0,1]$ holds
\[\lim_{n\to\infty}f_n(x)=0\]
since $f_n(0)=0$ and $f_n(x)=0$ if $x>\frac1n$. The sequence $(f_n)_{n\in\mathbb{N}}$ is however not uniformly convergent to $f=0$, since
\[\sup\{|f_n(x)-0|\,:\,x\in[0,1]\}\geq\left|f_n\left({\textstyle\frac1{2n}}\right)\right|=\frac{n}4.\]

\end{enumerate}
\end{example}

\begin{Theorem}{}\label{thm:unifconvbound}
Let $(f_n)_{n\in\mathbb{N}}$ be a~sequence of bounded functions $f_n:I\to\mathbb{K}$. Assume that $(f_n)_{n\in\mathbb{N}}$ converges uniformly to $f:I\to\mathbb{K}$. Then $f$ is bounded.
\end{Theorem}
{\em Proof:} For $\varepsilon=1$, there exists some $N$ such that for all $n\geq N$ and $x\in I$ holds \[|f_n(x)-f(x)|<1.\] In particular, we have $|f_N(x)-f(x)|<1$ for all $x\in I$.
This consequences that for all $x\in  I$, there holds \[|f(x)|<|f_N(x)|+1.\] The boundedness of $f_N$ then implies the boundedness of $f$.
\hfill$\Box$

\begin{Remark}{}
  Note that the assumption of uniform convergence is essential for the boundedness of $f$. For instance, consider the sequence $(f_n)_{n\in\mathbb{N}}$ of bounded functions $f_n:[0,\infty)\to\mathbb{R}$ with
\[f_n(x)=\begin{cases}x:&\text{ if }x<n\\0:&\text{ else.}\end{cases}\]
  First we argument that $(f_n)_{n\in\mathbb{N}}$ converges pointwisely to $f:[0,\infty)\rightarrow \mathbb{R}$ with $f(x)=x$:
  Let $x \in[0,\infty)$. Then there exists some $N\in\mathbb{N}$ with $x<N$. Hence, for all $n\geq N$, we have $f_n(x)=x$. This implies convergence to $f:[0,\infty)\to\mathbb{R}$ with $f(x)=x$.\\
Second we state that each $f_n$ is bounded: This is a~consequence of the fact that, by the definition of $f_n$, there holds $f_n(x)<n$ for all $x\in[0,\infty)$.\\
Altogether, we have found a~sequence of bounded functions pointwisely converging to some unbounded function. Hence, Theorem~\ref{thm:unifconvbound} is no longer valid, if we replace the phrase ``uniformly convergent''
by ``pointwisely convergent''.
\end{Remark}

