\usepackage{amsthm}

\newtheorem{theorem}{Theorem}[chapter]
\newtheorem{lemma}           [theorem] {Lemma}   
\newtheorem{folg}           [theorem] {Folgerung}   

\newtheorem{frage}       [theorem] {Frage}   
\newtheorem{question}       [theorem] {Question}   
\newtheorem{aufgabe}       [theorem] {Aufgabe}   
\newtheorem{exercise}       [theorem] {Exercise}  

\newtheorem{proposition}     [theorem] {Proposition}  
\newtheorem{satz}     [theorem] {Satz}  
\newtheorem{fact}{Fact}
\newtheorem{definition}      [theorem] {Definition} 

\theoremstyle{definition} 
\newtheorem{bemerkung}     [theorem] {Bemerkung}  
\newtheorem{beispiel}       [theorem] {Beispiel}  
\newtheorem{example}       [theorem] {Example}  
\newtheorem*{example*} {Example}  
\newtheorem{notation}       [theorem] {Notation}  
\newtheorem*{Faust}[theorem]{Rule of Thumb}
\newtheorem*{Boxx}[theorem]{Concept}

We will now give an example of an integral that will be computed according to the definition. This will turn out to be really exhausting even for this quite simple example.

\begin{example}
Consider the function $f:[0,1]\to\mathbb{R}$ with $f(x)=x$. Determine $\int_0^1f(x)\, dx=\int_0^1x\, dx$.
First we consider two sequences of step functions $(\phi_n)_{n\in\mathbb{N}}$, $(\psi_n)_{n\in\mathbb{N}}$
with
\[
\begin{aligned}
\phi_n(x)&=\frac{k-1}n\text{ for }x\in\left[\frac{k-1}n,\frac{k}n\right),\qquad k\in\{1,\ldots,n\},\\
\psi_n(x)&=\frac{k}n\text{ for }x\in\left[\frac{k-1}n,\frac{k}n\right),\qquad k\in\{1,\ldots,n\}.
\end{aligned}
\]
Then for all $n\in\mathbb{N}$ holds $\phi_n\leq f\leq \psi_n$. Now we calculate
\[
\begin{aligned}
\int_0^1\phi_n(x)\, dx&=\sum_{k=1}^n\frac{k-1}n\cdot\left(\frac{k}n-\frac{k-1}n\right)\\
&=
\sum_{k=1}^n\frac{k-1}{n}\cdot\frac{1}n\\&=\frac{1}{n^2}\sum_{k=1}^n(k-1)\\&=\frac{1}{n^2}\cdot\frac{n(n-1)}2=\frac12-\frac{1}{2n}
\end{aligned}
\]
and
\[\begin{aligned}
\int_0^1\psi_n(x)\, dx&=\sum_{k=1}^n\frac{k}n\cdot\left(\frac{k}n-\frac{k-1}n\right)\\&=
\sum_{k=1}^n\frac{k}{n}\cdot\frac{1}n\\&=\frac{1}{n^2}\sum_{k=1}^nk\\&=\frac{1}{n^2}\cdot\frac{n(n+1)}2=\frac12+\frac{1}{2n}.
\end{aligned}
\]
In particular, we have for all $n\in\mathbb{N}$ that
\[\frac12-\frac{1}{2n}=\int_0^1\phi_n(x)\, dx\leq\int_0^1x\, dx\leq \int_0^1\psi_n(x)\, dx=\frac12+\frac{1}{2n}\]
and thus
\[\int_0^1x\, dx=\frac12.\]
\end{example}

By the definition of the integral, it is not difficult to obtain that for $f\in\mathcal{R}([a,b])$ and $c\in(a,b)$ holds
\[\int_a^bf(x)\, dx=\int_a^cf(x)\, dx+\int_c^bf(x)\, dx.\]
To make this formula also valid for $c\geq b$ or $c\leq a$, we define for $a\geq b$ that
\[\int_a^bf(x)\, dx:=-\int_b^af(x)\, dx.\]

