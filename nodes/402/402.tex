\usepackage{amsthm}

\newtheorem{theorem}{Theorem}[chapter]
\newtheorem{lemma}           [theorem] {Lemma}   
\newtheorem{folg}           [theorem] {Folgerung}   

\newtheorem{frage}       [theorem] {Frage}   
\newtheorem{question}       [theorem] {Question}   
\newtheorem{aufgabe}       [theorem] {Aufgabe}   
\newtheorem{exercise}       [theorem] {Exercise}  

\newtheorem{proposition}     [theorem] {Proposition}  
\newtheorem{satz}     [theorem] {Satz}  
\newtheorem{fact}{Fact}
\newtheorem{definition}      [theorem] {Definition} 

\theoremstyle{definition} 
\newtheorem{bemerkung}     [theorem] {Bemerkung}  
\newtheorem{beispiel}       [theorem] {Beispiel}  
\newtheorem{example}       [theorem] {Example}  
\newtheorem*{example*} {Example}  
\newtheorem{notation}       [theorem] {Notation}  
\newtheorem*{Faust}[theorem]{Rule of Thumb}
\newtheorem*{Boxx}[theorem]{Concept}

%\section{Polynomials}
\begin{Definition}{Polynomials}
Let $a_0,\ldots,a_n\in\mathbb{K}$. Then a~real (complex) \emph{polynomial} is a~function $p:\mathbb{K}\to\mathbb{K}$ with
\[p(x)=\sum_{k=0}^na_kx^k.\]
If $a_n\neq0$, then $a_n$ is called \emph{leading coefficient} and $n$ is called \emph{degree of $p$}. 
We write $n=:\deg p$. If $p$ is the zero polynomial, we set $\deg p:=-\infty$.\\
The set of polynomials in $\mathbb{K}$ is denoted by $\mathbb{K}[x]$. Moreover, we set
\[\mathbb{K}_n[x]:=\{p\in\mathbb{K}[x]\;|\;\deg p\leq n\}.\]
\end{Definition}
\begin{Remark}{}
Since sums and scalar multiples of polynomials are again polynomials, they form a~vector space.
\end{Remark}
\begin{Theorem}{Rules for the degree}
For $p,q\in\mathbb{K}[x]$ holds
\[\deg(p\cdot q)=\deg(p)+\deg(q),\qquad \deg(p+q)\leq\max\{\deg(p),\deg(q)\}.\]
\end{Theorem}
\begin{proof}
Let $p(x)=\sum_{k=0}^na_kx^k$, $q(x)=\sum_{k=0}^mb_kx^k$ with $a_n\neq0$, $b_m\neq0$. The formula for $\deg(p\cdot q)$ follows from
\[p(x)\cdot q(x)=\sum_{k=0}^{n+m}c_kx^k\;\;\text{ for }c_k=\sum_{l=0}^ka_lb_{k-l} \ ,\]
where $a_r:=0=:b_s$ for $r\not\in\{0,...,n\}$ and $s\not\in\{0,...,m\}$.  
For the proof of the formula $\deg(p+q)$, we assume without loss of generality that $n\geq m$. Then
\[p(x)+q(x)=\sum_{k=0}^{m}(a_k+b_k)x^k+\sum_{k=m+1}^{n}a_kx^k.\]
As~a consequence, we have $\deg(p+q)\leq n=\max\{\deg(p),\deg(q)\}$.
\end{proof}
\begin{Remark}{}
As the example $p(x)=x$ and $q(x)=-x+1$ shows, it may indeed happen that $\deg(p+q)<\max\{\deg(p),\deg(q)\}$.\\
Since $\deg(0\cdot p)=\deg(0)=-\infty=-\infty+\deg(p)$ and $\deg(0+p)=\deg(p)=\max\{-\infty,\deg(p)\}$, the choice of $\deg p=-\infty$ 
makes indeed sense for preserving the above formulas. However, this belongs to the ``not so important facts'' of mathematical analysis.
\end{Remark}

