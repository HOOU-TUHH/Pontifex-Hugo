\usepackage{amsthm}

\newtheorem{theorem}{Theorem}[chapter]
\newtheorem{lemma}           [theorem] {Lemma}   
\newtheorem{folg}           [theorem] {Folgerung}   

\newtheorem{frage}       [theorem] {Frage}   
\newtheorem{question}       [theorem] {Question}   
\newtheorem{aufgabe}       [theorem] {Aufgabe}   
\newtheorem{exercise}       [theorem] {Exercise}  

\newtheorem{proposition}     [theorem] {Proposition}  
\newtheorem{satz}     [theorem] {Satz}  
\newtheorem{fact}{Fact}
\newtheorem{definition}      [theorem] {Definition} 

\theoremstyle{definition} 
\newtheorem{bemerkung}     [theorem] {Bemerkung}  
\newtheorem{beispiel}       [theorem] {Beispiel}  
\newtheorem{example}       [theorem] {Example}  
\newtheorem*{example*} {Example}  
\newtheorem{notation}       [theorem] {Notation}  
\newtheorem*{Faust}[theorem]{Rule of Thumb}
\newtheorem*{Boxx}[theorem]{Concept}

Now we begin to introduce the concept of continuity. 

\begin{Definition}[(One-sided) limits of Functions]
Let $I\subset\mathbb{K}$, let $f:I\to\mathbb{K}$ be a~function, and let $x_0\in I$. Then we define
\begin{enumerate}[(i)]
 \item \emph{the limit of $f$ as $x$ tends to $x_0$} by $c\in\mathbb{K}$
if \textbf{for all} sequences $(x_n)_{n\in\mathbb{N}}$ in $I\backslash\{x_0\}$ with $\lim_{n\to\infty}x_n=x_0$ holds
$\lim_{n\to\infty}f(x_n)=c$. In this case, we write
\[\lim_{x\to x_0}f(x)=c\]
 \item \emph{the limit of $f$ as $x$ tends from the left to $x_0$} by $c\in\mathbb{K}$
if $I\subset \mathbb{R}$ and if \textbf{for all} sequences $(x_n)_{n\in\mathbb{N}}$ in $\{x\in I\,:\,x< x_0\}$ with $\lim_{n\to\infty}x_n=x_0$ holds
$\lim_{n\to\infty}f(x_n)=c$. In this case, we write
\[\lim_{x\nearrow x_0}f(x)=c.\]
 \item \emph{the limit of $f$ as $x$ tends from the right to $x_0$} by $c\in\mathbb{K}$
if $I\subset \mathbb{R}$ and if \textbf{for all} sequences $(x_n)_{n\in\mathbb{N}}$ in $\{x\in I\,:\,x>x_0\}$ with $\lim_{n\to\infty}x_n=x_0$ holds
$\lim_{n\to\infty}f(x_n)=c$. In this case, we write
\[\lim_{x\searrow x_0}f(x)=c.\]
\end{enumerate}
In all three cases we assume that at least one sequence $(x_n)_{n\in\mathbb{N}}$ with the stated property exists.
\end{Definition}

\begin{Remark}{}
From the above definition, we can also conclude that $\lim_{x\to x_0}f(x)$ exists in the case $I\subset \mathbb{R}$ if and only if $\lim_{x\nearrow x_0}f(x)$ and $\lim_{x\searrow x_0}f(x)$ exist and are equal. In this case, there holds
\[\lim_{x\nearrow x_0}f(x)=\lim_{x\searrow x_0}f(x)=\lim_{x\to x_0}f(x).\]
Though not explicitly introduced in the above definition, it should be intuitively clear what is meant by the following expressions
\[\lim_{x\to\infty}f(x)=y,\qquad \lim_{x\to-\infty}f(x)=y,\qquad \lim_{x\to x_0}f(x)=\infty,\qquad \lim_{x\to x_0}f(x)=-\infty.\]
\end{Remark}

\begin{example}\label{ex:funclim}
\begin{enumerate}[a)]
 \item Consider the {\em Heaviside function} $H:\mathbb{R}\to\mathbb{R}$ with
\[H(x)=\begin{cases}1&,\text{ if }x\geq 0,\\0&,\text{ if }x<0.\end{cases}\]
Then we have $\lim_{x\nearrow 0}H(x)=0$, since for all $x_n\in\mathbb{R}$ with $x_n<0$ holds $H(x_n)=0$.
Further, $\lim_{x\searrow 0}H(x)=1$, since for all $x_n\in\mathbb{R}$ with $x_n>0$ holds $H(x_n)=1$. The limit $\lim_{x\to 0}H(x)$ does not exist. E.g., take the sequence $x_n=\frac{(-1)^n}n$. Then
\[H(x_n)=\begin{cases}1:&\text{if $n$ is even,}\\0:&\text{if $n$ is odd.}\end{cases}\]
Hence, $(H(x_n))_{n\in\mathbb{N}}$ is divergent.

\item Consider the function $f:\mathbb{R}\to\mathbb{R}$ with
\[f(x)=\begin{cases}1&,\text{ if }x= 0,\\0&,\text{ if }x\neq0.\end{cases}\]
Then for all sequences $(x_n)_{n\in\mathbb{N}}$ in $\mathbb{R}\backslash\{0\}$ holds that $(f(x_n))_{n\in\mathbb{N}}$ is a~constant zero sequence. Hence, $\lim_{x\to 0}f(x)=0$.

 \item Consider a {\em polynomial} $p:\mathbb{R}\to\mathbb{R}$ with $p(x)=a_nx^n+\ldots+a_1x+a_0$ for some given $a_0,\ldots,a_n\in\mathbb{R}$. Let $x_0\in\mathbb{R}$.  
By Theorem~\ref{thm:limformnormed}, we have that for
all real sequences $(x_n)_{n\in\mathbb{N}}$ converging to $x_0$ holds that $p(x_n)$ converges to $p(x_0)$, i.e.,
\[\lim_{x\to x_0}p(x)=p(x_0).\]
\end{enumerate}
\end{example}

