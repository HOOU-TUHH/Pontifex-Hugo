\usepackage{amsthm}

\newtheorem{theorem}{Theorem}[chapter]
\newtheorem{lemma}           [theorem] {Lemma}   
\newtheorem{folg}           [theorem] {Folgerung}   

\newtheorem{frage}       [theorem] {Frage}   
\newtheorem{question}       [theorem] {Question}   
\newtheorem{aufgabe}       [theorem] {Aufgabe}   
\newtheorem{exercise}       [theorem] {Exercise}  

\newtheorem{proposition}     [theorem] {Proposition}  
\newtheorem{satz}     [theorem] {Satz}  
\newtheorem{fact}{Fact}
\newtheorem{definition}      [theorem] {Definition} 

\theoremstyle{definition} 
\newtheorem{bemerkung}     [theorem] {Bemerkung}  
\newtheorem{beispiel}       [theorem] {Beispiel}  
\newtheorem{example}       [theorem] {Example}  
\newtheorem*{example*} {Example}  
\newtheorem{notation}       [theorem] {Notation}  
\newtheorem*{Faust}[theorem]{Rule of Thumb}
\newtheorem*{Boxx}[theorem]{Concept}

\begin{Definition}[Interior{,} closure{,} boundary]
For $C\subset \mathbb{K}$, we define the
\begin{enumerate}[(i)]
 \item \emph{interior of $C$} by the set
\[\mathring{C}=\{x\in C\,:\,\text{ there exists some }\varepsilon>0\text{ such that }B_{\varepsilon}(x)\subset C\}.\]
The elements of $\mathring{C}$ are called \textit{inner points of $C$}.
 \item \emph{closure of $C$} by the set
   \[\overline{C}=\{x\in \mathbb{K}\,:\,\text{ there exists a sequ. }(a_n)_{n\in\mathbb{N}}\text{ in $C$ with }\lim_{n\to\infty}a_{n}=x\}.\]
The elements of $\overline{C}$ are called \emph{osculation points of $C$}.
 \item \emph{boundary of $C$} by the set
\[\partial C=\overline{C}\backslash \mathring{C}.\]
The elements of $\partial{C}$ are called \emph{boundary points of $C$}.
\end{enumerate}
\end{Definition}
\begin{Remark}{}
  The relation $\mathring{C}\subset C\subset \overline{C}$ holds true for arbitrary subsets $C\subset \mathbb{K}$. The first inclusion holds true by definition of $\mathring{C}$. To verify $C\subset \overline{C}$ we take an arbitrary $x\in C$ and consider the constant sequence $(x)_{n\in\mathbb{N}}$. Since this sequence is completely contained in $C$ and converges to $x\in C$, we must have that $x\in\overline{C}$.\\
It can be shown that for all sets $C$, $\mathring{C}$ is always open and $\overline{C}, \partial C$ are always closed sets. In particular, if $C$ is open (closed), then we have $\mathring{C}=C$ (resp.\ $\overline{C}=C$)
\end{Remark}

\begin{example}
These are examples in the case $\mathbb{K} = \mathbb{R}$:
\begin{enumerate}[(a)]
 \item $C=[0,1]$, then $\mathring{C}=(0,1)$, $\overline{C}=[0,1]$ and $\partial C=\{0,1\}$;
 \item $C=\{\frac1n\,:\,n\in\mathbb{N}\}$, then $\mathring{C}=\emptyset$ and $\overline{C}=\partial C=\{0\}\cup \{\frac1n\,:\,n\in\mathbb{N}\}$.
\end{enumerate}
\end{example}

