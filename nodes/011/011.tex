\usepackage{amsthm}

\newtheorem{theorem}{Theorem}[chapter]
\newtheorem{lemma}           [theorem] {Lemma}   
\newtheorem{folg}           [theorem] {Folgerung}   

\newtheorem{frage}       [theorem] {Frage}   
\newtheorem{question}       [theorem] {Question}   
\newtheorem{aufgabe}       [theorem] {Aufgabe}   
\newtheorem{exercise}       [theorem] {Exercise}  

\newtheorem{proposition}     [theorem] {Proposition}  
\newtheorem{satz}     [theorem] {Satz}  
\newtheorem{fact}{Fact}
\newtheorem{definition}      [theorem] {Definition} 

\theoremstyle{definition} 
\newtheorem{bemerkung}     [theorem] {Bemerkung}  
\newtheorem{beispiel}       [theorem] {Beispiel}  
\newtheorem{example}       [theorem] {Example}  
\newtheorem*{example*} {Example}  
\newtheorem{notation}       [theorem] {Notation}  
\newtheorem*{Faust}[theorem]{Rule of Thumb}
\newtheorem*{Boxx}[theorem]{Concept}
%\subsection*{Sums and products}
We will use the following notations.
\begin{align*}
 \sum_{i=1}^n a_i &= a_1+a_2+\dots+a_{n-1}+a_n\\
 \prod_{i=1}^n a_i &= a_1 \cdot a_2 \cdot \dots \cdot a_{n-1}\cdot a_n\\
 \bigcup_{i=1}^n A_i &= A_1 \cup A_2 \cup \dots \cup A_{n-1}\cup A_n
\end{align*}
The union works also for an arbitrary index set $\mathcal{I}$:
$$
	\bigcup_{i \in \mathcal{I}} A_i = \{ x \,:\, \exists i \in \mathcal{I} \text{ with } x \in A_i \} \,.
$$

The first is a useful notation for a \emph{sum} which is the result of an addition. Two or
more \emph{summands} added. Instead of using points, we use the Greek letter $\sum$. For example,
\[
3+7+15+\ldots+127 
\]
is not an unambiguous way to describe the sum. 
Using the sum symbol, there is no confusion:
\begin{equation*} %\label{eq:sum1}
\sum_{i=2}^7 (2^i-1).
\end{equation*}
Of course, the parentheses are necessary here.
You can read this as a {\texttt{for}} loop:

\begin{Boxx}[for loop for the sum above]
\textnormal{\texttt{
\hspace*{10mm}sum := 0;\\
\hspace*{10mm}for i:=2 to 7 do \{\\
\hspace*{15mm}sum := sum + 2^i-1);\\
\hspace*{10mm}\}
}
}
\end{Boxx}

\begin{Faust}
Let $i$ run from $2$ to $7${,} calculate $2^i-1$
and add.
%\begin{tabular}{|ccrrrr|}
\begin{tabular}{ccrrrr}
%\hline
index variable: & $i=2$, & first summand: & $2^i-1=2^2-1=$&$4-1=$&$3$\\
index variable: & $i=3$, & second summand: & $2^i-1=2^3-1=$&$8-1=$&$7$\\
index variable: & $i=4$, & third summand: & $2^i-1=2^4-1=$&$16-1=$&$15$\\
% & $\vdots$ & & $\vdots$\\
index variable: & $i=5$, & fourth summand: & $2^i-1=2^5-1=$&$32-1=$&$31$\\
index variable: & $i=6$, & fifth summand: & $2^i-1=2^6-1=$&$64-1=$&$63$\\
index variable: & $i=7$, & last summand: & $2^i-1=2^7-1=$&\!\!$128-1=$&$127$\\
\hline
 & & Sum: & & &$246$
\end{tabular}
\end{Faust}

\begin{example}{}
\[
\sum\limits_{i=1}^{10} (2i-1) = 1+3+5+\ldots+19 \stackrel?= 100
\]
\[
\sum\limits_{i=-10}^{10} i = -10-9-8-\ldots-1+0+1+\dots+8+9+10 \stackrel?=0
\]
\end{example}

With the same construction,
we describe the result of a multiplication, called a \emph{product}, which consists of two or
more \emph{factors}. There we use the Greek letter $\prod$. 
For example:
\[
\prod_{i=1}^8 (2i) = (2\cdot 1)\cdot (2\cdot 2)\cdot (2\cdot 3)\cdot \ldots \cdot (2\cdot 8) 
\stackrel?= 10321920.
\]

