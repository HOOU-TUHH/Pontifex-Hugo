\usepackage{amsthm}

\newtheorem{theorem}{Theorem}[chapter]
\newtheorem{lemma}           [theorem] {Lemma}   
\newtheorem{folg}           [theorem] {Folgerung}   

\newtheorem{frage}       [theorem] {Frage}   
\newtheorem{question}       [theorem] {Question}   
\newtheorem{aufgabe}       [theorem] {Aufgabe}   
\newtheorem{exercise}       [theorem] {Exercise}  

\newtheorem{proposition}     [theorem] {Proposition}  
\newtheorem{satz}     [theorem] {Satz}  
\newtheorem{fact}{Fact}
\newtheorem{definition}      [theorem] {Definition} 

\theoremstyle{definition} 
\newtheorem{bemerkung}     [theorem] {Bemerkung}  
\newtheorem{beispiel}       [theorem] {Beispiel}  
\newtheorem{example}       [theorem] {Example}  
\newtheorem*{example*} {Example}  
\newtheorem{notation}       [theorem] {Notation}  
\newtheorem*{Faust}[theorem]{Rule of Thumb}
\newtheorem*{Boxx}[theorem]{Concept}

The next result now states that integrals can be determined by inversion of differentiation.
\begin{Theorem}[Fundamental theorem of differentiation and integration]
Let $I$ be an interval and a~continuous $f:I\to\mathbb{R}$ be given. Let $F:I\to\mathbb{R}$ be an antiderivative of $f$. Then for all $a,b\in I$ holds
\[\int_a^bf(x)\, dx=F(b)-F(a).\]
We write
\[\int_a^bf(x)\, dx=\left. F(x)\right|_{x=a}^{x=b}.\]
\end{Theorem}
\white{13cm}{
{\em Proof:} Consider the function $F_0: I\to\mathbb{R}$ defined by
\[F_0(x)=\int_a^xf(\xi)d\xi.\]
By Theorem~\ref{thm:antider}, we know that $F_0$ is an antiderivative of $f$. 
In particular, we have that $F_0(a)=\int_a^af(\xi)d\xi=0$ and thus $F_0(b)-F_0(a)=F_0(b)=\int_a^bf(\xi)d\xi=\int_a^bf(x)\, dx$. 
Let $FI:\to\mathbb{R}$ be an antiderivative of $f$. Theorem~\ref{thm:antider_diff} now implies that there exists some 
$c\in\mathbb{R}$ with $F(x)=F_0(x)+c$ for all $x\in I$. Therefore
 \[F(b)-F(a)=(F_0(b)+c)-(F_0(a)+c)=F_0(b)-F_0(a)=\int_a^bf(x)\, dx.\]
\hfill$\Box$
}

The above result gives rise to the following notation for an antiderivative:
\[\int f(x)\, dx:=F(x).\]
Based on our knowledge about differentiation, we now collect some antiderivatives of important functions in the following table.
\renewcommand{\arraystretch}{1.5}
\begin{table}[h!]\label{tab:SomeAntiderivatives}\begin{center}
\begin{tabular}{|c|c|}
\hline $f(x)$&$\int f(x)\, dx$\\
\hline\hline&\\[-0.7cm]
    $\displaystyle x^n,\quad n\in\mathbb{N}$& $\displaystyle \frac1{n+1}x^{n+1}$\\\hline
$\displaystyle x^{-1},\quad x\neq0$& $\displaystyle \log(|x|)$\\\hline
    $\displaystyle x^{-n},\quad x\neq0,n\in\mathbb{N}, n\neq1$& $\displaystyle \frac1{1-n}x^{1-n}$\\\hline
$\displaystyle \exp(x)$& $\displaystyle \exp(x)$\\\hline
$\displaystyle \sinh(x)$& $\displaystyle \cosh(x)$\\\hline
$\displaystyle \cosh(x)$& $\displaystyle \sinh(x)$\\\hline
$\displaystyle \frac1{\sqrt{1+x^2}}$&$\displaystyle\operatorname{arsinh}(x)$\\\hline 
$\displaystyle \frac1{\sqrt{x^2-1}},\quad x>1$&$\displaystyle\operatorname{arcosh}(x)$\\\hline
$\displaystyle \frac1{1-x^2},\quad |x|<1$&$\displaystyle\operatorname{artanh}(x)$\\\hline
$\displaystyle \sin(x)$& $\displaystyle-\cos(x)$\\\hline
$\displaystyle \cos(x)$& $\displaystyle\sin(x)$\\\hline
$\displaystyle \frac1{\cos^2(x)}=1+\tan^2(x)$& $\displaystyle \tan(x)$\\\hline
$\displaystyle \frac1{\sqrt{1-x^2}},\quad|x|<1$& $\displaystyle \arcsin(x)$\\\hline
$\displaystyle -\frac1{\sqrt{1-x^2}},\quad|x|<1$& $\displaystyle \arccos(x)$\\\hline
$\displaystyle \frac1{1+x^2}$& $\displaystyle \arctan(x)$\\\hline
\end{tabular}~\\~\\\caption{Some antiderivatives}\label{tab:antider}\end{center}
\end{table}
\renewcommand{\arraystretch}{1}


