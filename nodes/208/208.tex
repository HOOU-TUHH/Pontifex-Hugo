\usepackage{amsthm}

\newtheorem{theorem}{Theorem}[chapter]
\newtheorem{lemma}           [theorem] {Lemma}   
\newtheorem{folg}           [theorem] {Folgerung}   

\newtheorem{frage}       [theorem] {Frage}   
\newtheorem{question}       [theorem] {Question}   
\newtheorem{aufgabe}       [theorem] {Aufgabe}   
\newtheorem{exercise}       [theorem] {Exercise}  

\newtheorem{proposition}     [theorem] {Proposition}  
\newtheorem{satz}     [theorem] {Satz}  
\newtheorem{fact}{Fact}
\newtheorem{definition}      [theorem] {Definition} 

\theoremstyle{definition} 
\newtheorem{bemerkung}     [theorem] {Bemerkung}  
\newtheorem{beispiel}       [theorem] {Beispiel}  
\newtheorem{example}       [theorem] {Example}  
\newtheorem*{example*} {Example}  
\newtheorem{notation}       [theorem] {Notation}  
\newtheorem*{Faust}[theorem]{Rule of Thumb}
\newtheorem*{Boxx}[theorem]{Concept}

\begin{Theorem}[Quotient criterion]
\label{thm:quotkrit}
Let $n_0\in\mathbb{N}$ and let $\sum_{k=1}^\infty a_k$ be a~series in $\mathbb{K}$ with the following properties:
\begin{itemize}
 \item[--] $a_k\neq0$ for all $k\geq n_0$;
 \item[--] there exists some $q\in(0,1)$ such that for all $k\geq n_0$ holds
\[\frac{|a_{k+1}|}{|a_{k}|}\leq q.\]
\end{itemize}
Then $\sum_{k=1}^\infty a_k$ converges absolutely.
\end{Theorem}
%
{\em Proof:} We inductively obtain that for all $k\geq n_0$ holds
\[|a_k|\leq q^{k-n_0}|a_{n_0}| \ .\]
Therefore, the series $\sum_{k=1}^\infty q^{k-n_0}|a_{n_0}|$ is a majorant of $\sum_{k=1}^\infty a_k$. However, the majorant is convergent due to
\[\sum_{k=1}^\infty q^{k-n_0}|a_{n_0}|=\frac{|a_{n_0}|}{(1-q)q^{n_0-1}}\qquad\text{(see Example~\ref{ex:geomharm} a))}.\]
The majorant criterion then implies convergence.$\Box$

\begin{Remark}{}
Note that the quotient criterion is different from claiming $\frac{|a_{k+1}|}{|a_{k}|}<1$ (which is for instance fulfilled by the divergent harmonic series). 
There has to exist some $q<1$ such that the quotient is below $q$.\\
The quotient criterion is only sufficient for convergence and indeed, there are examples of absolutely convergent 
series that do not fulfill the quotient criterion. For instance, consider the absolutely convergent series
\[\sum_{k=1}^\infty \frac1{k^2}.\]
Observing that
\[\frac{|a_{k+1}|}{|a_{k}|}=\left|\frac{k}{k+1}\right|^2,\]
the fact that this expression converges to $1$ implies that there does not exist some $q<1$ for which the quotient criterion is fulfilled.
 However, this series is convergent as we have proven in the Example on the geometric series..\\
We could also formulate ``an alternative quotient criterion'' that gives us a~sufficient criterion for divergence. 
Namely, consider a~real series $\sum_{k=1}^\infty a_k$ with positive $a_k$ and assume that $\frac{a_{k+1}}{a_{k}}\geq 1$ 
for all $k\geq n_0$ for some fixed $n_0\in\mathbb{N}$. This gives us $0< a_{k}\leq a_{k+1}$, i.e., the sequence $(a_n)_{n\in\mathbb{N}}$ is positive and monotonically increasing. 
Such a~sequence cannot converge to zero and thus, $\sum_{k=1}^\infty a_k$ is divergent.
\end{Remark}
Now we present a~``limit form'' of the quotient criterion:
\begin{Theorem}{Quotient criterion (limit form)}
\label{thm:quotkritlim}
Let $\sum_{k=1}^\infty a_k$ be a~series in $\mathbb{K}$ and assume that there exists some $n_0\in \mathbb{N}$ such that $a_k\neq0$ for all $k\geq n_0$.
If\[\limsup_{k \rightarrow \infty} \frac{|a_{k+1}|}{|a_{k}|}<1,\]
then $\sum_{k=1}^\infty a_k$ converges absolutely.
\end{Theorem}
%
{\em Proof:} Set $c:=\lim\sup\frac{|a_{k+1}|}{|a_{k}|}<1$. Since $\lim\sup$ is defined to be the largest accumulation point of a~sequence, 
we have for every $\varepsilon>0$ that $\frac{|a_{k+1}|}{|a_{k}|}\geq c+\varepsilon$ holds true for at most finitely many $k\in\mathbb{N}$. 
Hence, for $\varepsilon:=\frac{1-c}2,$ there exists some $N\in\mathbb{N}$ such that for all $k\geq N$ holds
\[\frac{|a_{k+1}|}{|a_{k}|}<c+\varepsilon=c+\frac{1-c}2=\frac{1+c}2.\]
Thus, the quotient criterion holds true for $q:=\frac{1+c}2$ which satisfies $q<1$ due to $c<1$.

$\Box$

\begin{Remark}{}
Since, in case of convergence of $\frac{|a_{k+1}|}{|a_{k}|}$, the limit and limes superior coincide, the criterion
\[\lim_{k\to\infty}\frac{|a_{k+1}|}{|a_{k}|}<1\]
is also sufficient for absolute convergence of $\sum_{k=1}^\infty a_k$. However, this criterion requires the convergence of the quotient sequence and is therefore weaker than the above one.
\end{Remark}


\begin{example}\label{ex:expser}
 For $x\in\mathbb{K}$, consider the series
\[\sum_{k=0}^\infty\frac{x^k}{k!}.\]
Then an application of the limit form of the quotient criterion yields
\[\begin{aligned}
&\,\limsup_{k \rightarrow \infty}\frac{\left|\frac{x^{k+1}}{(k+1)!}\right|}{\left|\frac{x^k}{k!}\right|}
=\limsup_{k \rightarrow \infty}\frac{|x|^{k+1}}{|x|^k}\frac{k!}{(k+1)!}\\
=&\,\limsup_{k \rightarrow \infty}|x|\frac1{k+1}
=\,\lim_{k\to\infty}|x|\frac1{k+1}=0<1.
\end{aligned}\]
Hence, the series converges absolutely. The series $\sum_{k=0}^\infty\frac{x^k}{k!}$ is called \emph{exponential series} and we will indeed define $\exp(x)$ by this expression.
\end{example}

\begin{Theorem}{}
Let $x\in\mathbb{K}$. Then
$$
\lim_{n\rightarrow\infty}\left(1+\frac{x}{n}\right)^{n} = \sum_{k=0}^{\infty}\frac{x^{k}}{k!}.
$$
\end{Theorem}
{\em Proof:}
First we show that for fixed $k\in\mathbb{N}_0$ holds
\begin{eqnarray}\label{seqexp}
 \lim_{n\rightarrow \infty} \frac{n!}{(n-k)!n^k} = 1.
\end{eqnarray}\label{seqexp2}
Note that $\frac{n!}{(n-k)!n^k}$ is well defined for $n\geq k$. For such an $n\in\mathbb{N}$ we have
\begin{eqnarray}
  1\geq \prod_{j=0}^{k-1} \frac{n-j}{n} = \frac{n!}{(n-k)!n^k}\geq \frac{(n-k+1)^k}{n^k}=(1-\frac{k-1}{n})^k \ .
\end{eqnarray}
Since the right-hand side converges to $1$ for $n\rightarrow \infty$, the claim of the theorem follows.
\\ \\
Now let $\varepsilon>0$ and let $K\in\mathbb{N}$ be big enough such that 
$$
\sum_{k=K}^{\infty}\frac{|x|^{k}}{k!}<\frac{\varepsilon}{3}.
$$
By the convergence of the auxiliary sequence $\frac{n!}{(n-k)!n^k}$, there is an $N\geq K$ such that for all $n\geq N$ also
$$\sum_{k=0}^{K-1}\left|\frac{n!}{(n-k)!n^k}-1\right|\frac{|x|^{k}}{k!} <\frac{\varepsilon}{3} \ . $$

For $n\geq N$ we estimate
\begin{align*}
\left|\left(1+\frac{x}{n}\right)^{n}-\sum_{k=0}^{\infty}\frac{x^{k}}{k!}\right| 
&= \left|\sum_{k=0}^{n}\binom{n}{k}\frac{x^{k}}{n^{k}}-\sum_{k=0}^{\infty}\frac{x^{k}}{k!} \right|\\
&\leq \sum_{k=0}^{K-1}\left|\binom{n}{k}\frac{x^{k}}{n^{k}}-\frac{x^{k}}{k!}\right|+\sum_{k=K}^{n} \binom{n}{k}\frac{|x|^{k}}{n^{k}}
+\underbrace{\sum_{k=K}^{\infty}\frac{|x|^{k}}{k!}}_{<\,\varepsilon/3}. \\
&= \underbrace{\sum_{k=0}^{K-1}\left|\frac{n!}{(n-k)!n^k}-1\right|\frac{|x|^{k}}{k!}}_{<\,\varepsilon/3}
+\underbrace{\sum_{k=K}^{n} \underbrace{\frac{n!}{(n-k)!n^k}}_{\leq 1}\frac{|x|^{k}}{k!}}_{<\,\varepsilon/3}
+\underbrace{\sum_{k=K}^{\infty}\frac{|x|^{k}}{k!}}_{<\,\varepsilon/3} \\ & < \varepsilon.
\end{align*}

$\Box$

The following criterion of Raabe refines the quotient criterion.

\begin{Theorem}[Raabe criterion] \label{th:raabecrit}
Let $(a_k)_{k\in\mathbb{N}}$ be  a sequence in $\mathbb{K}$.
\begin{itemize}
 \item[a)] If there is a $k_0\in\mathbb{N}$ and a $\beta\in\;(1,\infty)$ such that $a_k\neq 0$ and 
          $$\frac{|a_{k+1}|}{|a_k|} \leq 1-\frac{\beta}{k}$$ for all $k\geq k_0$, 
          then the series $\sum_{k=1}^{\infty}a_k$ converges absolutely.
 \item[b)] If $\mathbb{K}=\mathbb{R}$ and if there is a $k_0\in\mathbb{N}$ such that $a_k\neq 0$ and 
           $$\frac{a_{k+1}}{a_k} \geq 1-\frac{1}{k}$$ for all $k\geq k_0$, 
           then the series $\sum_{k=1}^{\infty}a_k$ diverges.
\end{itemize}
\end{Theorem}
{\em Proof:}
  a) For $k\geq k_0$ holds $\frac{|a_{k+1}|}{|a_k|} \leq \frac{k-\beta}{k}$. Thus 
     $k|a_{k+1}|\leq (k-\beta)|a_k|$ and therefore $\beta|a_k|\leq k |a_k|-k|a_{k+1}|$. 
     Since $\beta>1$ and $|a_k|>0$ subtracting $|a_k|$ from both sides of this inequality yields
     \begin{eqnarray}\label{l1:raabecrit}
	0<(\beta-1)|a_k|\leq (k-1) |a_k|-k|a_{k+1}|=:b_k \ .
     \end{eqnarray}
     In particular, this shows that $(k|a_{k+1}|)_{k\geq k_0-1}$ is monotonically decreasing. 
     Since it is bounded from below by zero, it must converge to some limit $s:=\lim_{k\rightarrow \infty}k|a_{k+1}|$. 
     Thus also the telescoping series $\sum_{k=k_0}^\infty b_k$ converges to  
     $\sum_{k=k_0}^\infty (k-1) |a_k|-k|a_{k+1}| = (k_0-1)|a_{k_0}|-s$.
     From the above estimation and definition of $b_k$, we conclude that $\frac{1}{\beta-1}\sum_{k=k_0}^\infty b_k$ is a convergent majorant
     of $\sum_{k=k_0}^\infty |a_k|$. By the majorant criterion, $\sum_{k=1}^\infty a_k$ is absolutely convergent.
\\ \\
  b) We may assume that $k_0>1$. For $k\geq k_0$ holds $0\neq\frac{a_{k+1}}{a_k} \geq \frac{k-1}{k}>0$. Since the right-hand side is nonnegative,
     $a_{k}$ and $a_{k+1}$ must have the same sign (either + or -). Since for the same reason $a_{k+1}$ and $a_{k+2}$
     must also have the same sign we see that all $a_k$ have the same sign for $k\geq k_0$. Without loss of generality
     we may therefore assume that $a_k>0$ for all $k\geq k_0$. (Otherwise consider the sequence $(-a_k)_{k\in\mathbb{N}}$.)
     Then $ka_{k+1}\geq (k-1)a_k$ shows that $ka_{k+1}\geq (k_0-1)a_{k_0}=:\alpha >0$ for all $k\geq k_0$, that is 
     $a_{k+1}\geq \frac{\alpha}{k}$ for $k\geq k_0$.  
     Consequently, $\sum_{k=k_0}^{\infty}\frac{\alpha}{k}$ is a divergent minorant for the series $\sum_{k=k_0}^{\infty}a_{k+1}$ 
     so that the minorant criterion implies the divergence of $\sum_{k=1}^{\infty}a_k$.
$\Box$

We want to remark that the assumption $\mathbb{K}=\mathbb{R}$ in b) is not essential. The assertion also holds for $\mathbb{K}=\mathbb{C}$. In this case we have 
to argue that for $k\geq k_0$ all $a_k = |a_k|e^{i\varphi_k}$ must have the same argument $\varphi_k\in [0,2\pi)$. Then again without loss
of generality we may assume that $\varphi_k = 0$ for all $k\geq k_0$.

