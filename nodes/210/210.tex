\usepackage{amsthm}

\newtheorem{theorem}{Theorem}[chapter]
\newtheorem{lemma}           [theorem] {Lemma}   
\newtheorem{folg}           [theorem] {Folgerung}   

\newtheorem{frage}       [theorem] {Frage}   
\newtheorem{question}       [theorem] {Question}   
\newtheorem{aufgabe}       [theorem] {Aufgabe}   
\newtheorem{exercise}       [theorem] {Exercise}  

\newtheorem{proposition}     [theorem] {Proposition}  
\newtheorem{satz}     [theorem] {Satz}  
\newtheorem{fact}{Fact}
\newtheorem{definition}      [theorem] {Definition} 

\theoremstyle{definition} 
\newtheorem{bemerkung}     [theorem] {Bemerkung}  
\newtheorem{beispiel}       [theorem] {Beispiel}  
\newtheorem{example}       [theorem] {Example}  
\newtheorem*{example*} {Example}  
\newtheorem{notation}       [theorem] {Notation}  
\newtheorem*{Faust}[theorem]{Rule of Thumb}
\newtheorem*{Boxx}[theorem]{Concept}

\begin{Definition}[Reordering]
 Let $\sum_{k=1}^\infty a_k$ be a~series in $\mathbb{K}$ and let $\tau:\mathbb{N}\to\mathbb{N}$ be a~bijective mapping. Then the series
\[\sum_{k=1}^\infty a_{\tau(k)}\]
is called a~\emph{reordering} of $\sum_{k=1}^\infty a_k$.
\end{Definition}

\begin{Theorem}[Reordering of absolutely convergent series]
Let $\sum_{k=1}^\infty a_k$ be an absolutely convergent series in  $\mathbb{K}$ and let $\sum_{k=1}^\infty a_{\tau(k)}$ be a~reordering. Then $\sum_{k=1}^\infty a_{\tau(k)}$ is also absolutely convergent. Moreover,
\[\sum_{k=1}^\infty a_{\tau(k)}=\sum_{k=1}^\infty a_{k}.\]
\end{Theorem}

{\em Proof:} Let $a=\sum_{k=1}^\infty a_k$ and let $\tau:\mathbb{N}\to\mathbb{N}$ be~bijective. Let $\varepsilon>0$. Since we have absolute convergence, there exists some $N_1$ such that
\[\sum_{k=N_1}^\infty |a_k|<\frac\varepsilon2\]
and thus
\[\left|a-\sum_{k=1}^{N_1-1} a_k\right|=\left|\sum_{k=N_1}^{\infty} a_k\right|\leq\sum_{k=N_1}^{\infty} |a_k|<\frac\varepsilon2.\]
Now choose $N:=\max\{\tau^{-1}(1),\tau^{-1}(2),...,\tau^{-1}(N_1-1)\}$.Then
\[\{1,2,\ldots,N_1-1\}\subset\{\tau(1),\tau(2),\ldots,\tau(N)\}.\]
Then, for all $n\geq N$ holds
\[\left|a-\sum_{k=1}^{n} a_{\tau(k)}\right|\leq
\left|a-\sum_{k=1}^{N_1-1} a_k\right|+\left|\sum_{k=1}^{N_1-1} a_k-\sum_{k=1}^{n} a_{\tau(k)}\right|< \frac\varepsilon2+\sum_{k=N_1}^{\infty} |a_k|<\frac\varepsilon2+\frac\varepsilon2=\varepsilon.
\]
The absolute convergence of $\sum_{k=1}^\infty a_{\tau(k)}$ can be shown by an application of the above argumentation to the series $\sum_{k=1}^\infty |a_k|$ and $\sum_{k=1}^\infty |a_{\tau(k)}|$.
\hfill$\Box$

