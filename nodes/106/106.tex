\usepackage{amsthm}

\newtheorem{theorem}{Theorem}[chapter]
\newtheorem{lemma}           [theorem] {Lemma}   
\newtheorem{folg}           [theorem] {Folgerung}   

\newtheorem{frage}       [theorem] {Frage}   
\newtheorem{question}       [theorem] {Question}   
\newtheorem{aufgabe}       [theorem] {Aufgabe}   
\newtheorem{exercise}       [theorem] {Exercise}  

\newtheorem{proposition}     [theorem] {Proposition}  
\newtheorem{satz}     [theorem] {Satz}  
\newtheorem{fact}{Fact}
\newtheorem{definition}      [theorem] {Definition} 

\theoremstyle{definition} 
\newtheorem{bemerkung}     [theorem] {Bemerkung}  
\newtheorem{beispiel}       [theorem] {Beispiel}  
\newtheorem{example}       [theorem] {Example}  
\newtheorem*{example*} {Example}  
\newtheorem{notation}       [theorem] {Notation}  
\newtheorem*{Faust}[theorem]{Rule of Thumb}
\newtheorem*{Boxx}[theorem]{Concept}


Now we show that Cauchy sequences in $\mathbb{K}$ are even convergent:
\begin{Theorem}{}\label{thm:Rcompl}
Every Cauchy sequence $(a_n)_{n\in\mathbb{N}}$ in $\mathbb{K}$ converges.
\end{Theorem}

{\em Proof:} 
Every Cauchy sequence is bounded. Therefore, $(a_n)_{n\in\mathbb{N}}$ is bounded. By the Theorem of Bolzano-Weierstra\ss \ it has a convergent 
subsequence $(a_{n_k})_{k\in\mathbb{N}}$. Set $a:=\lim_{k\rightarrow\infty} a_{n_k}$. For given $\varepsilon>0$ there exist $N_1,N_2\in\mathbb{N}$ such 
that $|a_{n_k}-a|<\varepsilon/2$ for all $k\geq N_1$ and $|a_n-a_m|<\varepsilon/2$ for all $n,m\geq N_2$. Thus for $n\geq N:=\max\{N_1,N_2\}$ holds
$n_n\geq n\geq N$ and $$|a_n-a|\leq|a_n-a_{n_n}+a_{n_n}-a|\leq|a_n-a_{n_n}|+|a_{n_n}-a|< \varepsilon/2+\varepsilon/2=\varepsilon \ . \qquad\Box$$

Theorem \ref{thm:Rcompl} is not true for arbitrary normed $\mathbb{K}$-vector spaces. 
Those normed $\mathbb{K}$-vector spaces $(V,||\cdot||)$ for which every Cauchy sequence has a limit in $V$ are called complete
or Banach spaces (in honour of the Polish mathematician Stefan Banach). 
Without proof we state that all finite dimensional normed $\mathbb{K}$-vector spaces are Banach spaces.

The next result concerns the special property of the real numbers that supremum and infimum are defined for all subsets of the real numbers. This theorem goes back to \textsc{Julius Wilhelm Richard Dedekind} (1831--1916). It follows from the completeness axiom (C):
\begin{Theorem}[Dedekind's Theorem]
\label{thm:bndmonseq}
    Every non-empty bounded set $M\subset \mathbb{R}$ has a~supremum and an~infimum with $\sup M,\inf M\in\mathbb{R}$.
\end{Theorem}

