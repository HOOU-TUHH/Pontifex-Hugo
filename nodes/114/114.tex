\usepackage{amsthm}

\newtheorem{theorem}{Theorem}[chapter]
\newtheorem{lemma}           [theorem] {Lemma}   
\newtheorem{folg}           [theorem] {Folgerung}   

\newtheorem{frage}       [theorem] {Frage}   
\newtheorem{question}       [theorem] {Question}   
\newtheorem{aufgabe}       [theorem] {Aufgabe}   
\newtheorem{exercise}       [theorem] {Exercise}  

\newtheorem{proposition}     [theorem] {Proposition}  
\newtheorem{satz}     [theorem] {Satz}  
\newtheorem{fact}{Fact}
\newtheorem{definition}      [theorem] {Definition} 

\theoremstyle{definition} 
\newtheorem{bemerkung}     [theorem] {Bemerkung}  
\newtheorem{beispiel}       [theorem] {Beispiel}  
\newtheorem{example}       [theorem] {Example}  
\newtheorem*{example*} {Example}  
\newtheorem{notation}       [theorem] {Notation}  
\newtheorem*{Faust}[theorem]{Rule of Thumb}
\newtheorem*{Boxx}[theorem]{Concept}

\begin{Definition}[Monotonicity]
  A~\textbf{real} sequence $(a_n)_{n\in\mathbb{N}}$ is called
\begin{enumerate}[(a)]
  \item \emph{monotonically increasing} if for all $n\in\mathbb{N}$ holds $a_n\leq a_{n+1}$.
  \item \emph{strictly monotonically increasing} if for all $n\in\mathbb{N}$ holds $a_n < a_{n+1}$.
  \item \emph{monotonically decreasing} if for all $n\in\mathbb{N}$ holds $a_n\geq a_{n+1}$.
  \item \emph{strictly monotonically decreasing} if for all $n\in\mathbb{N}$ holds $a_n > a_{n+1}$.
\end{enumerate}
\end{Definition}

We make essential use of Dedekind's theorem to prove the following result:
\begin{Theorem}[Convergence of bounded and monotonic sequences]
 \label{thm:monbndseq}
 Let $(a_n)_{n\in\mathbb{N}}$ be a~real sequence that has one of the following properties:
\begin{itemize}
 \item[--] $(a_n)_{n\in\mathbb{N}}$ is monotonically increasing and bounded from above;
 \item[--] $(a_n)_{n\in\mathbb{N}}$ is monotonically decreasing and bounded from below;
\end{itemize}
Then $(a_n)_{n\in\mathbb{N}}$ is convergent.
\end{Theorem}
{\em Proof:} Let us first assume that $(a_n)_{n\in\mathbb{N}}$ is monotonically increasing and bounded from above. Define the set $M=\{a_n\,:\,n\in\mathbb{N}\}$. 
Since $M$ is bounded, Dedekind's theorem implies that there exists some $K\in\mathbb{R}$ such that \[K=\sup M.\]
%
%
We show that $K$ is indeed the limit of the sequence $(a_n)_{n\in\mathbb{N}}$.\\
Let $\varepsilon>0$. By the definition of the supremum, we have that $a_n\leq K$ for all $n\in\mathbb{N}$ and there exists some $N\in\mathbb{N}$ such that $a_N>K-\varepsilon$. The monotonicity of $(a_n)_{n\in\mathbb{N}}$ implies that for all $n\geq N$ holds $a_N\leq a_n$. Altogether, we have \[K-\varepsilon<a_N\leq a_n\leq K\]
and thus $|K-a_n|=K-a_n<\varepsilon$ for all $n\geq N$. This implies convergence to $K$.\\
To prove that convergence is also guaranteed in the case where $(a_n)_{n\in\mathbb{N}}$ is monotonically decreasing and bounded from below, we consider the sequence $(-a_n)_{n\in\mathbb{N}}$, which is now bounded from above and monotonically increasing. By the (already proven) first statement of this theorem, the sequence $(-a_n)_{n\in\mathbb{N}}$ is convergent, whence $(a_n)_{n\in\mathbb{N}}$ is convergent as well.\hfill$\Box$

\begin{Remark}{}\label{rem:monseqgen}
    By the same argumentation as in Remark from page \pageref{rem:n_0mon}, the monotone increase (decrease) of $(a_n)_{n\in\mathbb{N}}$ can be slightly relaxed by only claiming that $a_n\leq a_{n+1}$ ($a_n\geq a_{n+1}$) for all $n\geq n_0$ for some $n_0$ in $\mathbb{N}$. In such a~case, the limit of the sequence is then given by $\sup\{a_n\,:\,n\geq n_0\}$ (resp.\ $\inf\{a_n\,:\,n\geq n_0\}$).
\end{Remark}

\begin{example}{}\label{ex:monseqconv}
 \begin{enumerate}[a)]
  \item Consider the sequence $(a_n)_{n\in\mathbb{N}}$ which is recursively defined via $a_1=1$ and 
  \[a_{n+1}=\frac{a_n+\frac2{a_n}}{2}\;\text{  for $n\geq 1$.}\] 
  We now prove that this sequence is convergent by showing that
it is bounded from below and for all $n\geq2$ holds $a_{n+1}\leq a_n$.\\[2ex]
{\em Proof:} To show boundedness from below, we use the inequality $\sqrt{xy}\leq\frac{x+y}2$ for all nonnegative $x,y\in\mathbb{R}$. This inequality is a~consequence of
\[0\leq\frac{(\sqrt{x}-\sqrt{y})^2}2=\frac{x+y}2-\sqrt{xy}.\]
The first inequality is a consequence of the fact that squares of real numbers cannot be negative.\\
Using this inequality, we obtain for $n\geq1$
\[a_{n+1}=\frac{a_n+\frac{2}{a_n}}{2}\geq \sqrt{a_n\cdot\frac2{a_n}}=\sqrt{2}.\]
Thus, $(a_n)$ is bounded from below. For showing monotonicity, we consider
\[a_{n+1}-a_n=\frac{a_n+\frac2{a_n}}{2}-a_n=\frac{1}{2a_n}(2-a_n^2).\]
In particular, if $n\geq2$, we have that $a_n>0$ and $2-a_n^2\leq0$. Thus, $a_{n+1}-a_n\leq0$ for $n\geq2$. An~application of Theorem~\ref{thm:monbndseq} (resp.\ the slight generalisation in Remark %\ref{rem:monseqgen}
from above) 
now leads to the existence of some $a\in\mathbb{R}$ with $a=\lim_{n	\to\infty}a_n$.\\[2ex]
To compute the limit, we make use of the relation $\lim_{n \to\infty}a_n=\lim_{n\to\infty}a_{n+1}$ (follows directly from the Definition of limits) and the formulae for limits. This yields
\[a=\lim_{n \to\infty}a_n=\lim_{n\to\infty}a_{n+1}=\lim_{n\to\infty}\frac{a_n+\frac2{a_n}}{2}=\frac{a+\frac2{a}}{2}.\]
This relation leads to the equation $2-a^2=0$, i.e., we either have $a=\sqrt{2}$ or $a=-\sqrt{2}$. However, the latter solution cannot be a limit since all sequence elements are positive. Therefore, we have
\[\lim_{n\to\infty}a_n=\sqrt{2}.\]
\item Let $x\in\mathbb{R}$ with $x>1$. Consider the sequence $(\sqrt[n]{x})_{n\in\mathbb{N}}$. It can be directly seen that $(\sqrt[n]{x})_{n\in\mathbb{N}}$ is monotonically decreasing and bounded from below by one. Therefore, the limit
\[a=\lim_{n\to\infty}\sqrt[n]{x}\]
exists with $a\geq1$. To show that $a=1$, we assume that $a>1$ and lead this to a~contradiction.\\
         The equation $a>1$ leads to the existence of some $n\in\mathbb{N}$ with $a^n>x$, and thus $a>\sqrt[n]{x}$. On the other hand, the monotone decrease of $(\sqrt[n]{x})_{n\in\mathbb{N}}$ implies that
\[a=\lim_{n	o\infty}\sqrt[n]{x}=\inf\{\sqrt[n]{x}\,:\,n\in\mathbb{N}\}<a,\]
which is a~contradiction.
\item Let $x\in\mathbb{R}$ with $0< x<1$. Consider the sequence $(a_n)_{n\in\mathbb{N}}=(\sqrt[n]{x})_{n\in\mathbb{N}}$. Then we have by Example b) and the rules for calculating limits that
\[\lim_{n\to\infty}\sqrt[n]{x}=\frac1{\lim_{n\to\infty}\sqrt[n]{\frac1x}}=\frac11=1.\]
\item Let $(a_{n})$ be a nonnegative sequence with $a_{n}\rightarrow a$ and $k\in\mathbb{N}$. Then for all $\varepsilon>0$ there exists $N>0$ such that $|a_{n}-a|<\varepsilon^{k}$. From this it follows that
$$
|\sqrt[k]{a_{n}}-\sqrt[k]{a}|\leq \sqrt[k]{|a_{n}-a|}<\varepsilon.
$$
Thus $(\sqrt[k]{a_{n}})$ is convergent with limit $\sqrt[k]{a}$.\\
\item The sequence $(a_{n})_{n\in\mathbb{N}}$ defined as $a_{n} := \left(1+\frac{1}{n}\right)^{n}$ is convergent.\\
\end{enumerate}
\end{example}

\begin{Remark}{}
The limit of the sequence \[(a_n)_{n \in \mathbb{N}N} = \left(  \Big(1+\frac{1}{n} \,\Big)^{n}\right)_{n \in \mathbb{N}},\] 
i.e. $e:=\lim_{n\rightarrow\infty}\left(1+\frac{1}{n}\right)^{n}$ is well known as \textit{Euler's number}. Later on we will define the exponential function $\exp$. It holds that $e=\exp(1)\approx 2.7182818...$ . Indeed, we will show later on that $e^{z}=\lim_{n\rightarrow\infty}\left(1+\frac{z}{n}\right)^{n}=\exp(z)$.
\end{Remark}

