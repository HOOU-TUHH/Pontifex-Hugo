\usepackage{amsthm}

\newtheorem{theorem}{Theorem}[chapter]
\newtheorem{lemma}           [theorem] {Lemma}   
\newtheorem{folg}           [theorem] {Folgerung}   

\newtheorem{frage}       [theorem] {Frage}   
\newtheorem{question}       [theorem] {Question}   
\newtheorem{aufgabe}       [theorem] {Aufgabe}   
\newtheorem{exercise}       [theorem] {Exercise}  

\newtheorem{proposition}     [theorem] {Proposition}  
\newtheorem{satz}     [theorem] {Satz}  
\newtheorem{fact}{Fact}
\newtheorem{definition}      [theorem] {Definition} 

\theoremstyle{definition} 
\newtheorem{bemerkung}     [theorem] {Bemerkung}  
\newtheorem{beispiel}       [theorem] {Beispiel}  
\newtheorem{example}       [theorem] {Example}  
\newtheorem*{example*} {Example}  
\newtheorem{notation}       [theorem] {Notation}  
\newtheorem*{Faust}[theorem]{Rule of Thumb}
\newtheorem*{Boxx}[theorem]{Concept}
Next, we define some particular sets and special properties of sets.
\begin{Definition}[$\varepsilon$-neighbourhood]
 Let $x\in \mathbb{K}$. Then for $\varepsilon>0$, the \emph{$\varepsilon$-neighbourhood of $x$} is defined by the set
\[B_{\varepsilon}(x)=\{y\in \mathbb{K}\,:\,|x-y|<\varepsilon\}.\]
A~set $M\subset \mathbb{K}$ is called \emph{neighbourhood of $x$}, if there exists some $\varepsilon>0$ such that
\[B_{\varepsilon}(x)\subset M.\]
\end{Definition}
\begin{example}
 \begin{enumerate}[(a)]
  \item If $\mathbb{K}=\mathbb{R}$, then the $\varepsilon$-neighbourhood of $x\in\mathbb{R}$ is given by the interval
\[B_{\varepsilon}(x)=(x-\varepsilon,x+\varepsilon).\]
\item If $\mathbb{K}=\mathbb{C}$, $\varepsilon>0$, then the $\varepsilon$-neighbourhood of $x\in\mathbb{C}$ consists of all complex numbers being in the interior of a~circle in the complex plane with midpoint $x$ and radius $\varepsilon$.
%\item $M=\mathbb{R}^3$, $\varepsilon>0$, then the $\eps$-neighborhood of $x\in\mathbb{R}^3$  consists of all real triples in the tree-dimensional Cartesian coordinate %system which are in the interior of a~ball with midpoint $x$ and radius $\eps$.
\item $[0,1]$ is a~neighbourhood of $\frac12$ (also of $\frac34$, $\frac1{\sqrt{2}}$ etc.), but it is not a neighbourhood of $0$ or $1$.
 \end{enumerate}
\end{example}

\begin{Definition}[Bounded{,} Open{,} closed{,} compact sets]\label{def:clopco}
 Let $M\subset \mathbb{K}$. Then $M$ is called
\begin{enumerate}[(i)]
 \item \emph{bounded} if there exists some $c \in \mathbb{R}$ such that for all $x\in M$ holds: $|x|\leq c$. 
 \item \emph{open} if for all $x\in M$ holds: $M$ is a~neighbourhood of $x$.
 \item \emph{closed} if for all convergent sequences $(a_n)_{n\in\mathbb{N}}$ with $a_n\in M$ for all $n\in\mathbb{N}$ holds: $\lim_{n\to\infty}a_n=a\in M$.
 \item \emph{compact}  if for all sequences $(a_n)_{n\in\mathbb{N}}$ with $a_n\in M$ for all $n\in\mathbb{N}$ holds: There exists some convergent subsequence $(a_{n_k})_{k\in\mathbb{N}}$ with $\lim_{k\to\infty}a_{n_k}=a\in M$.
\end{enumerate}
\end{Definition}

\begin{example}
 \begin{enumerate}[(a)]
\item The interval $(0,1)$ is open. \\
{\em Proof:} Consider $x\in(0,1)$. Then for $\varepsilon=\min\{x,1-x\}$ holds $\eps>0$ and \linebreak$B_\varepsilon(x)=(x-\varepsilon,x+\varepsilon)\;\subset\;(0,1)$.\hfill$\Box$
\item The interval $(0,1)$ is not closed.\\
{\em Proof:} Consider the sequence $(a_n)_{n\in\mathbb{N}}=(\frac1{n+1})_{n\in\mathbb{N}}$. Clearly, for all $n\in\mathbb{N}$ holds $a_n=\frac1{n+1}\in(0,1)$, but $(a_n)_{n\in\mathbb{N}}$ converges to $0\notin(0,1)$.
\hfill$\Box$
\item The interval $(0,1)$ is not compact.\\
{\em Proof:} Again consider the sequence $(a_n)_{n\in\mathbb{N}}=(\frac1{n+1})_{n\in\mathbb{N}}$ in $(0,1)$. The convergence of $(a_n)_{n\in\mathbb{N}}$ to $0\notin(0,1)$ also implies that this holds true for any subsequence $(a_{n_k})_{k\in\mathbb{N}}$ (see Theorem~\ref{thm:convsubseq}). Hence, any subsequence of the above constructed one is not convergent to some value in $(0,1)$.
\hfill$\Box$
\item The interval $(0,1]$ is neither open nor closed.\\
{\em Proof:} The closedness can be disproved by considering again the sequence \linebreak$(a_n)_{n\in\mathbb{N}}=(\frac1{n+1})_{n\in\mathbb{N}}$, whereas the non-openness follows from the fact that $(0,1]$ is not a~neighbourhood of $1$.
\hfill$\Box$
\item The set $\mathbb{R}$ is open and closed but not compact.\\
{\em Proof:} Openness and closedness are easy to verify. To see that this set is not compact, consider the sequence $(a_n)_{n\in\mathbb{N}}=(n)_{n\in\mathbb{N}}$ (which is of course in $\mathbb{R}$). It can be readily verified that any subsequence $(a_{n_k})_{k\in\mathbb{N}}=(n_k)_{k\in\mathbb{N}}$ is unbounded, too. Therefore, arbitrary subsequences $(a_{n_k})_{k\in\mathbb{N}}=(n_k)_{k\in\mathbb{N}}$ cannot converge.\hfill$\Box$
\item The empty set $\emptyset$ is open, closed and compact.\\
{\em Proof:}
$\emptyset$ is a~neighbourhood of all $x\in\emptyset$ (there is none, but the statement ``for all $x\in\emptyset$'' holds then true more than ever). By the same kind of argumentation, we can show that this set is compact and closed. The non-existence of a~sequence in $\emptyset$ implies that every statement holds true for them. In particular, all sequences $(a_{n})_{n\in\mathbb{N}}$ in $\emptyset$ converge to some $x\in\emptyset$ and have a convergent subsequence with limit in $\emptyset$.\hfill$\Box$
\end{enumerate}
\end{example}

Next we relate these three concepts to each other.
\begin{Theorem}{}
  For a~set $C\subset \mathbb{K}$, the following statements are equivalent:
\begin{enumerate}[(i)]
 \item $C$ is open;
 \item $\mathbb{K}\backslash C$ is closed.
\end{enumerate}
\end{Theorem}
{\em Proof:}\\
``(i)$\Rightarrow$(ii)'':
Let $C$ be open. Consider a convergent sequence $(a_{n})_{n\in\mathbb{N}}$ with $a_n\in \mathbb{K}\backslash C$. We have to show that for $a=\lim_{n\to\infty}a_n$ holds $a\in \mathbb{K}\backslash C$. Assume the converse, i.e., $a\in C$. Since $C$ is open, we have that $B_{\varepsilon}(a)\subset C$ for some $\varepsilon>0$. By the definition of convergence, there exists some $N$ such that for all $n\geq N$ holds $|a-a_n|<\varepsilon$, i.e., \[a_n\in B_{\varepsilon}(a)\subset C.\] However, this is a~contradiction to $a_n\in \mathbb{K}\backslash C$.\\

``(ii)$\Rightarrow$(i)'':
Let $\mathbb{K}\backslash C$ be closed. We have to show that $C$ is open. Assume the converse, i.e., $C$ is not open. In particular, this means that there exists some $a\in C$ such that for all $n\in\mathbb{N}$ holds $B_{\frac1n}(a)\not\subset C$. This means that for all $n\in \mathbb{N}$, we can find some $a_n\in \mathbb{K}\backslash C$ with $a_n\in B_{\frac1n}(a)$, i.e., $|a-a_n|<\frac1n$. As a~consequence, for the sequence $(a_n)_{n\in\mathbb{N}}$ holds that
\[\lim_{n\to\infty}a_n=a\in C,\]
but $a_n\in \mathbb{K}\backslash C$ for all $n\in\mathbb{N}$. This is a~contradiction to the closedness of $\mathbb{K}\backslash C$.$\Box$

