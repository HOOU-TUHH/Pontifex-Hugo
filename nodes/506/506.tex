\usepackage{amsthm}

\newtheorem{theorem}{Theorem}[chapter]
\newtheorem{lemma}           [theorem] {Lemma}   
\newtheorem{folg}           [theorem] {Folgerung}   

\newtheorem{frage}       [theorem] {Frage}   
\newtheorem{question}       [theorem] {Question}   
\newtheorem{aufgabe}       [theorem] {Aufgabe}   
\newtheorem{exercise}       [theorem] {Exercise}  

\newtheorem{proposition}     [theorem] {Proposition}  
\newtheorem{satz}     [theorem] {Satz}  
\newtheorem{fact}{Fact}
\newtheorem{definition}      [theorem] {Definition} 

\theoremstyle{definition} 
\newtheorem{bemerkung}     [theorem] {Bemerkung}  
\newtheorem{beispiel}       [theorem] {Beispiel}  
\newtheorem{example}       [theorem] {Example}  
\newtheorem*{example*} {Example}  
\newtheorem{notation}       [theorem] {Notation}  
\newtheorem*{Faust}[theorem]{Rule of Thumb}
\newtheorem*{Boxx}[theorem]{Concept}

\begin{Theorem}{}
    Let $I$ be an interval and $f:I \to\mathbb{R}$ be a~function that is differentiable in $x_0\in I$. Assume that $x_0$ is an interior point of $I$ and that $x_0$ is a local extremum. Then $f'(x_0)=0$.
\end{Theorem}

{\em Proof:} We assume that $x_0$ is a local maximum (the case of minimum is shown analogously). Let $U$ be a~neighbourhood of $x_0$ with $U\subset I$ and $f(x_0)=\max\{f(x):\;x\in U\}$. Let
\[f(x)=f(x_0)+(x-x_0)\cdot\Delta_{f,x_0}(x).\]
Assume that $f'(x_0)=\Delta_{f,x_0}(x_0)>0$. Since $\Delta_{f,x_0}$ is continuous in $x_0$, then there exists a~neighbourhood $V\subset U$ of $x_0$ such that $\Delta_{f,x_0}(x)>0$ for all $x\in V$. Then for all $x_1\in V$ with $x_1>x_0$ holds $f(x_1)=f(x_0)+(x_1-x_0)\cdot\Delta_{f,x_0}(x_1)>f(x_0)$. This is a~contradiction.\\
On the other hand, assume that $f'(x_0)=\Delta_{f,x_0}(x_0)<0$. Since $\Delta_{f,x_0}$ is continuous in $x_0$, then there exists a~neighbourhood $V\subset U$ such that $\Delta_{f,x_0}(x)<0$ for all $x\in V$. Then for all $x_1\in V$ with $x_1<x_0$ holds $f(x_1)=f(x_0)+(x_1-x_0)\cdot\Delta_{f,x_0}(x_1)>f(x_0)$. This is also a~contradiction.$\Box$

As a~consequence, we will formulate the following result stating that derivatives of functions with equal boundary conditions have at least one zero.
\begin{Theorem}[Theorem of Rolle]
Let $f:[a,b]\to\mathbb{R}$ be differentiable in $[a,b]$ and let $f(a)=f(b)$. Then there exists some $x\in(a,b)$ such that $f'(x)=0$.
\end{Theorem}
{\em Proof:} If $f$ is constant, the statement is clear (since then $f'(x)=0$ for all $x\in(a,b)$). If $f$ is not constant, consider the maximum and the minimum of $f$ on $[a,b]$ (we know by the theorem about compactness of images under continuout functions the that they exist). So, let $x_-,x_+\in[a,b]$ such that
\[f(x_+)=\max\{f(x)\;:\;x\in[a,b]\},\qquad f(x_-)=\min\{f(x)\;:\;x\in[a,b]\}.\]
Then we have that $x_+\in(a,b)$ or $x_-\in(a,b)$ since, otherwise, $f(x_+)=f(x_-)$ (constant).\\
Then $f'(x_-)=0$ or $f'(x_+)=0$.\hfill$\Box$

As a~corollary, we have that for a function $f$ differentiable in some interval $I$, the following holds: Between two zeros of $f$, there always exists some point $x_0$ with $f'(x_0)=0$.

