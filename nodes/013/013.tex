\usepackage{amsthm}

\newtheorem{theorem}{Theorem}[chapter]
\newtheorem{lemma}           [theorem] {Lemma}   
\newtheorem{folg}           [theorem] {Folgerung}   

\newtheorem{frage}       [theorem] {Frage}   
\newtheorem{question}       [theorem] {Question}   
\newtheorem{aufgabe}       [theorem] {Aufgabe}   
\newtheorem{exercise}       [theorem] {Exercise}  

\newtheorem{proposition}     [theorem] {Proposition}  
\newtheorem{satz}     [theorem] {Satz}  
\newtheorem{fact}{Fact}
\newtheorem{definition}      [theorem] {Definition} 

\theoremstyle{definition} 
\newtheorem{bemerkung}     [theorem] {Bemerkung}  
\newtheorem{beispiel}       [theorem] {Beispiel}  
\newtheorem{example}       [theorem] {Example}  
\newtheorem*{example*} {Example}  
\newtheorem{notation}       [theorem] {Notation}  
\newtheorem*{Faust}[theorem]{Rule of Thumb}
\newtheorem*{Boxx}[theorem]{Concept}

\begin{Definition}[Upper and lower bounds for sets]
Let $M \subset \mathbb{R}$ be any subset of real numbers. A real number $b$ is called an \emph{upper bound of $M$}
if $x \leq b$ for all $x \in M$. Analogously, $a \in \mathbb{R}$ is called a \emph{lower bound of $M$}
if $a \leq x$ for all $x \in M$.
\end{Definition}

\begin{Definition}[Bounded from below or above for sets]
Let $M \subset \mathbb{R}$. If there is an upper bound for $M$, then
one calls the set \emph{bounded from above}. 
If there is a lower bound for $M$, then
one calls the set \emph{bounded from below}. If both properties hold, we call the set simply \emph{bounded}.
\end{Definition}

\begin{Definition}[Maximal and minimal element of a set]
Let $M \subset \mathbb{R}$. An element $d \in M$ is called \emph{maximal} if
$ x\leq d$ for all $x \in M$. An element $c \in M$ is called \emph{minimal} if
$ x\geq c$ for all $x \in M$. If these numbers exist, one write $\max M = d$
and $\min M = c$.
\end{Definition}

