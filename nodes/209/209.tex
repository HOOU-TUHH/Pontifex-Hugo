\usepackage{amsthm}

\newtheorem{theorem}{Theorem}[chapter]
\newtheorem{lemma}           [theorem] {Lemma}   
\newtheorem{folg}           [theorem] {Folgerung}   

\newtheorem{frage}       [theorem] {Frage}   
\newtheorem{question}       [theorem] {Question}   
\newtheorem{aufgabe}       [theorem] {Aufgabe}   
\newtheorem{exercise}       [theorem] {Exercise}  

\newtheorem{proposition}     [theorem] {Proposition}  
\newtheorem{satz}     [theorem] {Satz}  
\newtheorem{fact}{Fact}
\newtheorem{definition}      [theorem] {Definition} 

\theoremstyle{definition} 
\newtheorem{bemerkung}     [theorem] {Bemerkung}  
\newtheorem{beispiel}       [theorem] {Beispiel}  
\newtheorem{example}       [theorem] {Example}  
\newtheorem*{example*} {Example}  
\newtheorem{notation}       [theorem] {Notation}  
\newtheorem*{Faust}[theorem]{Rule of Thumb}
\newtheorem*{Boxx}[theorem]{Concept}

\begin{Theorem}[Root criterion]\label{thm:rootkrit}
Let $n_0\in\mathbb{N}$ and let $\sum_{k=1}^\infty a_k$ be a~series in $\mathbb{K}$ and assume that there exists some $q\in(0,1)$ such that for all $k\geq n_0$ holds
\[\sqrt[k]{|a_{k}|} \leq q.\]
Then $\sum_{k=1}^\infty a_k$ converges absolutely.
\end{Theorem}

{\em Proof:} Taking the $k$-th power of the inequality $\sqrt[k]{|a_{k}|}<q$, we obtain that for all $k\geq n_0$ holds
\[|a_k|< q^{k}\]
Therefore, the convergent geometric series $\sum_{k=1}^\infty q^{k}$ is a majorant of $\sum_{k=1}^\infty a_k$ and thus, 
we have absolute convergence.\hfill$\Box$

\begin{Theorem}[Root criterion (limit form)]\label{thm:rootkritlim}
Let $\sum_{k=1}^\infty a_k$ be a~series in $\mathbb{K}$ and assume that
\[\limsup_{k \rightarrow \infty} \sqrt[k]{|a_{k}|}<1.\]
Then $\sum_{k=1}^\infty a_k$ converges absolutely.
\end{Theorem}
\white{4cm}{
{\em Proof:} The argumentation is analogous as in the proof of Theorem~\ref{thm:quotkritlim}. Let $c:=\lim\sup\sqrt[k]{|a_{k}|}<1$. 
Then for $\varepsilon:=\frac{1-c}2,$ there exists some $N\in\mathbb{N}$ such that for all $k\geq N$ holds
\[\sqrt[k]{|a_{k}|}<c+\varepsilon=c+\frac{1-c}2=\frac{1+c}2<1.\]
The root criterion with $q:=\frac{1+c}{2}<1$  now implies convergence. \hfill$\Box$}

\begin{example}
 Consider the series
\[\sum_{k=1}^\infty \frac{k^5}{3^k}.\]
Then we have
\[\limsup_{k \rightarrow \infty} \sqrt[k]{\left|\frac{k^5}{3^k}\right|}
=\limsup_{k \rightarrow \infty} \frac{\sqrt[k]{k}^5}{3}
\]
Since we know that $\sqrt[k]{k}$ converges to $1$, the whole expression converges to $\frac13<1$. Hence, the series converges.
\end{example}

%  We will now state two convergence criteria for series of the form $\sum_{k=1}^\infty a_kb_k$.
%  They are easily deduced from the following lemma.
%  
%  \begin{Lemma}[Abel's partial sums] \label{lem:abelsum}
%    For $n\in\mathbb{N}$ and $a_1,...,a_n,b_1,...,b_{n+1}\in\mathbb{K}$ holds 
%    $$ \sum_{k=1}^na_kb_k = A_nb_{n+1}+\sum_{k=1}^nA_k(b_k-b_{k+1})~ ,$$
%    where $A_k:=\sum_{i=1}^ka_i$ for $k\in\{1,...,n\}$. 
%  \end{Lemma}
%  \textit{Proof:} If we additionally define $A_0:=0$, then 
%  \begin{eqnarray*}
%    \sum_{k=1}^na_kb_k &=& \sum_{k=1}^n(A_k-A_{k-1})b_k =\sum_{k=1}^nA_kb_k-\sum_{k=1}^nA_{k-1}b_k 
%                       = \sum_{k=1}^nA_kb_k-\sum_{k=1}^{n-1}A_kb_{k+1}    \\
%                       &=& \sum_{k=1}^nA_kb_k-\sum_{k=1}^{n}A_kb_{k+1}+A_nb_{n+1} 
%                       = \sum_{k=1}^nA_k(b_k-b_{k+1})+A_nb_{n+1}  \ .
%  \end{eqnarray*}
%  
%  \begin{Theorem}[Abel criterion]\label{th:abelcrit}
%    If the series $\sum_{k=1}^\infty a_k$ in $\mathbb{K}$ converges and if the real sequence $(b_{k})_{k\in\mathbb{N}}$ is monotonic and bounded, then 
%    the series $\sum_{k=1}^\infty a_kb_k$ converges.
%  \end{Theorem}
%  {\em Proof:}
%    Set $A_k:=\sum_{i=1}^k a_i$. By assumption both sequences $(A_k)_{k\in\mathbb{N}}$ and $(b_k)_{k\in\mathbb{N}}$ converge so that 
%    $(A_kb_{k+1})_{k\in\mathbb{N}}$ converges also. Since $(b_{k})_{k\in\mathbb{N}}$ is monotonic,
%    the telescoping series $\sum_{k=1}^\infty (b_k-b_{k+1})$ converges absolutely as
%    $$\sum_{k=1}^n |b_k-b_{k+1}|= |\sum_{k=1}^n (b_k-b_{k+1})| 
%    = |b_1-b_{n+1}| 
%    \xrightarrow{n \rightarrow \infty} |b_1-\lim_{k\rightarrow\infty}b_k| \ .$$
%    Since $(A_k)_{k\in\mathbb{N}}$ is bounded, $\sum_{k=1}^\infty A_k(b_k-b_{k+1})$ is also absolutely convergent. Summing up,
%    Lemma~\ref{lem:abelsum} implies that the series $\sum_{k=1}^\infty a_kb_k$ converges to the limit
%    $$\sum_{k=1}^\infty a_kb_k =  \lim_{n\rightarrow\infty}A_n b_{n+1} + \sum_{k=1}^\infty A_k(b_k-b_{k+1})\ . $$
%  \hfill$\Box$
%  
%  \begin{Theorem}[Dirichlet criterion]\label{th:dirichletcrit}
%    If the series $\sum_{k=1}^\infty a_k$ in $\mathbb{K}$ is bounded and if the real sequence $(b_{k})_{k\in\mathbb{N}}$ converges monotonically to zero, then 
%    the series $\sum_{k=1}^\infty a_kb_k$ converges.
%  \end{Theorem}
%  {\em Proof:}
%    Set $A_k:=\sum_{i=1}^k a_i$. By assumption $(A_k)_{k\in\mathbb{N}}$ is bounded. Hence $(A_kb_{k+1})_{k\in\mathbb{N}}$ converges to zero.
%    By the same argument as in the proof of Theorem \ref{th:abelcrit}, the series $\sum_{k=1}^\infty A_k(b_k-b_{k+1})$ is absolutely convergent. 
%    Lemma~\ref{lem:abelsum} again implies that the series $\sum_{k=1}^\infty a_kb_k$ converges to the limit
%    $$\sum_{k=1}^\infty a_kb_k =  \underbrace{\lim_{n\rightarrow\infty}A_n b_{n+1}}_{=0} + \sum_{k=1}^\infty A_k(b_k-b_{k+1})\ . $$
%  \hfill$\Box$
%  \\ \\
%  Note that the Leibniz criterion follows from the Dirichlet criterion by taking $a_k:=(-1)^k$.

