\usepackage{amsthm}

\newtheorem{theorem}{Theorem}[chapter]
\newtheorem{lemma}           [theorem] {Lemma}   
\newtheorem{folg}           [theorem] {Folgerung}   

\newtheorem{frage}       [theorem] {Frage}   
\newtheorem{question}       [theorem] {Question}   
\newtheorem{aufgabe}       [theorem] {Aufgabe}   
\newtheorem{exercise}       [theorem] {Exercise}  

\newtheorem{proposition}     [theorem] {Proposition}  
\newtheorem{satz}     [theorem] {Satz}  
\newtheorem{fact}{Fact}
\newtheorem{definition}      [theorem] {Definition} 

\theoremstyle{definition} 
\newtheorem{bemerkung}     [theorem] {Bemerkung}  
\newtheorem{beispiel}       [theorem] {Beispiel}  
\newtheorem{example}       [theorem] {Example}  
\newtheorem*{example*} {Example}  
\newtheorem{notation}       [theorem] {Notation}  
\newtheorem*{Faust}[theorem]{Rule of Thumb}
\newtheorem*{Boxx}[theorem]{Concept}

%\section{The Cauchy Product of Series}
\begin{Definition}[Cauchy product of series]
Let $\sum_{k=0}^\infty a_k$, $\sum_{k=0}^\infty b_k$ be two series in $\mathbb{K}$. Then the {\em Cauchy product of $\sum_{k=0}^\infty a_k$, $\sum_{k=0}^\infty b_k$} is given by
\[\sum_{k=0}^\infty c_k\quad\text{ with }c_k=\sum_{l=0}^ka_lb_{k-l}.\]
\end{Definition}

\begin{Remark}{}
Note that we considered series with lower summation index 0. The Cauchy product can be also defined for sequences $\sum_{k=n_0}^\infty a_k$, $\sum_{k=n_1}^\infty b_k$ with arbitrary $n_0,n_1\in\mathbb{N}$ (or even $n_0,n_1\in\mathbb{Z}$). In this case, the Cauchy product is given by
\[\sum_{k=n_0+n_1}^\infty c_k\quad\text{ with }c_k=\sum_{l=n_0}^{k-n_1}a_lb_{k-l}.\]
In order to ``keep the set of indices manageable'', this is not further treated here.
Note that the following result about convergence properties of the Cauchy product still hold true in this above mentioned more general case.
\end{Remark}
\whiteskip

The following theorem justifies the name ``product'' in the above definition.
\begin{Theorem}[Convergence of the Cauchy product]\label{thm:cprod}
Let $\sum_{k=0}^\infty a_k$, $\sum_{k=0}^\infty b_k$ be series in $\mathbb{K}$. Assume that $\sum_{k=0}^\infty a_k$ is absolutely convergent and $\sum_{k=0}^\infty b_k$ is convergent. Then the Cauchy product $\sum_{k=0}^\infty c_k$ is absolutely convergent. Moreover, the limit satisfies
\[\sum_{k=0}^\infty c_k=\left(\sum_{k=0}^\infty a_k\right)\cdot\left(\sum_{k=0}^\infty b_k\right).\]
\end{Theorem}
{\em Proof:}
Denote the sequence of partial sums of $\sum_{k=0}^\infty a_k$, $\sum_{k=0}^\infty b_k$ and $\sum_{k=0}^\infty c_k$ by $(A_n)_{n\in\mathbb{N}}$, $(B_n)_{n\in\mathbb{N}}$ and $(C_n)_{n\in\mathbb{N}}$, 
respectively. Moreover, set $a:=\sum_{k=0}^\infty a_k$ and $b=\sum_{k=0}^\infty b_k$.\\
Then we have 
%(by using the ugly summation formula of Theorem~\ref{thm:sumform} in the appendix)
\[C_n=\sum_{k=0}^nc_k=\sum_{k=0}^n\sum_{l=0}^ka_lb_{k-l}=\sum_{k=0}^na_{n-k}\sum_{l=0}^{k}b_l=\sum_{k=0}^na_{n-k}B_k.\]
Using this expression, we obtain
\[C_n=\sum_{k=0}^na_{n-k}(B_k-b)+\sum_{k=0}^na_{n-k}b=\sum_{k=0}^na_{n-k}(B_k-b)+A_nb.\]
Let $\varepsilon>0$. Since $\sum_{k=0}^\infty a_k$ converges absolutely, there exists some $N_0$ such that for all $n\geq N_{0}$ holds
\[|B_n-b|<\frac{\varepsilon}{4(\sum_{k=0}^{\infty}|a_k|+1)}.\]
Since $(a_{n})_{n\in\mathbb{N}}$ converges to zero by the necessary property of sequences, there exists some $N_1$ such that for all $n\geq N_{1}$ holds
\[|a_{n}|<\frac{\varepsilon}{4 N_{0}\ (\sup\{|b-B_l|\,:\,l\in\mathbb{N}\}+1)}.\]
Also there exists some $N_2$ such that for all $n\geq N_2$ holds
\[|A_{n}-a|<\frac{\varepsilon}{2(|b|+1)}.\]
Therefore, with $N=\max\{N_0+N_1,N_2\}$, we have that for all $n\geq N$ holds
\[
\begin{aligned}
|C_n-ab|=&\;\left|\sum_{k=0}^na_{n-k}(B_k-b)+b(A_n-a)\right|
\leq \sum_{k=0}^n|a_{n-k}||B_k-b|+|b||A_n-a|\\
=&\;\sum_{k=0}^{N_{0}-1}\underbrace{|a_{n-k}|}_{\color{blue}<\frac{\varepsilon}{4N_{0}\ (\sup\{|b-B_l|\,:\,l\in\mathbb{N}\}+1)}}|B_k-b|+
\sum_{k=N_{0}}^{n}|a_{n-k}|\underbrace{|B_k-b|}_{\color{blue}<\frac{\varepsilon}{4(\sum_{k=0}^{\infty}|a_k|+1)}}+
|b|\underbrace{|A_n-a|}_{\color{blue}<\frac{\varepsilon}{2(|b|+1)}}\\
<&\;\sum_{k=0}^{N_{0}-1}\frac{\varepsilon|B_k-b|}{4N_{0}\ (\sup\{|b-B_l|\,|\,l\in\mathbb{N}\}+1)}+
 \sum_{k=N_{0}}^{n}\frac{\varepsilon|a_{n-k}|}{4(\sum_{k=0}^{\infty}|a_k|+1)}+
 \frac{\varepsilon|b|}{2(|b|+1)}\\
<&\;\frac{\varepsilon}{4}+\frac{\varepsilon}{4}+\frac{\varepsilon}{2}=\varepsilon
\end{aligned}\]

\begin{example}\label{ex:expsercauch}
Let $x,y\in\mathbb{K}$ and consider the series
\[\sum_{k=0}^\infty\frac{x^k}{k!},\qquad \sum_{k=0}^\infty\frac{y^k}{k!}\]
which are absolutely convergent as shown as an Example covered in the node for the quotient criterion.
Then the Cauchy product of both series is given by
$\sum_{k=0}^\infty c_k$ with
\[c_k=\sum_{l=0}^k\frac{x^l}{l!}\frac{y^{k-l}}{(k-l)!}=\frac{1}{k!}\cdot\sum_{l=0}^k\begin{pmatrix}k\\l\end{pmatrix}x^ly^{k-l}.\]
By the Binomial Theorem, we obtain
\[\sum_{l=0}^k\begin{pmatrix}k\\l\end{pmatrix}x^ly^{k-l}=(x+y)^k.\]
Hence, by the Theorem on the Cauchy product, we have
\[\left(\sum_{k=0}^\infty\frac{x^k}{k!}\right)\cdot\left(\sum_{k=0}^\infty\frac{y^k}{k!}\right)=\sum_{k=0}^\infty\frac{(x+y)^k}{k!}.\]
Altogether, this means that the function
\[f(x)=\sum_{k=0}^\infty\frac{x^k}{k!}\]
    fulfills $f(x+y)=f(x)\cdot f(y)$ for all $x,y\in\mathbb{R}$ (and even $x,y\in\mathbb{C}$). 
\white{2cm}{This property is for instance fulfilled by the exponential function. 
Indeed, we have that $f$ as defined above fulfills $f(x)=e^x$.
}
\end{example}

