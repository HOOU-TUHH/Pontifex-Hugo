\usepackage{amsthm}

\newtheorem{theorem}{Theorem}[chapter]
\newtheorem{lemma}           [theorem] {Lemma}   
\newtheorem{folg}           [theorem] {Folgerung}   

\newtheorem{frage}       [theorem] {Frage}   
\newtheorem{question}       [theorem] {Question}   
\newtheorem{aufgabe}       [theorem] {Aufgabe}   
\newtheorem{exercise}       [theorem] {Exercise}  

\newtheorem{proposition}     [theorem] {Proposition}  
\newtheorem{satz}     [theorem] {Satz}  
\newtheorem{fact}{Fact}
\newtheorem{definition}      [theorem] {Definition} 

\theoremstyle{definition} 
\newtheorem{bemerkung}     [theorem] {Bemerkung}  
\newtheorem{beispiel}       [theorem] {Beispiel}  
\newtheorem{example}       [theorem] {Example}  
\newtheorem*{example*} {Example}  
\newtheorem{notation}       [theorem] {Notation}  
\newtheorem*{Faust}[theorem]{Rule of Thumb}
\newtheorem*{Boxx}[theorem]{Concept}

%\section{Real Numbers}
Everybody has got to know at school, the rational numbers, the real, and basic arithmetics. There are certain rules that 
we can apply, and usually we do not think about them. 

In our course, we will get to know other objects than real numbers (vectors, matrices), where some of these laws do not 
apply any more. So try to have a fresh look at those well known laws:

We can add ($a+b$) and multiply ($ab$ or $a\cdot b$) real numbers and use parentheses $($,$)$ to 
describe the order of the computations. We have the notational \emph{convention}
that multiplication binds stronger than addition: ($ab+c$ means $(ab)+c$ and not $ab+c=a(b+c)$)

Some \emph{laws} apply:
\begin{align*}
 a+(b+c)&=(a+b)+c, \qquad a(bc)=(ab)c  & \mbox{ associative law }\\
 a+b &=b+a \qquad \qquad \;  ab = ba & \mbox{ commutative law }\\
 a(b+c) &= ab+ac  & \mbox{ distributive law }
\end{align*}
Furthermore, we are used to have the neutral numbers $0$ and $1$ with special properties:
\begin{align*}
 a+0 = a \qquad a \cdot 1 = a
\end{align*}
and additive inverse element $-a$ and also the multiplicative inverse $a^{-1}=1/a$
for $a \neq 0$. They fulfil $a+(-a)=0$ and $a a^{-1}=1$. 

A set with such properties is called a \emph{field}. Here we have the field of real numbers $\mathbb{R}$.

It is also well known that the real numbers can be ordered, i.e., the relation $a < b$ makes sense. It has
turned out, that the following rules are sufficient to derive all known rules concerning ordering of numbers. 

\begin{itemize}
 \item For any $a \in \mathbb{R}$ exactly one of the three relations hold
 \[
  a < 0, \mbox{ or } a > 0 \mbox{ or } a = 0   
 \]
 \item For all $a,b\in \mathbb{R}$ with $a>0$ and $b>0$ one has $a+b > 0$ and $a b > 0$.
\end{itemize}

Then, as a definition we write:
$$
 a < b \quad : \Leftrightarrow \quad a-b < 0
$$
and
$$
 a \leq b  \quad : \Leftrightarrow \quad a-b < 0  \text{ or } a = b \,.
$$

In particular, we have for $a \neq 0$ that always $a^2 > 0$, because $a^2=(-a)^2 > 0$ by the last rule applied to one of these terms.

The order relations are the reason, why we can think of the real numbers as a line, the ''real line``. 

For describing subsets of the real numbers, we will use \emph{intervals}. 
Let $a,b\in\mathbb{R}$. Then we define
\begin{align*}
\left[a,b\right] &:= \{x\in\mathbb{R} \mid a\le x\le b\}\\
(a,b] &:= \{x\in\mathbb{R} \mid a< x\le b\} \\
[a,b) &:= \{x\in\mathbb{R} \mid a\le x< b\}\\
(a,b) &:= \{x\in\mathbb{R} \mid a< x< b\}.
\end{align*}

Obviously, in the case $a>b$, all the sets above are empty.
We also can define unbounded intervals:
\begin{align*}
\left[a,\infty\right) := \{x\in\mathbb{R} \mid a\le x\}&,\qquad
(a,\infty) := \{x\in\mathbb{R} \mid a< x\} \\
\left(-\infty,b\right] := \{x\in\mathbb{R}\mid x\le b\}&,\qquad
(-\infty,b) := \{x\in\mathbb{R}\mid x< b\}.
\end{align*}

\begin{Definition}[Absolute value for real numbers]
The \emph{absolute value} of a number $x\in\mathbb{R}$ is defined by
\[
|x|:=\begin{cases}
~~x & \text{ if } x\ge 0,\\
-x & \text{ if } x< 0.
\end{cases}
\]
\end{Definition}
%
\begin{question}
Which of the following claims are true?
\[
|-3.14|=3.14, \quad |3|=3 , \quad |-\tfrac75|=\tfrac75, 
\quad {-|-\tfrac35|=\tfrac35},
\quad {|0| \text{ is not well-defined}}.
\]
\end{question}

\begin{Proposition}[Two important properties]
For any two real numbers $x,y\in\mathbb{R}$, one has
\begin{abc}
\item $|x\cdot y| = |x| \cdot |y|$, ($|\cdot|$ is multiplicative),
\item $|x+y| \le |x| + |y|$, ($|\cdot|$ fulfils the triangle inequality).
\end{abc}
\end{Proposition}

\begin{Proposition}[Real numbers]
The real numbers are a non-empty set $\mathbb{R}$
together with the operations $+ : \mathbb{R} \times \mathbb{R} \rightarrow \mathbb{R}$
and  $\cdot : \mathbb{R} \times \mathbb{R} \rightarrow \mathbb{R}$
and an ordering relation $<: \mathbb{R} \times \mathbb{R} \rightarrow \{\text{True}, \text{False}\}$
that fulfil the following rules
\begin{enumerate}
	\item[(A)] Addition
	\begin{enumerate}[(1)]
		\item[(A1)]  associative: $x + (y + z) = (x + y) + z$
		\item[(A2)]  neutral element: There is a (unique) element $0$
		with $x + 0 = x$ for all $x$.
		\item[(A3)]  inverse element: For all $x$ there is a (unique) $y$
		with $x + y = 0$. We write for this element simply $-x$.
		\item[(A4)]  commutative: $x + y = y + x$
	\end{enumerate}
 	\item[(M)]  Multiplication
	\begin{enumerate}[(1)]
		\item[(M1)] associative: $x \cdot (y \cdot z) = (x \cdot y) \cdot z$
		\item[(M2)] neutral element: There is a (unique) element $1 \! \neq \! 0$
		with $x \! \cdot \! 1 = x$ for all $x$.
		\item[(M3)] inverse element: For all $x \neq 0$ there is a (unique) $y$
		with $x \cdot y = 1$. We write for this element simply $x^{-1}$.
		\item[(M4)] commutative: $x \cdot y = y \cdot x$
	\end{enumerate}
	\item[(D)]  Distributivity: $x \cdot (y + z) = x \cdot y + x \cdot z$.
	\item[(O)]  Ordering
		\begin{enumerate}[(1)]
			\item[(O1)] for given $x,y$ exactly one of the following three
			assertions is true: $x<y$, $y<x$, $x=y$.
			\item[(O2)] transitive: $x<y$ and $y<z$ imply $x<z$.
			\item[(O3)] $x < y$ implies $x + z < y + z$ for all $z$.
			\item[(O4)] $x < y$ implies $x \cdot z < y \cdot z$ for all $z>0$.
			\item[(O5)] $x>0$ and $\varepsilon>0$ implies $x < \varepsilon + \cdots + \varepsilon$ for
			sufficiently many summands.
		\end{enumerate}
	\item[(C)]  Completeness: Every sequence $(a_n)_{n\in \mathbb{N}}$ with the
	property \emph{For all $\varepsilon > 0$ there is an $N \in \mathbb{N}$
	with $|a_n - a_m| < \varepsilon$ for all $n,m > N$} has a limit.
\end{enumerate}
\end{Proposition}


\begin{Definition}[Field]
Every set $M$
together with two the operations $+ : M \times M \rightarrow M$
and  $\cdot : M \times M \rightarrow M$
that fulfil (A), (M) and (D) is called a \emph{field}.
\end{Definition}


\subsection*{Rational versus real numbers}


For most practical purposes the rational numbers (all fractions)
\[
 \mathbb{Q} = \left\{ x : x = \frac{n}{d} \text{ with }  n \in \mathbb{Z}, ~ d \in \mathbb{N} \right\}
\]
are enough. All numbers that can somehow be stored sensibly on a computer are rational. 

\white{5cm}{
But not all quantities can be written as a fraction, such as the zeros of the following function:
\[
 f(x)=x^2-2
\]
We can, however, approximate these ``numbers'' (we cannot call it a number, yet) to arbitrary precision in $\mathbb{Q}$. 

How to finally arrive at an explanation of what this number really is (we cannot just write it down) is a topic of analysis (see next semester!)

Now we just give it a name, namely $\pm \sqrt{2}$, and remark that the real numbers $\mathbb{R} \supset \mathbb{Q}$ is 
a larger set of numbers, that can all be approximated by $\mathbb{Q}$, and the other way round: if something can be approximated to arbitrary precision by rational numbers, it is in $\mathbb{R}$ ``by definition''. 
}

Mathematicians say: $\mathbb{R}$ is complete, $\mathbb{Q}$ is dense in $\mathbb{R}$, $\mathbb{R}$ is the completion of $\mathbb{Q}$. 

Real numbers are like public transport. While quite easy to use, it is not easy to understand in detail why and how it works. (The only difference: real numbers can be used reliably). 

%\subsection*{Order relations}








