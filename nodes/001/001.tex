\usepackage{amsthm}

\newtheorem{theorem}{Theorem}[chapter]
\newtheorem{lemma}           [theorem] {Lemma}   
\newtheorem{folg}           [theorem] {Folgerung}   

\newtheorem{frage}       [theorem] {Frage}   
\newtheorem{question}       [theorem] {Question}   
\newtheorem{aufgabe}       [theorem] {Aufgabe}   
\newtheorem{exercise}       [theorem] {Exercise}  

\newtheorem{proposition}     [theorem] {Proposition}  
\newtheorem{satz}     [theorem] {Satz}  
\newtheorem{fact}{Fact}
\newtheorem{definition}      [theorem] {Definition} 

\theoremstyle{definition} 
\newtheorem{bemerkung}     [theorem] {Bemerkung}  
\newtheorem{beispiel}       [theorem] {Beispiel}  
\newtheorem{example}       [theorem] {Example}  
\newtheorem*{example*} {Example}  
\newtheorem{notation}       [theorem] {Notation}  
\newtheorem*{Faust}[theorem]{Rule of Thumb}
\newtheorem*{Boxx}[theorem]{Concept}

%\subsection*{Sets}
%
Modern mathematics does not say what sets are, but only specifies rules. This is, however, too difficult for us right now, 
and we rather cite the attempt of a definition by Georg Cantor:

\begin{Boxx}{}
``Unter einer ‚Menge‘ verstehen wir jede Zusammenfassung von bestimmten wohlunterschiedenen Objekten unserer Anschauung oder unseres Denkens zu einem Ganzen.''
\end{Boxx}

\begin{Definition}[Set{,} element]
A \emph{set} is a collection into a whole of definite, distinct objects of our perception or of our thought.
Such an object $x$ of a set $M$ is called an \textit{element} of $M$ 
and one writes $x\in M$. If $x$ is not such an object of $M$, we write $x\not\in M$.
\end{Definition}

A set is defined by giving all its elements $M:=\{1,4,9\}$.

\begin{Boxx}{}
The symbol ``$:=$'' is read as \emphblue{defined by} and
means that the symbol $M$ is newly introduced as a set by the given elements.
\end{Boxx}
%

\begin{example}
\begin{itemize}
 \item The empty set $\{\} = \emptyset = \varnothing$
 is the unique set that has no elements at all.
 \item The set that contains the empty set $\{ \varnothing \}$, which is non-empty
 since it has exactly one element.
 \item A finite set of numbers is $\{ 1,2,3\}$.
\end{itemize}
\end{example}

\begin{notation}
Let $A,B$ be sets:
\begin{itemize}
 \item $x \in A$ means $x$ is an element of $A$
 \item $x\not\in A$ means $x$ is not an element of $A$
 \item $A \subset B$ means $A$ is a subset of $B$: every element of $A$ is contained in $B$
 \item $A \supset B$ means $A$ is a superset of $B$: every element of $B$ is contained in $A$
 \item $A=B$ means $A \subset B \wedge A \supset B$. 
 Note that the order of the elements does not matter in sets.
 If we want the order to matter, we rather define \emph{tuples}:
$(1,2,3) \neq (1,3,2)$. For sets, we always have $\{ 1,2,3 \} = \{1, 3,2\}$.
 \item $A \subsetneq B$ means $A$ is a ``proper'' subset of $B$, every element of $A$ is contained in $B$, but $A \neq B$. 
 \end{itemize}
\end{notation}

\begin{Boxx}[The important number sets]
\begin{itemize}\itemsep0mm
    \item $\mathbb{N}$ is the set of the natural numbers $1,2,3,\ldots$\index{natural numbers}; 
    \item $\mathbb{N}_0$ is the set of the natural numbers and zero: $0, 1, 2, 3,\dots$;
    \item $\mathbb{Z}$ is the set of the integers, which means  $\ldots,-3,-2,-1,0,1,2,3,\ldots$;
\item $\mathbb{Q}$ is the set of the rational numbers, which means 
  all fractions $\frac pq$ with $p\in\mathbb{Z}$ and $q\in\mathbb{N}$;
\item $\mathbb{R}$ is the set of the real numbers.
\end{itemize}
\end{Boxx}

Other ways to define sets:
\begin{align*}
 A &= \{ n \in \mathbb{N} : 1 \le n \le 300 \}\\
 \mathbf{P}(B) &= \{ M : M \subset B \} \mbox{ power set: set of all subsets of } B\\
 I &= \{ x\in \mathbb{R} : 1 \le x < \pi \} = [1,\pi) \mbox{ half-open interval }
\end{align*}
%

\begin{Definition}[Cardinality]
We use vertical bars $|\cdot|$ around a set
to denote the number of elements.
For example, we have $|\{1,4,9\}|=3$. The number of elements is called the \emph{cardinality} of the set. 
\end{Definition}

\begin{example}
 $|\{1,3,3,1\}|=2$, $ \quad |\{1,2,3, \ldots, n \}| = n$,
  $ \quad | \mathbb{N} | = \infty$ (?)
\end{example}

% \begin{exercise}[Which of the following logical statements are true?]
% $$
%   \corr{3\in \mathbb{N}}, \qquad \corr{12034\in\mathbb{N}}, \qquad \false{-1\in\mathbb{N}}, \qquad \false{0\in\mathbb{N}}, \qquad \corr{0\in\mathbb{N}_0}
% $$
% $$
%   \corr{-1\in\mathbb{Z}}, \qquad \false{0\notin\mathbb{Z}}, \qquad \false{-2.7\in\mathbb{Z}}, \qquad \false{\tfrac23\in\mathbb{Z}},
% $$
% $$
%   \corr{\tfrac23\in\mathbb{Q}}, \qquad \corr{-3\in\mathbb{Q}}, \qquad \corr{-2.7\in\mathbb{Q}}, \qquad \false{\sqrt{2}\in\mathbb{Q}},
% $$
% $$
% \corr{\sqrt{2}\in\R}, \qquad \false{\sqrt{-2}\in\R}, \qquad \corr{-\tfrac23\in\R}, \qquad \corr{0\in\R}.
% $$
% \end{exercise}

