\usepackage{amsthm}

\newtheorem{theorem}{Theorem}[chapter]
\newtheorem{lemma}           [theorem] {Lemma}   
\newtheorem{folg}           [theorem] {Folgerung}   

\newtheorem{frage}       [theorem] {Frage}   
\newtheorem{question}       [theorem] {Question}   
\newtheorem{aufgabe}       [theorem] {Aufgabe}   
\newtheorem{exercise}       [theorem] {Exercise}  

\newtheorem{proposition}     [theorem] {Proposition}  
\newtheorem{satz}     [theorem] {Satz}  
\newtheorem{fact}{Fact}
\newtheorem{definition}      [theorem] {Definition} 

\theoremstyle{definition} 
\newtheorem{bemerkung}     [theorem] {Bemerkung}  
\newtheorem{beispiel}       [theorem] {Beispiel}  
\newtheorem{example}       [theorem] {Example}  
\newtheorem*{example*} {Example}  
\newtheorem{notation}       [theorem] {Notation}  
\newtheorem*{Faust}[theorem]{Rule of Thumb}
\newtheorem*{Boxx}[theorem]{Concept}

\begin{Definition}{}
Let $I$ be an interval and $f:I\to\mathbb{R}$ be continuous. Then a
differentiable function $F:I \to\mathbb{R}$ is called an \emph{antiderivative of $f$} if $F'=f$.
\end{Definition}

\begin{Theorem}{}
\label{thm:antider}
Let $I$ be an interval, $f:I\to\mathbb{R}$ be continuous and $a\in I$. For $x\in I$ define
\[F(x)=\int_a^xf(\xi)d\xi.\]
Then $F$ is differentiable and an antiderivative of $f$.
\end{Theorem}
{\em Proof:} Let $x\in I$ and $h\neq0$ such that $x+h\in I$. Then, by using the mean value theorem of integration we obtain
\[
\begin{aligned}
\frac1h \left(F(x+h)-F(x)\right)=&\;\frac1h\left(\int_a^{x+h}f(\xi)d\xi-\int_a^xf(\xi)d\xi\right)\\
=&\frac1h\int_x^{x+h}f(\xi)d\xi=\frac1h\cdot h f(\hat{x})=f(\hat{x})
\end{aligned}
\]
for some $\hat{x}$ between $x$ and $x+h$. If $h$ tends to $0$ then $\hat{x}\to x$ and thus
\[\lim_{h\to0}\frac1h \left(F(x+h)-F(x)\right)=f(x).\]
This shows the desired result.$\Box$

Now we consider how two antiderivatives of a~given continuous $f:I\to\mathbb{R}$ differ.
\begin{Theorem}{}\label{thm:antider_diff}
Let $I$ be an interval, $f:I\to\mathbb{R}$ be given and let $F:I\to\mathbb{R}$ be an antiderivative of $f$, i.e., $F'=f$. Then $G:I\to\mathbb{R}$ is an antiderivative of $f$ if and only if $F-G$ is constant.
\end{Theorem}
{\em Proof:}
``$\Rightarrow$'': Let $G:I\to\mathbb{R}$ be an antiderivative of $f$. Then \[(F-G)'=F'-G'=f-f=0\] and hence, $F-G$ is constant due to the mean value theorem of differentiation.\\
``$\Leftarrow$'': If $F-G$ is constant, i.e.,  $F(x)-G(x)=c$ for some $c\in\mathbb{R}$ and all $x\in I$, then $0=(F-G)'=F'-G'$ and thus $f=F'=G'$.
$\Box$


\begin{Remark}{}
It is very important to note that the statement of Theorem~\ref{thm:antider_diff} is only valid for functions defined on intervals. For instance, consider the function $f:\mathbb{R}\backslash\{0\}\to\mathbb{R}$ with $f(x)=\frac1x$. An antiderivative is given by
$F:\mathbb{R}\backslash\{0\}\to\mathbb{R}$ with
\[F(x)=\begin{cases}\log(x)&:x>0,\\\log(-x)&:x<0\end{cases}=\log(|x|).\]
Another antiderivative is given by
\[G(x)=\begin{cases}\log(x)&:x>0,\\\log(-x)+1&:x<0\end{cases}.\]
The difference between $G$ and $F$ is given by
\[G(x)-F(x)=\begin{cases}0&:x>0,\\1&:x<0\end{cases}\]
and therefore not constant.
\end{Remark}
