\usepackage{amsthm}

\newtheorem{theorem}{Theorem}[chapter]
\newtheorem{lemma}           [theorem] {Lemma}   
\newtheorem{folg}           [theorem] {Folgerung}   

\newtheorem{frage}       [theorem] {Frage}   
\newtheorem{question}       [theorem] {Question}   
\newtheorem{aufgabe}       [theorem] {Aufgabe}   
\newtheorem{exercise}       [theorem] {Exercise}  

\newtheorem{proposition}     [theorem] {Proposition}  
\newtheorem{satz}     [theorem] {Satz}  
\newtheorem{fact}{Fact}
\newtheorem{definition}      [theorem] {Definition} 

\theoremstyle{definition} 
\newtheorem{bemerkung}     [theorem] {Bemerkung}  
\newtheorem{beispiel}       [theorem] {Beispiel}  
\newtheorem{example}       [theorem] {Example}  
\newtheorem*{example*} {Example}  
\newtheorem{notation}       [theorem] {Notation}  
\newtheorem*{Faust}[theorem]{Rule of Thumb}
\newtheorem*{Boxx}[theorem]{Concept}
\subsection*{Logical Statements and Operations}

Basic logic is something, we usually accomplish intuitively right. However, in mathematics we have to define it in an unambiguous way
and it may differ a little bit from the everyday logic.
It is very important and useful to bring into our attention some of the basic rules and notations of logic. 
For Computer Science students, logic is considered in more detail in other courses. 

Let us start with a definition:
\begin{definition}[logical statement{,} proposition]
A \emph{logical statement} (or \emph{proposition})
is a statement, which means a meaningful declarative sentence,
that is either true or false.
\end{definition}

Instead of \emph{true}, one often writes $T$ or $1$
and instead of \emph{false}, one often writes $F$ or $0$.

Not every meaningful declarative fulfils this requirement. 
There are opinions, alternative facts, self-contradictory statements, undecidable statements and so on. In fact, a lot of examples here, outside the mathematical world, work only if we give the words unambiguous definitions
which we will implicitly do.

\begin{example}
Which of these are logical statements?
 	\begin{enumerate}[(a)]
 		\item Hamburg is a city.
 		\item $1 + 1 = 2$.
 		\item The number $5$ is smaller than the number $2$.
 		\item Good morning!
 		\item $x + 1 = 1$.
 		\item Today is Tuesday.
 	\end{enumerate}
\end{example}
%
The last two examples are not logical statements but so-called predicates and will be considered later.
%
For given logical statements, one can form new logical statements with so-called \emph{logical operations}.
In the following, we will consider two logical statements $A$ and $B$.

\begin{definition}[Negation $\neg A$ (``not $A$'')]
 $\neg A$ is true if and only if $A$ is false.
\end{definition}

   \begin{equation}
   \mbox{ Truth table }\qquad 
    \begin{array}{c|c}
     A & \neg A\\ \hline
     T & F\\
     F & T
    \end{array}
   \end{equation}

\begin{example}
What are the negations of the following logical statements?
 	\begin{enumerate}[(a)]
 		\item The wine bottle is full.
 		 \white{2cm}{}
 		\item The number $5$ is smaller than the number $2$.
 		\white{2cm}{}
 		\item All students are in the lecture hall.
 		\white{2cm}{}
 	\end{enumerate}
\end{example}

\begin{definition}[Conjunction $A \wedge B$ (``$A$ and $B$'')]
 $A \wedge B$ is true if and only if both $A$  and $B$ are true.
\end{definition}
 
  
   \begin{equation}
   \mbox{ Truth table }\qquad 
    \begin{array}{cc|c}
     A & B & A \wedge B\\ \hline
     T & T& T\\
     T & F & F\\
     F & T & F\\
     F & F & F
    \end{array}
   \end{equation}

\begin{definition}[Disjunction $A \vee B$ (``$A$ or $B$'')]
   $A \vee B$ is true if and only if at least one of  $A$ or $B$ is true.
\end{definition}
 
\begin{equation}
\mbox{ Truth table }\qquad 
 \begin{array}{cc|c}
  A & B & A \vee B \\ \hline
  T & T& T\\
  T & F & T\\
  F & T & T\\
  F & F & F
 \end{array}
\end{equation}
