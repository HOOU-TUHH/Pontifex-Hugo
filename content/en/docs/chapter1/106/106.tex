\section{Cauchy Sequences}
\begin{Definition}{Cauchy sequences}
A sequence $(a_n)_{n\in\mathbb{N}}$ in $\mathbb{K}$ is called \textit{Cauchy sequence} if for all $\varepsilon>0$, there exists some $N$ such that for all $n,m\geq N$ holds
\[|a_n-a_m|<\varepsilon.\]
\end{Definition}
\begin{Remark}{}
By the expression ``$n,m\geq N$'', we mean that both $n$ and $m$ are greater or equal than $N$, i.e., $n\geq N$ \underline{and} $m\geq N$.
\end{Remark}

Now we show that convergent sequences are indeed Cauchy sequences.
\begin{Theorem}{}\label{thm:convcauch}
Let $(a_n)_{n\in\mathbb{N}}$ be a~convergent sequence. Then $(a_n)_{n\in\N}$ is a~Cauchy sequence.
\end{Theorem}
\white{4cm}{
{\em Proof:}
Let $a=\lim_{n	o\infty}a_{n}$ and $\varepsilon>0$. Then there exists some $N$ such that for all $k\geq N$ holds $|a-a_k|<\frac{\eps}2$. Hence, for all $m,n\geq N$ holds
\[|a_n-a_m|=|(a_n-a)+(a-a_m)|\leq|a_n-a|+|a-a_m|< \frac{\varepsilon}2+\frac{\eps}2=\eps.\]
\hfill$\Box$
}

The following theorem is closely related to Theorem~\ref{thm:convseqbnd}.
\begin{Theorem}{Cauchy sequences are bounded}\label{thm:cauchseqbnd}
Let $(a_n)_{n\in\mathbb{N}}$ be a~Cauchy sequence. Then $(a_n)_{n\in\N}$ is bounded.
\end{Theorem}
\white{8cm}{
{\em Proof:} Take $\varepsilon=1$. Then there exists some $N$ such that for all $n,m\geq N$ holds $|a_n-a_m|<1$. Thus, for all $n\geq N$ holds
\[|a_n|=|a_n-a_N+a_N|\leq |a_n-a_N|+|a_N|<1+|a_N|.\]
Now choose
\[c=\max\{|a_1|,|a_2|,\ldots,|a_{N-1}|,|a_N|+1\}\]
and consider some arbitrary sequence element $a_k$.\\
If $k<N$, we have that $|a_k|\leq \max\{|a_1|,|a_2|,\ldots,|a_{N-1}|\}\leq c$.\\
If $k\geq N$, we have, by the above calculations, that $|a_k|<|a_N|+1\leq c$.\\
Altogether, this implies that $|a_k|\leq c$ for all $k\in\mathbb{N}$, so $(a_n)_{n\in\N}$ is bounded by $c$.\hfill$\Box$
}

Now we show that Cauchy sequences in $\mathbb{K}$ are even convergent:
\begin{Theorem}{}\label{thm:Rcompl}
Every Cauchy sequence $(a_n)_{n\in\mathbb{N}}$ in $\mathbb{K}$ converges.
\end{Theorem}

{\em Proof:} 
By Theorem~\ref{thm:cauchseqbnd}, $(a_n)_{n\in\mathbb{N}}$ is bounded. By Theorem \ref{thm:bzr} of Bolzano-Weierstra\ss \ it has a convergent 
subsequence $(a_{n_k})_{k\in\mathbb{N}}$. Set $a:=\lim_{k\rightarrow\infty} a_{n_k}$. For given $\varepsilon>0$ there exist $N_1,N_2\in\N$ such 
that $|a_{n_k}-a|<\varepsilon/2$ for all $k\geq N_1$ and $|a_n-a_m|<\eps/2$ for all $n,m\geq N_2$. Thus for $n\geq N:=\max\{N_1,N_2\}$ holds
$n_n\geq n\geq N$ and $$|a_n-a|\leq|a_n-a_{n_n}+a_{n_n}-a|\leq|a_n-a_{n_n}|+|a_{n_n}-a|< \varepsilon/2+\eps/2=\eps \ . \qquad\Box$$


Theorem \ref{thm:Rcompl} is not true for arbitrary normed $\mathbb{K}$-vector spaces. 
Those normed $\mathbb{K}$-vector spaces $(V,||\cdot||)$ for which every Cauchy sequence has a limit in $V$ are called complete
or Banach spaces (in honour of the Polish mathematician Stefan Banach). 
Without proof we state that all finite dimensional normed $\mathbb{K}$-vector spaces are Banach spaces.

The next result concerns the special property of the real numbers that supremum and infimum are defined for all subsets of the real numbers. This theorem goes back to \textsc{Julius Wilhelm Richard Dedekind} (1831--1916). It follows from the completeness axiom (C):
\begin{Theorem}{Dedekind's Theorem}
\label{thm:bndmonseq}
 Every non-empty bounded set $M\subset \mathbb{R}$ has a~supremum and an~infimum with $\sup M,\inf M\in\R$.
\end{Theorem}


We make essential use of Dedekind's theorem to prove the following result:
\begin{Theorem}{Convergence of bounded and monotonic sequences}\label{thm:monbndseq}
 Let $(a_n)_{n\in\mathbb{N}}$ be a~real sequence that has one of the following properties:
\begin{itemize}
 \item[--] $(a_n)_{n\in\mathbb{N}}$ is monotonically increasing and bounded from above;
 \item[--] $(a_n)_{n\in\mathbb{N}}$ is monotonically decreasing and bounded from below;
\end{itemize}
Then $(a_n)_{n\in\mathbb{N}}$ is convergent.
\end{Theorem}
{\em Proof:} Let us first assume that $(a_n)_{n\in\mathbb{N}}$ is monotonically increasing and bounded from above. Define the set $M=\{a_n\,:\,n\in\N\}$. 
\white{6cm}{Since $M$ is bounded, Dedekind's theorem implies that there exists some $K\in\mathbb{R}$ such that \[K=\sup M.\]
%
%
We show that $K$ is indeed the limit of the sequence $(a_n)_{n\in\mathbb{N}}$.\\
Let $\varepsilon>0$. By the definition of the supremum, we have that $a_n\leq K$ for all $n\in\mathbb{N}$ and there exists some $N\in\N$ such that $a_N>K-\eps$. The monotonicity of $(a_n)_{n\in\N}$ implies that for all $n\geq N$ holds $a_N\leq a_n$. Altogether, we have \[K-\eps<a_N\leq a_n\leq K\]
and thus $|K-a_n|=K-a_n<\varepsilon$ for all $n\geq N$. This implies convergence to $K$.\\
To prove that convergence is also guaranteed in the case where $(a_n)_{n\in\mathbb{N}}$ is monotonically decreasing and bounded from below, we consider the sequence $(-a_n)_{n\in\N}$, which is now bounded from above and monotonically increasing. By the (already proven) first statement of this theorem, the sequence $(-a_n)_{n\in\N}$ is convergent, whence $(a_n)_{n\in\N}$ is convergent as well.\hfill$\Box$
}

\begin{Remark}{}\label{rem:monseqgen}
 By the same argumentation as in Remark from page \pageref{rem:n_0mon}, the monotone increase (decrease) of $(a_n)_{n\in\mathbb{N}}$ can be slightly relaxed by only claiming that $a_n\leq a_{n+1}$ ($a_n\geq a_{n+1}$) for all $n\geq n_0$ for some $n_0$ in $\N$. In such a~case, the limit of the sequence is then given by $\sup\{a_n\,:\,n\geq n_0\}$ (resp.\ $\inf\{a_n\,:\,n\geq n_0\}$).
\end{Remark}

\begin{example}{}\label{ex:monseqconv}
 \begin{enumerate}[a)]
  \item Consider the sequence $(a_n)_{n\in\mathbb{N}}$ which is recursively defined via $a_1=1$ and 
  \[a_{n+1}=\frac{a_n+\frac2{a_n}}{2}\;\text{  for $n\geq 1$.}\] 
  We now prove that this sequence is convergent by showing that
it is bounded from below and for all $n\geq2$ holds $a_{n+1}\leq a_n$.\\[2ex]
{\em Proof:} To show boundedness from below, we use the inequality $\sqrt{xy}\leq\frac{x+y}2$ for all nonnegative $x,y\in\mathbb{R}$. This inequality is a~consequence of
\white{3cm}{
\[0\leq\frac{(\sqrt{x}-\sqrt{y})^2}2=\frac{x+y}2-\sqrt{xy}.\]
The first inequality is a consequence of the fact that squares of real numbers cannot be negative.\\
}
Using this inequality, we obtain for $n\geq1$
\[a_{n+1}=\frac{a_n+\frac{2}{a_n}}{2}\geq \sqrt{a_n\cdot\frac2{a_n}}=\sqrt{2}.\]
Thus, $(a_n)$ is bounded from below. For showing monotonicity, we consider
\[a_{n+1}-a_n=\frac{a_n+\frac2{a_n}}{2}-a_n=\frac{1}{2a_n}(2-a_n^2).\]
In particular, if $n\geq2$, we have that $a_n>0$ and $2-a_n^2\leq0$. Thus, $a_{n+1}-a_n\leq0$ for $n\geq2$. An~application of Theorem~\ref{thm:bndmonseq} (resp.\ the slight generalisation in Remark %\ref{rem:monseqgen}
from above) 
now leads to the existence of some $a\in\mathbb{R}$ with $a=\lim_{n	o\infty}a_n$.\\[2ex]
To compute the limit, we make use of the relation $\lim_{n	o\infty}a_n=\lim_{n\To\infty}a_{n+1}$ (follows directly from Definition~\ref{def:convlim}) and the formulae for limits in Theorem~\ref{thm:limformnormed}. This yields
\[a=\lim_{n	o\infty}a_n=\lim_{n\To\infty}a_{n+1}=\lim_{n\To\infty}\frac{a_n+\frac2{a_n}}{2}=\frac{a+\frac2{a}}{2}.\]
\white{3cm}{
This relation leads to the equation $2-a^2=0$, i.e., we either have $a=\sqrt{2}$ or $a=-\sqrt{2}$. However, the latter solution cannot be a limit since all sequence elements are positive. Therefore, we have
\[\lim_{n	o\infty}a_n=\sqrt{2}.\]
}
\item Let $x\in\mathbb{R}$ with $x>1$. Consider the sequence $(\sqrt[n]{x})_{n\in\mathbb{N}}$. It can be directly seen that $(\sqrt[n]{x})_{n\in\N}$ is monotonically decreasing and bounded from below by one. Therefore, the limit
\[a=\lim_{n	o\infty}\sqrt[n]{x}\]
exists with $a\geq1$. To show that $a=1$, we assume that $a>1$ and lead this to a~contradiction.\\
The equation $a>1$ leads to the existence of some $n\in\mathbb{N}$ with $a^n>x$, and thus $a>\sqrt[n]{x}$. On the other hand, the monotone decrease of $(\sqrt[n]{x})_{n\in\N}$ implies that
\[a=\lim_{n	o\infty}\sqrt[n]{x}=\inf\{\sqrt[n]{x}\,:\,n\in\mathbb{N}\}<a,\]
which is a~contradiction.
\item Let $x\in\mathbb{R}$ with $0< x<1$. Consider the sequence $(a_n)_{n\in\mathbb{N}}=(\sqrt[n]{x})_{n\in\N}$. Then we have by Example b) and Theorem~\ref{thm:limformnormed} that
\[\lim_{n	o\infty}\sqrt[n]{x}=\frac1{\lim_{n\To\infty}\sqrt[n]{\frac1x}}=\frac11=1.\]
\item Let $(a_{n})$ be a nonnegative sequence with $a_{n}\rightarrow a$ and $k\in\mathbb{N}$. Then for all $\varepsilon>0$ there exists $N>0$ such that $|a_{n}-a|<\varepsilon^{k}$. From this it follows that
$$
|\sqrt[k]{a_{n}}-\sqrt[k]{a}|\leq \sqrt[k]{|a_{n}-a|}<\varepsilon.
$$
Thus $(\sqrt[k]{a_{n}})$ is convergent with limit $\sqrt[k]{a}$.\\
\item The sequence $(a_{n})_{n\in\mathbb{N}}$ defined as $a_{n} := \left(1+\frac{1}{n}\right)^{n}$ is convergent.\\
\end{enumerate}
\end{example}

\begin{Remark}{}
The limit of the sequence \[(a_n)_{n \in \mathbb{N}N} = \left(  \Big(1+\frac{1}{n} \,\Big)^{n}\right)_{n \in \NN},\] 
i.e. $e:=\lim_{n\rightarrow\infty}\left(1+\frac{1}{n}\right)^{n}$ is well known as \textit{Euler's number}. Later on we will define the exponential function $\exp$. It holds that $e=\exp(1)\approx 2.7182818...$ . Indeed, we will show later on that $e^{z}=\lim_{n\rightarrow\infty}\left(1+\frac{z}{n}\right)^{n}=\exp(z)$.
\end{Remark}
