\section{Natural numbers and induction}
The natural numbers are $\NN = \{1,2,3 \ldots \}$.

\white{}{
Using natural numbers is our first mathematical abstraction. We learn this as children in the kindergarden. 

What is this abstraction? A number is an abstraction for all finite sets of the same size.
}

\begin{itemize}
 \item \emph{Question 1:} When are two sets $S,T$ of the same size?  Have the same \emph{cardinality} $|S|=|T|$?
 \emph{Answer:} They have the same size
 if there is a bijective map $S\to T$.
 \white{3cm}{For example, $\NN$ and the set of all even numbers have the same cardinality.}
 \item \emph{Question 2:} When is a set $S$ finite? \emph{Answer:}
 It is finite if removing one element changes the \emph{cardinality} of $S$. 
 \white{3cm}{}
\end{itemize}

In mathematical language: ``Natural numbers are equivalence classes of finite sets of the same cardinality.'' 


\subsubsection{Mathematical induction}

Mathematical induction is an important technique of proof: Proof step by step. It is a close relative to 
recursion in computer science. 

\white{}{
``Assume I can solve a problem of size $n$. How can I solve one of size $n+1$?''
}


In mathematics:

``If an assertion is true for $n$, show that it is true for $n+1$''


\white{3cm}{}


\begin{example}
What is the sum of the first $n$ natural numbers?
\[
 s_n := \sum_{k=1}^n k = \,?
\]


To make this practical, we need three ingredients:
\begin{itemize}
 \item[(i)] An idea what the result could be. (Induction hypothesis)
 \item[(ii)] The verification that our hypothesis is true for $n=1$ (Base case)
 \item[(iii)] A proof, that if it holds for $n$, then also for $n+1$. (Induction step)
\end{itemize}
\white{5cm}{
Getting the first ingredient is often the most difficult one. Often one has to try it out:
\begin{align*}
 s_1 = 1\\
 s_2 = s_1+2 = 3\\
 s_3 = s_2+3 = 6\\
 s_4 = s_3+4=10\\
 s_5 = s_4+5=15\\
 s_{n+1}=s_n+n+1
\end{align*}
}
Ideas? Let's take the hypothesis
\[
 s_n = \frac{(n+1)n}{2}\qquad  \mbox{ (Induction hypothesis). }
\]
Very good! We can verify our formula for these examples. In particular:
\[
 s_1 = \frac{(1+1)1}{2}=1 \qquad  \mbox{ (Base case). }
\]
\emph{Induction step:} We have to show
\[
 \frac{(n+2)(n+1)}{2} \mbox{ is equal to } s_{n+1}=s_n + (n+1)
 = \frac{(n+1)n}{2}+n+1
\]
where we used the induction hypothesis in the last step.
So let us compute:
\white{4cm}{
\begin{align*}
 s_n + (n+1) 
 = \frac{(n+1)n}{2}+n+1&=\frac{n^2+n+2n+2}{2} =
 \frac{(n+2)(n+1)}{2}
 \,.
\end{align*}
This proves that $s_n = \frac{(n+1)n}{2}$ for all $n \in \NN$.}
\end{example}


We will get plenty of other examples later. 

\begin{Faust}{Mathematical induction}
 To show that the predicate $A(n)$ is true \emphblue{for all} $n \in \NN$,
 we have to show two things:
 	\begin{enumerate}[(1)]
 		\item Show that $A(1)$ is true.
 		\item Show that $A(n+1)$ is true under the assumption that $A(n)$
 		is true.
 	\end{enumerate}
\end{Faust}

\white{4cm}{
Sometimes it can happen that a claim $A(n)$ is indeed false
for finitely many natural numbers, but it is eventually true. This means
that the base case cannot be shown for $n=1$ but
for some other natural number $n_0 \in \NN$. Then the induction proof
shows that $A(n)$ is true for all natural number $n \geq n_0$.
}
% 
% \subsubsection{A very important algorithmic idea of induction:}
% 
% Consider a large $n\times n$ matrix:
% \[
%  A=\left(
%  \begin{array}{ccc}
%   a_{11} & \dots & a_{1n}\\
%   \vdots &  & \vdots\\
%   a_{n1} & \dots & a_{nn}
%   \end{array}
% \right)
% \]
% Suppose, we have a clever algorithm (Alg. Column) that:
% \begin{itemize}
%  \item can delete all entries of a column of an arbitrary matrix $A$, 
% except for the first
% \item maybe some other useful property
% \end{itemize}
% Example: a few steps of Gauss elimination. 
% 
% So we can compute:
% \[
%  A=\left(
%  \begin{array}{ccc}
%    a_{11} & \dots & a_{1n}\\
%    \vdots &  & \vdots\\
%    a_{n1} & \dots & a_{nn}
%   \end{array}
% \right)\to \hat A
% =\left(
%  \begin{array}{cccc}
%    \hat a_{11} &\hat a_{12}& \dots & \hat a_{1n}\\
%    0 & \hat a_{22} & \dots & \hat a_{2n}\\
%    \vdots & \vdots & & \vdots\\
%    0 &\hat a_{n2}& \dots & \hat a_{nn}
%   \end{array}
% \right)
% \]
% Then we can put any square $n\times n$ matrix $A$ into triangular form:
% \[
%  R = \left(
%  \begin{array}{ccc}
%    r_{11} & \dots & r_{1n}\\
%    0 & \ddots & \vdots\\
%    0 & 0 & r_{nn}
%   \end{array}
% \right)
% \]
% Induction hypothesis: we have an algorithm that can put $A \to R$, where $A$ is an $n\times n$ matrix. .
% 
% Base case: $n=1$: we have to do nothing, because $A=a_{11}$ is already ''triangular``.
% 
% \emph{Recursive algorithm:} (Alg. Rec.)
% 
% Suppose, we have $A$ of the size $n+1\times n+1$. Do the following algorithm:
% 
% \begin{itemize}
%  \item Apply (Alg. Column) to $A$, deleting the entries of its the first non-zero column (except for the first) $\to \hat A$:
% \[
%  \hat A
% =\left(
%  \begin{array}{cccc}
%    \hat a_{11} &\hat a_{12}& \dots & \hat a_{1n}\\
%     0 & & \tilde A & \\
%   \end{array}
% \right)
% \]
%  \item Apply (Alg. Rec.) to the matrix $\tilde A$ which results from deleting the first row and column
% \[
%  R=\left(
%  \begin{array}{cccc}
%    \hat a_{11} &\hat a_{12}& \dots & \hat a_{1n}\\
%     0 & & \tilde R & \\
%   \end{array}
% \right)
% \]
% \end{itemize}
% 
% Application: the famous Gauss-Algorithm for the solution of linear systems of equations is such a kind of algorithm.
% 
% Triangular systems $Rx=b$ can easily be solved by substitution.
% 

%%%%%%%%%%%%%%%%%%%%%%%%%%%%%%%%%%%%%%%%%%%%%%%%%%%%%%%%%%%%%%%%%%%%%%%%%%%%%%%%%%%%%%%%%%%%






\section{Summary}
\begin{itemize}
	\item For doing Mathematics, we need logic and sets. A set
	is just a gathering of its elements.
\item Important symbols: $\in,\ \not\in,\ \varnothing,\ \forall,\ \exists,\ \subset,\ \subsetneq,\ \cap,\ \cup,\ \setminus$
\item Implication $A\Rightarrow B$: If $A$ holds, then also $B$.
\item Equivalence $A\Leftrightarrow B$: The statement $A$
holds if and only if $B$ holds.
\item Sums and products $\Sigma, \Pi$
\item A \textsl{map} or \textsl{function} $f:X\to Y$ 
sends each $x\in X$ to exactly one $y\in Y$.
\white{40mm}{}
\item $f$ is  \textsl{surjective}: Each $y\in Y$ is ``hit'' (one or more times).
\item $f$ is  \textsl{injective}: Each $y\in Y$ is ``hit'' at most one time.
\item $f$ is \textsl{bijective}: Each $y\in Y$ is  ``hit'' exactly once.
\item Is $f:X\to Y$ bijective, then the \textsl{inverse map} $f^{-1}:Y\to X$ sends each $y\in Y$ to exactly one $x\in X$.
\item The composition $g\circ f:X\to Z$ is the application 
of the function $g:Y\to Z$ to the result of another function $f:X\to Y$:
$(g\circ f)(x)=g(f(x))$.
\item Mathematical induction is a tool for proving mathematical statements for
all natural numbers at once. You have to show a base case and then do the induction step.
\end{itemize}


\section{Exercises}
\subsection*{Exercise 1}
	Calculate the following numbers and sets:
	\begin{multicols}{4}
		\begin{enumerate}[(a)]
			\item
			$\prod\limits_{j=2}^{4} \frac{j}{j+1}$, 
			
			\item
			$\sum\limits_{i=0}^{4} 3$, 
			
			\item
			$\bigcup\limits_{n=0}^{5} [2n, 2n+2)$, 
			
			\item
			$\sum\limits_{k=1}^{50}k$. 
		\end{enumerate}
	\end{multicols}

\subsection*{Exercise 2}
\begin{enumerate}[(a)]
	\item
	Consider the two functions $f_1:\, \R \to \R$, $x \mapsto x^2$ and $f_2:\, [0, \infty) \to \R$, $x \mapsto x^2$. 
	For both functions calculate preimages of the sets $\{1\}$, $[4,9)$ and $(-1,0)$. 
	
	\item
	Consider the two functions $g_1:\, \R \to [0,1]$, $x \mapsto \abs{\sin (x)}$ and $g_2:\, [0, 2\pi] \to \R$, $x \mapsto \sin (x)$. 
	For both functions calculate images of the sets $(0, \pi/2)$, $[0,\pi)$ and $(0,2\pi]$.
	
	\item
	Consider the two functions $h_1:\, \R \to \R$ and $h_2:\, [-1, 1] \to [\sqrt{3},2]$ given by
	\[
	x = \left(h_1(x)+2\right)^2-2 \quad \text{and} \quad x^2 + h_2(x)^2 = 4. 
	\] 
	Check whether $h_1$ and $h_2$ respectively are correctly defined. 
	
	\item
	Consider all 6 functions from above and find out which of them are injective, surjective and bijective. 
	Try to provide proofs and counterexamples. 
\end{enumerate}

\subsection*{Exercise 3}
Let $X$ be the set of all fish in a given aquarium. 
Define a function $f:X \to Y$ by mapping every fish on its species where $Y$ denotes the set of all species of fish. 
What does it mean if $f$ is injective or surjective or bijective? 

\subsection*{Exercise 4}
In the lecture you already learnt about the example $(A \rightarrow B) \; \Leftrightarrow \; (\neg B \rightarrow \neg A)$ of two logically equivalent statements. 
Show that the following statements are also logically equivalent by using truth tables:
\begin{enumerate}[(a)]
	\item
	$\neg (A \wedge \neg B) \; \Leftrightarrow \: (A \rightarrow B),$
	\item
	$\neg (A \wedge B) \; \Leftrightarrow \; \neg A \vee \neg B.$
\end{enumerate}


\subsection*{Exercise 5}
One usually deals with subsets $A$, $B$, etc.\ of a given fixed set $X$. 
In such a situation it is useful to introduce $A^{\text{c}} := X \setminus A$ which is called the \emph{complement} of $A$ (with respect to (w.r.t.) the set $X$). 
Show for $A,B \subset X$
\begin{enumerate}[(a)]
	\item
	$A \setminus B = A \cap B^{\text{c}}$, 
	
	\item
	$(A \cap B)^{\text{c}} = A^{\text{c}} \cup B^{\text{c}}$. 
\end{enumerate}


\subsection*{Exercise 6}
Let $A,B$ and $C$ be sets. 
\begin{enumerate}[(a)]
	\item
	Show $A \times (B \cup C) = (A \times B) \cup (A \times C)$. 
	
	\item
	Let $\abs{A}=n$ and $\abs{B}=m$ where $n,m \in \N$. 
	Show that
	\[
	\abs{A \times B} = n \cdot m. 
	\]	
\end{enumerate}



