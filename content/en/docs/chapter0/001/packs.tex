\usepackage{amsthm}

\newtheorem{theorem}{Theorem}[chapter]
\newtheorem{lemma}           [theorem] {Lemma}   
\newtheorem{folg}           [theorem] {Folgerung}   

\newtheorem{frage}       [theorem] {Frage}   
\newtheorem{question}       [theorem] {Question}   
\newtheorem{aufgabe}       [theorem] {Aufgabe}   
\newtheorem{exercise}       [theorem] {Exercise}  

\newtheorem{proposition}     [theorem] {Proposition}  
\newtheorem{satz}     [theorem] {Satz}  
\newtheorem{fact}{Fact}
\newtheorem{definition}      [theorem] {Definition} 

\theoremstyle{definition} 
\newtheorem{bemerkung}     [theorem] {Bemerkung}  
\newtheorem{beispiel}       [theorem] {Beispiel}  
\newtheorem{example}       [theorem] {Example}  
\newtheorem*{example*} {Example}  
\newtheorem{notation}       [theorem] {Notation}  
\newtheorem*{Faust}[theorem]{Rule of Thumb}
\newtheorem*{Boxx}[theorem]{Concept}