\usepackage{amsthm}

\newtheorem{theorem}{Theorem}[chapter]
\newtheorem{lemma}           [theorem] {Lemma}   
\newtheorem{folg}           [theorem] {Folgerung}   

\newtheorem{frage}       [theorem] {Frage}   
\newtheorem{question}       [theorem] {Question}   
\newtheorem{aufgabe}       [theorem] {Aufgabe}   
\newtheorem{exercise}       [theorem] {Exercise}  

\newtheorem{proposition}     [theorem] {Proposition}  
\newtheorem{satz}     [theorem] {Satz}  
\newtheorem{fact}{Fact}
\newtheorem{definition}      [theorem] {Definition} 

\theoremstyle{definition} 
\newtheorem{bemerkung}     [theorem] {Bemerkung}  
\newtheorem{beispiel}       [theorem] {Beispiel}  
\newtheorem{example}       [theorem] {Example}  
\newtheorem*{example*} {Example}  
\newtheorem{notation}       [theorem] {Notation}  
\newtheorem*{Faust}[theorem]{Rule of Thumb}
\newtheorem*{Boxx}[theorem]{Concept}
\section{Maps}% \label{sec:map}

\begin{Definition}[Function or map]% \label{Def:Funktion}
Let $X,Y$ be non-empty sets. A rule 
that assigns to 
each \defi{argument} $x\in X$ a unique \defi{value} $y\in Y$
is called a \defi{map} or \defi{function}
from $X$ into $Y$. One writes for this $y$ usually $f(x)$.\\[0.5em]
Notation:\\[-2em]
\begin{align*}
f:X &\rightarrow Y \\
x &\mapsto f(x)
\end{align*}
Here, $X$ is called \emph{domain} of $f$, and $Y$ is called \emph{codomain}. 
\end{Definition} 

\begin{Attention}[Two arrows!]
We use the arrow `` $\to$ '' only between the sets, domain and codomain,
and `` $\mapsto$ '' only between the elements.
\end{Attention}

\begin{example}{} %\label{Bsp:Funktion}
%\begin{itemize}
%\itemsep0.7mm
%\item 
$f:\mathbb{N} \rightarrow \mathbb{N}$ with $f(x)=x^2$ 
maps each natural number to its square.
% \begin{center}
     % \begin{tikzpicture}[scale=0.9]
% %% Mengen:
% \draw (0,0) ellipse (2 and 5/2);
% \draw (5,0) ellipse (2 and 5/2);
% %% linke Menge:
% \node at (-0.7,1.2) {$1$};
% \fill (-0.5,1.2) circle (0.07);
% \node at (0.3,0.6) {$2$};
% \fill (0.5,0.6) circle (0.07);
% \node at (-0.7,0) {$3$};
% \fill (-0.5,0) circle (0.07);
% \node at (0.3,-0.6) {$4$};
% \fill (0.5,-0.6) circle (0.07);
% \node at (-0.7,-1.2) {$5$};
% \fill (-0.5,-1.2) circle (0.07);
% \node at (0,-1.7) {$\dots$};
% %% rechte Menge:
% %Erste Zeile:
% \fill (4,1.2) circle (0.05); \fill (4.9,1.2) circle (0.05); \fill (5.8,1.2) circle (0.05);
% \node at (4.2,1.2) {$1$}; \node at (5.1,1.2) {$2$}; \node at (6,1.2) {$3$};
% % Zweite zeile:
% \fill (3.7,0.72) circle (0.05); \fill (4.3,0.72) circle (0.05); \fill (4.9,0.72) circle (0.05); \fill (5.5,0.72) circle (0.05); \fill (6.1,0.72) circle (0.05);
% \node at (3.9,0.72) {$4$}; \node at (4.5,0.72) {$5$}; \node at (5.1,0.72) {$6$}; \node at (5.7,0.72) {$7$}; \node at (6.3,0.72) {$8$}; 
% % Dritte Zeile:
 % \fill (3.7,0.24) circle (0.05); \fill (4.5,0.24) circle (0.05); \fill (5.3,0.24) circle (0.05); \fill (6.1,0.24) circle (0.05);
% \node at (3.9,0.24) {$9$}; \node at (4.75,0.24) {$10$}; \node at (5.55,0.24) {$11$}; \node at (6.35,0.24) {$12$};
% % Vierte Zeile:
% \fill (4.2,-0.24) circle (0.05); \fill (4.9,-0.24) circle (0.05); \fill (5.6,-0.24) circle (0.05);
% \node at (4.45,-0.24) {$13$}; \node at (5.15,-0.24) {$14$}; \node at (5.85,-0.24) {$15$};
% % Fünfte Zeile:
% \fill (4.7,-0.72) circle (0.05); \node at (5.7,-0.72) {$\dots$};
% \node at (4.95,-0.72) {$16$};
% % Sechste Zeile:
% \fill (4.9,-1.2) circle (0.05);
% \node at (4.9,-1.7) {$\dots$};
% \node at (5.2,-1.2) {$25$};
% %% Beschriftung:
% \node at (2.5,1.5) {$f$};
% \node at (0,2.8) {$X=\mathbb{N}$};
% \node at (5,2.8) {$Y=\mathbb{N}$};
% %% Pfeile:
% \draw[->] (-0.5,1.2) -- (3.9,1.2);
% \draw[->] (0.5,0.6) -- (3.6,0.72);
% \draw[->] (-0.5,0) -- (3.6,0.24);
% \draw[->] (0.5,-0.6) -- (4.6,-0.72);
% \draw[->] (-0.5,-1.2) -- (4.8,-1.2);
% \end{tikzpicture}
% \end{center}
% \item  
% \begin{align*}
 % f : \mathbb{R}^2 &\to \mathbb{R}\\
       % (x_1,x_2) &\mapsto x_1^2 + x_2^2\\
% \end{align*}
% \item
% \begin{align*}
 % f : \mathbb{Z}&\times \mathbb{N} \to \mathbb{Q} \\      
       % (q,p) &\mapsto \frac{q}{p}
% \end{align*}
% \end{itemize}
\end{example}

\subsection*{Well-definedness}
What can go wrong with the definition of a map? 
Sometimes, when defining a function, it is not completely clear, if this makes sense. 
Then one has to work and make this function well-defined. 

\subsubsection{Example: the square-root}

Try to define a map $a \to \sqrt{a}$ in a mathematically rigorous way.

Naive definition:
\begin{align*}
 \sqrt{\hphantom{x}} : \mathbb{R} &\to \mathbb{R}\\
         a &\mapsto \mbox{ the solution of } x^2=a.
\end{align*}
Problem of well-definedness: As we all know, the above equation has two ($a>0$), one ($a=0$), or zero ($a<0$) solutions.

\emph{First way}: restrict the domain of definition and the codomain
\[
 \mathbb{R}^+_0 =  \{ a \in \mathbb{R}: a \ge 0 \}
\]
Then:
\begin{align*}
 \sqrt{\hphantom{x}} : \mathbb{R}^+_0 &\to \mathbb{R}^+_0\\
         a &\mapsto  \mbox{ the non-negative solution of } x^2=a.
\end{align*}
This yields the classical square-root. 
